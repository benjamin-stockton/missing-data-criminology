% Options for packages loaded elsewhere
\PassOptionsToPackage{unicode}{hyperref}
\PassOptionsToPackage{hyphens}{url}
\PassOptionsToPackage{dvipsnames,svgnames,x11names}{xcolor}
%
\documentclass[
  letterpaper,
  DIV=11,
  numbers=noendperiod]{scrartcl}

\usepackage{amsmath,amssymb}
\usepackage{iftex}
\ifPDFTeX
  \usepackage[T1]{fontenc}
  \usepackage[utf8]{inputenc}
  \usepackage{textcomp} % provide euro and other symbols
\else % if luatex or xetex
  \usepackage{unicode-math}
  \defaultfontfeatures{Scale=MatchLowercase}
  \defaultfontfeatures[\rmfamily]{Ligatures=TeX,Scale=1}
\fi
\usepackage{lmodern}
\ifPDFTeX\else  
    % xetex/luatex font selection
\fi
% Use upquote if available, for straight quotes in verbatim environments
\IfFileExists{upquote.sty}{\usepackage{upquote}}{}
\IfFileExists{microtype.sty}{% use microtype if available
  \usepackage[]{microtype}
  \UseMicrotypeSet[protrusion]{basicmath} % disable protrusion for tt fonts
}{}
\makeatletter
\@ifundefined{KOMAClassName}{% if non-KOMA class
  \IfFileExists{parskip.sty}{%
    \usepackage{parskip}
  }{% else
    \setlength{\parindent}{0pt}
    \setlength{\parskip}{6pt plus 2pt minus 1pt}}
}{% if KOMA class
  \KOMAoptions{parskip=half}}
\makeatother
\usepackage{xcolor}
\setlength{\emergencystretch}{3em} % prevent overfull lines
\setcounter{secnumdepth}{2}
% Make \paragraph and \subparagraph free-standing
\ifx\paragraph\undefined\else
  \let\oldparagraph\paragraph
  \renewcommand{\paragraph}[1]{\oldparagraph{#1}\mbox{}}
\fi
\ifx\subparagraph\undefined\else
  \let\oldsubparagraph\subparagraph
  \renewcommand{\subparagraph}[1]{\oldsubparagraph{#1}\mbox{}}
\fi


\providecommand{\tightlist}{%
  \setlength{\itemsep}{0pt}\setlength{\parskip}{0pt}}\usepackage{longtable,booktabs,array}
\usepackage{calc} % for calculating minipage widths
% Correct order of tables after \paragraph or \subparagraph
\usepackage{etoolbox}
\makeatletter
\patchcmd\longtable{\par}{\if@noskipsec\mbox{}\fi\par}{}{}
\makeatother
% Allow footnotes in longtable head/foot
\IfFileExists{footnotehyper.sty}{\usepackage{footnotehyper}}{\usepackage{footnote}}
\makesavenoteenv{longtable}
\usepackage{graphicx}
\makeatletter
\def\maxwidth{\ifdim\Gin@nat@width>\linewidth\linewidth\else\Gin@nat@width\fi}
\def\maxheight{\ifdim\Gin@nat@height>\textheight\textheight\else\Gin@nat@height\fi}
\makeatother
% Scale images if necessary, so that they will not overflow the page
% margins by default, and it is still possible to overwrite the defaults
% using explicit options in \includegraphics[width, height, ...]{}
\setkeys{Gin}{width=\maxwidth,height=\maxheight,keepaspectratio}
% Set default figure placement to htbp
\makeatletter
\def\fps@figure{htbp}
\makeatother
\newlength{\cslhangindent}
\setlength{\cslhangindent}{1.5em}
\newlength{\csllabelwidth}
\setlength{\csllabelwidth}{3em}
\newlength{\cslentryspacingunit} % times entry-spacing
\setlength{\cslentryspacingunit}{\parskip}
\newenvironment{CSLReferences}[2] % #1 hanging-ident, #2 entry spacing
 {% don't indent paragraphs
  \setlength{\parindent}{0pt}
  % turn on hanging indent if param 1 is 1
  \ifodd #1
  \let\oldpar\par
  \def\par{\hangindent=\cslhangindent\oldpar}
  \fi
  % set entry spacing
  \setlength{\parskip}{#2\cslentryspacingunit}
 }%
 {}
\usepackage{calc}
\newcommand{\CSLBlock}[1]{#1\hfill\break}
\newcommand{\CSLLeftMargin}[1]{\parbox[t]{\csllabelwidth}{#1}}
\newcommand{\CSLRightInline}[1]{\parbox[t]{\linewidth - \csllabelwidth}{#1}\break}
\newcommand{\CSLIndent}[1]{\hspace{\cslhangindent}#1}

\KOMAoption{captions}{tableheading}
\makeatletter
\makeatother
\makeatletter
\makeatother
\makeatletter
\@ifpackageloaded{caption}{}{\usepackage{caption}}
\AtBeginDocument{%
\ifdefined\contentsname
  \renewcommand*\contentsname{Table of contents}
\else
  \newcommand\contentsname{Table of contents}
\fi
\ifdefined\listfigurename
  \renewcommand*\listfigurename{List of Figures}
\else
  \newcommand\listfigurename{List of Figures}
\fi
\ifdefined\listtablename
  \renewcommand*\listtablename{List of Tables}
\else
  \newcommand\listtablename{List of Tables}
\fi
\ifdefined\figurename
  \renewcommand*\figurename{Figure}
\else
  \newcommand\figurename{Figure}
\fi
\ifdefined\tablename
  \renewcommand*\tablename{Table}
\else
  \newcommand\tablename{Table}
\fi
}
\@ifpackageloaded{float}{}{\usepackage{float}}
\floatstyle{ruled}
\@ifundefined{c@chapter}{\newfloat{codelisting}{h}{lop}}{\newfloat{codelisting}{h}{lop}[chapter]}
\floatname{codelisting}{Listing}
\newcommand*\listoflistings{\listof{codelisting}{List of Listings}}
\makeatother
\makeatletter
\@ifpackageloaded{caption}{}{\usepackage{caption}}
\@ifpackageloaded{subcaption}{}{\usepackage{subcaption}}
\makeatother
\makeatletter
\@ifpackageloaded{tcolorbox}{}{\usepackage[skins,breakable]{tcolorbox}}
\makeatother
\makeatletter
\@ifundefined{shadecolor}{\definecolor{shadecolor}{rgb}{.97, .97, .97}}
\makeatother
\makeatletter
\makeatother
\makeatletter
\makeatother
\ifLuaTeX
  \usepackage{selnolig}  % disable illegal ligatures
\fi
\IfFileExists{bookmark.sty}{\usepackage{bookmark}}{\usepackage{hyperref}}
\IfFileExists{xurl.sty}{\usepackage{xurl}}{} % add URL line breaks if available
\urlstyle{same} % disable monospaced font for URLs
\hypersetup{
  pdftitle={Sentencing Analysis with Pattern Mixture Modeling},
  pdfauthor={C. Clare Strange; Benjamin Stockton; Jordan Zvonkovich; Ofer Harel},
  pdfkeywords={incomplete data, pattern-mixture model},
  colorlinks=true,
  linkcolor={blue},
  filecolor={Maroon},
  citecolor={Blue},
  urlcolor={Blue},
  pdfcreator={LaTeX via pandoc}}

\title{Sentencing Analysis with Pattern Mixture Modeling}
\author{C. Clare Strange \and Benjamin Stockton \and Jordan
Zvonkovich \and Ofer Harel}
\date{2024-02-22}

\begin{document}
\maketitle
\ifdefined\Shaded\renewenvironment{Shaded}{\begin{tcolorbox}[interior hidden, borderline west={3pt}{0pt}{shadecolor}, breakable, boxrule=0pt, enhanced, frame hidden, sharp corners]}{\end{tcolorbox}}\fi

\hypertarget{methods}{%
\section{Methods}\label{methods}}

We propose using multiple imputation and pattern-mixture models to
perform sensitivity analyses for the race effect estimates on the
jail/prison sentence length under incompleteness of the race variable.
We chose to use hurdle models with a lognormal generalized linear model
for the non-zero sentence lengths. Hurdle models can be used to model
complex data where the dependent variable is a combination of true zeros
and a continuous distribution for non-zero observations. Sentence length
is one such instance of this phenomena where offenders sentenced to jail
or prison time receive a sentence of \(Y > 0\) days (\(Y^* = Y/30\)
months) while offenders sentenced to a community sentence receive a
jail/prison sentence of \(Y = 0\) days (\(Y^* = 0\) months). Note the
distinction between sentence length which would be inclusive of
community and jail/prison sentence lengths, which are generally not
comparable, and our dependent variable, which is solely jail/prison
sentence length. When we refer to sentence length in this manuscript, we
are referring specifically to jail/prison sentence length.

Hurdle regression combines selection models that determine boundary
points of the dependent variable with an outcome model that determines
its nonbounded values. They treat these boundary values as observed
rather than censored, i.e., observations where the dependent variable is
equal to one of the boundary values are not the result of an inability
to observe the distribution above or below a certain point (Wooldridge,
2010). As defined by Mullahy (1986), ``the idea underlying the hurdle
formation is that a binomial probability model governs the binary
outcome of whether a count variate has a zero or a positive realization.
If the realization is positive, the `hurdle' is crossed, and the
conditional distribution of the positives is governed by a
truncated-at-zero count data model'' (p.345). The hurdle model we
consider treats the log of the number of days that an individual is
sentenced to incarceration as a positive continuous variable and first
model the zeros (i.e., individuals that do not receive incarceration
sentences). A benefit to hurdle models is that they can be used in a
multilevel context and are particularly appropriate when there is an
over-dispersed count distribution (both of which are common in
sentencing research) or when there's a mixture of a discrete variable
and continuous variable (Hester \& Hartman, 2017; Rydberg et al., 2018;
Thompson et al., 2020).

We are defining the race effect measured by the hurdle model as the
change in number of jail/prison days sentenced to between Black and
White offenders. This effect considers the zeros corresponding to
non-incarceration as true zeros, i.e.~the offenders are considered to
have served zero days in jail/prison.

The incomplete data analysis is performed in the multiple imputation
framework (Rubin, 1987). In this set-up the incomplete variables
sentence length, race, age, recommended minimum, and previous record are
imputed or filled in \(M\) times using draws from a predictive
distribution using a regression model. Less than 5\% of observations are
incomplete, with race being incomplete on 3\% of cases and the other
variables incomplete on less than 0.5\% of cases each.

The imputations are then used to create \(M\) completed data sets that
are then analyzed separately using a standard complete data method, in
this case a hurdle model. The estimates from each of the \(M\) model
fits are combined using Rubin's rules. In particular, we are going to
use random forests (Wright \& Ziegler, 2017) to perform the imputations
in the multiple imputation by chained equations framework (S. van Buuren
\& Groothuis-Oudshoorn, 2010; Raghunathan et al., 2001).

Pattern-mixture models can be used in conjunction with multiple
imputation to perform a sensitivity analysis for the model of interest
to particular perturbations of the distribution from which the
imputations are drawn (S. van Buuren, 2018, sec. 3.8). This allows us to
investigate the impacts nonignorable missingness could potentially have
on our analysis.

The pattern-mixture model approach allows the analyst to specify the
exact assumptions of the missingness model and assumptions of how the
distribution varies over different patterns. In our analysis, the two
patterns of interest are the offenders who have a reported race and
those who do not. It is plausible that these two groups have different
characteristics and that the latter group may not have a reported race
because of their true racial/ethnic identity (Stockton et al., 2023).
This is a form of what's known as non-ignorable missingness.

To assess the impacts of the non-ignorable missingness on the race
effects, we can use slight perturbations of the imputed race labels to
get a broader view of the range of potential estimates under different
distributional assumptions. Varying the distribution of imputations
based on the missing data pattern is what allows us to bring
pattern-mixture modeling into the fold.

\hypertarget{analysis}{%
\section{Analysis}\label{analysis}}

Our analysis begins with the multiple imputation step. We create
\(M = 5\) completed data sets using the chained equations framework
under the default MAR assumption. Then the sensitivity analysis perturbs
the race/ethnicity label to generate new imputations, followed by
estimating the race effects using the lognormal hurdle model.

\textbf{Dependent Variable}

The length of the jail/prison sentence modeled as a dependent variable
by our hurdle models. The length is measured in days and is zero for
offenders sentenced to a community sentence.

\textbf{Independent Variables}

The independent variables of interest are the offender's race and sex.
Race is coded as White, Black, Latino or Other. Sex is coded as male or
female.

\textbf{Legally Relevant Variables}

We also include legally relevant variables including the crime type,
whether the minimum sentence was recommended, if there was a trial or
plea, the offender's previous record, and the offense gravity score
(OGS). Additionally, the offender's age is included in the model. OGS
and age were centered and scaled before inclusion in the model.

\textbf{County-level Variable}

County-level random effects are included for both the lognormal model
for non-zero lengths, and the logistic model for the incarceration
decision. For each county, the average of the offense gravity score was
included to provide context for the typical cases in the county as well
as the proportion of cases that each county processes out of the total
number of cases processed in the state of Pennsylvania.

\hypertarget{multiple-imputation}{%
\subsection{Multiple Imputation}\label{multiple-imputation}}

Prior to the multiple imputation, the offense gravity score (OGS),
offender age, and the square of both OGS and age are centered and
scaled. The imputations were generated using random forests in the
multiple imputation by chained equations algorithm. We used \(M = 5\)
since less than 5\% of all observations are missing (mainly on offender
race) (Graham et al., 2007). Random forests are a machine learning
method for modeling categorical and continuous outcomes (Breiman, 2001).

All variables included in the hurdle model are also included in the
imputation model; log of sentence length in days, offender race, sex,
age, previous record, offense gravity score, trial, minimum sentence
recommendation, year, and county.

\hypertarget{hurdle-models}{%
\subsection{Hurdle Models}\label{hurdle-models}}

A hurdle model models data with a high number of zeros (compared to
standard distributions). The model is composed of two components: the
hurdle for the zeros and the GLM for the non-zero part. Let
\(\pi_i = P(Y_i = 0)\) be the probability that the \(i\)th observation
is zero and \(P(Y_i \neq 0) = f_{y\neq 0}(y_i)\) where \(f_{y\neq 0}\)
is a truncated probability density function (Cragg, 1971). We are first
presenting the analysis of the data with multiple imputations under the
MAR assumption to demonstrate how the model fits the data and how to
interpret the race effect estimates.

\hypertarget{lognormal-glm-hurdle-model-with-predictors-on-hurdle-parameter}{%
\subsubsection{Lognormal GLM Hurdle Model with Predictors on Hurdle
Parameter}\label{lognormal-glm-hurdle-model-with-predictors-on-hurdle-parameter}}

Under this first model, we will model the probability of \(Y_i = 0\) as
a logistic regression on the independent and legally relevant variables
with the R package \texttt{brms} (Bürkner, 2018).

\begin{equation}\protect\hypertarget{eq-hurdle-logistic}{}{
\mathrm{logit}^{-1}(P(Y_i = 0)) = \mathbf{x}_i \boldsymbol{\alpha} + \mathbf{z}_i \mathbf{v}.
}\label{eq-hurdle-logistic}\end{equation}

\begin{equation}\protect\hypertarget{eq-hurdle-glm}{}{
\log(Y_i) = \mathbf{x}_i \boldsymbol{\beta} + \mathbf{z}_i \mathbf{u} + \epsilon_i
}\label{eq-hurdle-glm}\end{equation} where \(\mathbf{v}\) and
\(\mathbf{u}\) are independent MVN with
\(E(\mathbf{v}) = E(\mathbf{u}) = 0\) and covariance matrices
\(Cov(\mathbf{v}) = G_v\) and \(Cov(\mathbf{u}) = G_u\),
\(\epsilon_i \overset{iid}{\sim} N(0, \sigma^2)\) and
\(\mathbf{v}, \mathbf{u}\) and \(\boldsymbol{\epsilon}\) are mutually
independent. \(\mathbf{x}_i\) and \(\mathbf{z}_i\) are rows from two
known design matrices for the population-level and group-level effects
respectively.

Here we include group-level effects for the year, the county, and the
crime-type in the county in case the judicial system sentences the
different crime-types differently relative to other counties. Each of
the population-level regression coefficients are given a
weakly-informative normal prior \(\beta_j \sim N(0, 100).\) The
group-level effects for the intercepts and effects for crime-type are
given noncentral t-distributions \(v_k \sim t_{3; 0, 2.5}\) and
\(u_k \sim t_{3; 0, 2.5}\) while the correlations between the
county-level slopes for proportion of cases and county-average OGS and
county-level intercepts are given \(\rho_{i.j} \sim lkj(1)\) priors. The
error term's variance also gets a noncentral t prior
\(\sigma \sim t_{3; 0, 2.5}.\) The coefficients \(\boldsymbol{\alpha}\)
are given normal priors \(\alpha_k \overset{iid}{\sim} N(0, 100).\)

Under this model we are assuming the incomplete data are MAR and the
missingness can be modeled entirely by the multiple imputation
procedure.

\hypertarget{tbl-brms-hurdle-model-summary-2-racesex}{}
\begin{longtable}[]{@{}lrrrr@{}}
\caption{\label{tbl-brms-hurdle-model-summary-2-racesex}Fixed/population-level
effects for the non-zero part of the lognormal hurdle
model.}\tabularnewline
\toprule\noalign{}
Term & Estimate & SE & LB 95\% CI & UB 95\% CI \\
\midrule\noalign{}
\endfirsthead
\toprule\noalign{}
Term & Estimate & SE & LB 95\% CI & UB 95\% CI \\
\midrule\noalign{}
\endhead
\bottomrule\noalign{}
\endlastfoot
BLACK & -0.09 & 0.01 & -0.12 & -0.07 \\
LATINO & 0.03 & 0.05 & -0.07 & 0.13 \\
OTHER & 0.15 & 0.06 & 0.03 & 0.27 \\
Male \& BLACK & 0.07 & 0.01 & 0.05 & 0.09 \\
Male \& LATINO & 0.13 & 0.05 & 0.03 & 0.23 \\
Male \& OTHER & -0.12 & 0.06 & -0.24 & 0.00 \\
\end{longtable}

\hypertarget{tbl-brms-hurdle-model-summary-2-zero-racesex}{}
\begin{longtable}[]{@{}lrrrr@{}}
\caption{\label{tbl-brms-hurdle-model-summary-2-zero-racesex}Fixed/population-level
effects for the zero part of the lognormal hurdle model.}\tabularnewline
\toprule\noalign{}
Term & Estimate & SE & LB 95\% CI & UB 95\% CI \\
\midrule\noalign{}
\endfirsthead
\toprule\noalign{}
Term & Estimate & SE & LB 95\% CI & UB 95\% CI \\
\midrule\noalign{}
\endhead
\bottomrule\noalign{}
\endlastfoot
BLACK & -0.04 & 0.13 & -0.29 & 0.22 \\
LATINO & 0.57 & 0.83 & -0.98 & 2.27 \\
OTHER & 1.15 & 0.92 & -0.42 & 3.18 \\
Male \& BLACK & -0.15 & 0.14 & -0.43 & 0.13 \\
Male \& LATINO & -1.22 & 0.89 & -3.06 & 0.47 \\
Male \& OTHER & -1.28 & 1.00 & -3.44 & 0.48 \\
\end{longtable}

The standard deviation parameter \(\sigma\) of the lognormal
distribution has a posterior mean of 1.04 (95\% CI: {[}1.04, 1.05{]}).

The county-level random/group-level effects and year-level
random/group-level effects are reported in Table~\ref{tbl-brms2-re}.

\hypertarget{tbl-brms2-re}{}
\begin{longtable}[]{@{}lrrrr@{}}
\caption{\label{tbl-brms2-re}Random/group-level effect standard
deviation estimates for the lognormal hurdle model.}\tabularnewline
\toprule\noalign{}
Term & Estimate & SE & LB 95\% CI & UB 95\% CI \\
\midrule\noalign{}
\endfirsthead
\toprule\noalign{}
Term & Estimate & SE & LB 95\% CI & UB 95\% CI \\
\midrule\noalign{}
\endhead
\bottomrule\noalign{}
\endlastfoot
sd(Intercept) & 0.20 & 0.02 & 0.17 & 0.24 \\
sd(Hurdle Intercept) & 0.66 & 0.06 & 0.56 & 0.79 \\
\end{longtable}

In Figure~\ref{fig-cond-eff-brm2-racesex}, we can visualize the impacts
of race and sex on the sentence length in aggregate (including
zero-length sentences) (Figure~\ref{fig-cond-eff-brm2-racesex-1}) and on
the incarceration decision (Figure~\ref{fig-cond-eff-brm2-racesex-2}).
Numeric estimates are reported in Table~\ref{tbl-race-sex-effects}.
Under this model and the MAR missingness assumption, we find that there
is no significant difference in the sentence lengths between the
different racial groups within each sex. There is overlap between all
four credible intervals for both male offenders and female offenders.
Between the sexes, we do see significant differences. Female offenders
receive shorter sentences and are more likely to receive community
sentences regardless of race. There is a large amount of uncertainty in
the effect estimate for Latino offenders. The Other group also has a
relatively high amount of uncertainty and the male Other race offenders'
CI overlaps with all three other groups' CIs for both sexes, although
the mean posterior estimate is lower for sentence length.

\begin{figure}

\begin{minipage}[t]{0.47\linewidth}

{\centering 

\raisebox{-\height}{

\includegraphics{pattern-mixture-modeling_files/figure-pdf/fig-cond-eff-brm2-racesex-1.pdf}

}

}

\subcaption{\label{fig-cond-eff-brm2-racesex-1}Estimates on marginal
sentence length (includes zeros).}
\end{minipage}%
%
\begin{minipage}[t]{0.06\linewidth}

{\centering 

~

}

\end{minipage}%
%
\begin{minipage}[t]{0.47\linewidth}

{\centering 

\raisebox{-\height}{

\includegraphics{pattern-mixture-modeling_files/figure-pdf/fig-cond-eff-brm2-racesex-2.pdf}

}

}

\subcaption{\label{fig-cond-eff-brm2-racesex-2}Estimates on probability
of incarceration from the logistic regression on the zero-part of the
lognormal hurdle model.}
\end{minipage}%

\caption{\label{fig-cond-eff-brm2-racesex}Posterior estimates of the
interaction between race and sex effects on the sentence length in
days.}

\end{figure}

Results for the other terms in the model are reported in the Appendix.

\hypertarget{tbl-race-sex-effects}{}
\begin{longtable}[]{@{}llrrrr@{}}
\caption{\label{tbl-race-sex-effects}Estimates for sentence length
combining the non-zero and zero predictions for the eight combinations
of race and sex.}\tabularnewline
\toprule\noalign{}
Sex & Race & Est & SE & LB 95\% CI & UB 95\% CI \\
\midrule\noalign{}
\endfirsthead
\toprule\noalign{}
Sex & Race & Est & SE & LB 95\% CI & UB 95\% CI \\
\midrule\noalign{}
\endhead
\bottomrule\noalign{}
\endlastfoot
Female & WHITE & 17.31 & 1.15 & 15.25 & 19.74 \\
Female & BLACK & 15.39 & 1.06 & 13.44 & 17.57 \\
Female & LATINO & 22.21 & 2.21 & 18.33 & 26.81 \\
Female & OTHER & 21.33 & 2.30 & 17.18 & 26.35 \\
Male & WHITE & 25.83 & 1.61 & 22.92 & 29.17 \\
Male & BLACK & 30.39 & 1.80 & 27.23 & 34.10 \\
Male & LATINO & 46.70 & 2.62 & 41.84 & 52.23 \\
Male & OTHER & 29.62 & 2.06 & 25.79 & 33.99 \\
\end{longtable}

\hypertarget{sensitivity-analysis}{%
\section{Sensitivity Analysis}\label{sensitivity-analysis}}

Evaluating the impacts of various nonignorable missingness mechanisms
can be accomplished with pattern-mixture models. The values of the
incomplete numeric data can be scaled or shifted. The same cannot be
done with categorical imputations, instead the proportion of each
category can be varied within the imputations compared to the observed
distribution or the MAR imputed distribution.

While there are very few missing sentence lengths, I could also modify
the imputed sentence lengths with a scale \(c\) or shift \(\delta\).
Scaling maintains that offenders that are non-incarcerated will remain
non-incarcerated while the sentences of incarcerated defenders would
shift. A positive shift would make all offenders incarcerated while a
negative shift would impose negative sentence lengths that would need to
be corrected to use Poisson or lognormal hurdle glms.

To perturb the imputations, we modify the vector of probabilities
\(\mathbf{p} = (p_1, p_2, p_3, p_4)'\) of class membership for each
racial/ethnic group under consideration (White, Black, Latino, or
Other). A vector of scale parameters
\(\mathbf{c} = (c_1, c_2, c_3, c_4)'\) is chosen such that the new
racial/ethnic group label will be drawn from the normalized vector
\(\mathbf{p}^* = \frac{1}{\sum_{j=1}^4 c_j p_j} (c_1 p_1, c_2 p_2, c_3 p_3, c_4 p_4)'\)
so that the probability vectors remains the same if \(c_j = 1\) for each
\(j = 1,\dots,4.\) For simplicity, we will focus on varying the
probability of assigning a White label by changing only \(c_1\) while
holding \(c_2 = c_3 = c_4 = 1.\)

The vector of probabilities \(\mathbf{p}_i\) is obtained for each
incomplete observation \(i\) by re-fitting the same random forest used
in the initial mulitple imputation step. This vector is then scaled and
normalized using the same scaling vector \(\mathbf{c}\) for all
incomplete observations. The probability vectors \(\mathbf{p}^*\) are
used to draw new imputed racial/ethnic group labels for the incomplete
observations. This procedure is repeated, including the model fitting,
across each of the \(M\) completed data sets resulting in a new set of
\(M\) completed data sets with modified imputations. The new set of
completed data sets is then analyzed as before using an lognormal hurdle
model. Estimates from each variation of \(\mathbf{c}\) that is chosen
are then compared graphically in Figure~\ref{fig-sens-analysis-res}.

We choose to vary \(c_1\) along the sequence
\(\{0.1, 0.33, 0.75, 0.9, 1, 1.1, 1.25, 1.5, 2\}.\) When \(c_1 = 1\),
the imputations correspond to the original imputation model under the
MAR assumption. For \(c_1 < 1\), fewer observations are imputed with a
White label than under the MAR assumption reflecting a more diverse
population of offenders who have unknown or unreported race labels. The
opposite is true for \(c_1 > 1\), reflecting a more White population of
offenders. We include cases such as \(c_1 = 0.1\) and \(c_1 = 0.33\) as
well as \(c_1 = 1.5\) and \(c_1 = 2\) as rather implausible extreme
conditions to investigate what could happen in the most extreme
circumstances.

\begin{figure}

\begin{minipage}[t]{0.50\linewidth}

{\centering 

\raisebox{-\height}{

\includegraphics{pattern-mixture-modeling_files/figure-pdf/fig-sens-analysis-res-1.pdf}

}

}

\subcaption{\label{fig-sens-analysis-res-1}White offenders}
\end{minipage}%
%
\begin{minipage}[t]{0.50\linewidth}

{\centering 

\raisebox{-\height}{

\includegraphics{pattern-mixture-modeling_files/figure-pdf/fig-sens-analysis-res-2.pdf}

}

}

\subcaption{\label{fig-sens-analysis-res-2}Black offenders}
\end{minipage}%
\newline
\begin{minipage}[t]{0.50\linewidth}

{\centering 

\raisebox{-\height}{

\includegraphics{pattern-mixture-modeling_files/figure-pdf/fig-sens-analysis-res-3.pdf}

}

}

\subcaption{\label{fig-sens-analysis-res-3}Latino offenders}
\end{minipage}%
%
\begin{minipage}[t]{0.50\linewidth}

{\centering 

\raisebox{-\height}{

\includegraphics{pattern-mixture-modeling_files/figure-pdf/fig-sens-analysis-res-4.pdf}

}

}

\subcaption{\label{fig-sens-analysis-res-4}Other ethnicity offenders}
\end{minipage}%

\caption{\label{fig-sens-analysis-res}Coefficient estimates for the
logistic regression model predicting incarceration.}

\end{figure}

Figure~\ref{fig-sens-analysis-res} displays the race effects on sentence
length for the White, Black, Latino, and Other racial/ethnic groups. In
general, we see that estimates for each ethnic group's expected
incarceration decision and sentence length tend to be very stable
regardless of how the distribution of race is perturbed. This indicates
that our estimate under the MAR assumption is likely to be robust
against violations of the MAR assumption on the race or ethnicity of
offenders.

\newpage

\hypertarget{references}{%
\section*{References}\label{references}}
\addcontentsline{toc}{section}{References}

\hypertarget{refs}{}
\begin{CSLReferences}{1}{0}
\leavevmode\vadjust pre{\hypertarget{ref-breiman2001}{}}%
Breiman, L. (2001). Random Forests. \emph{Machine Learning},
\emph{45}(1), 5--32. \url{https://doi.org/10.1023/A:1010933404324}

\leavevmode\vadjust pre{\hypertarget{ref-buxfcrkner2018}{}}%
Bürkner, P.-C. (2018). Advanced Bayesian Multilevel Modeling with the R
Package brms. \emph{The R Journal}, \emph{10}(1), 395.
\url{https://doi.org/10.32614/RJ-2018-017}

\leavevmode\vadjust pre{\hypertarget{ref-vanbuuren2018}{}}%
Buuren, S. van. (2018). \emph{Flexible imputation of missing data} (2nd
ed.). Chapman; Hall/CRC. \url{https://stefvanbuuren.name/fimd/}

\leavevmode\vadjust pre{\hypertarget{ref-buuren2010mice}{}}%
Buuren, S. van, \& Groothuis-Oudshoorn, K. (2010). Mice: Multivariate
imputation by chained equations in r. \emph{Journal of Statistical
Software}, 168.

\leavevmode\vadjust pre{\hypertarget{ref-craggStatisticalModelsLimited1971}{}}%
Cragg, J. G. (1971). Some {Statistical Models} for {Limited Dependent
Variables} with {Application} to the {Demand} for {Durable Goods}.
\emph{Econometrica}, \emph{39}(5), 829--844.
\url{https://doi.org/10.2307/1909582}

\leavevmode\vadjust pre{\hypertarget{ref-grahamHowManyImputations2007}{}}%
Graham, J. W., Olchowski, A. E., \& Gilreath, T. D. (2007). How {Many
Imputations} are {Really Needed}? {Some Practical Clarifications} of
{Multiple Imputation Theory}. \emph{Prevention Science}, \emph{8}(3),
206--213. \url{https://doi.org/10.1007/s11121-007-0070-9}

\leavevmode\vadjust pre{\hypertarget{ref-hesterConditionalRaceDisparities2017}{}}%
Hester, R., \& Hartman, T. K. (2017). Conditional {Race Disparities} in
{Criminal Sentencing}: {A Test} of the {Liberation Hypothesis From} a
{Non-Guidelines State}. \emph{Journal of Quantitative Criminology},
\emph{33}(1), 77--100. \url{https://doi.org/10.1007/s10940-016-9283-z}

\leavevmode\vadjust pre{\hypertarget{ref-mullahySpecificationTestingModified1986}{}}%
Mullahy, J. (1986). Specification and testing of some modified count
data models. \emph{Journal of Econometrics}, \emph{33}(3), 341--365.
\url{https://doi.org/10.1016/0304-4076(86)90002-3}

\leavevmode\vadjust pre{\hypertarget{ref-raghunathanMultivariateTechniqueMultiply2001}{}}%
Raghunathan, T. E., Lepkowski, J. M., Hoewyk, J. V., \& Solenberger, P.
(2001). A {Multivariate Technique} for {Multiply Imputing Missing Values
Using} a {Sequence} of {Regression Models}. \emph{Statistics Canada},
\emph{27}(1), 85--95.

\leavevmode\vadjust pre{\hypertarget{ref-rubinMultipleImputationNonresponse1987}{}}%
Rubin, D. B. (1987). \emph{{Multiple Imputation for Nonresponse in
Surveys {\textbar} Wiley Series in Probability and Statistics}}.
{Wiley}.

\leavevmode\vadjust pre{\hypertarget{ref-rydbergPunishingWickedExamining2018}{}}%
Rydberg, J., Cassidy, M., \& Socia, K. M. (2018). Punishing the
{Wicked}: {Examining} the {Correlates} of {Sentence Severity} for
{Convicted Sex Offenders}. \emph{Journal of Quantitative Criminology},
\emph{34}(4), 943--970. \url{https://doi.org/10.1007/s10940-017-9360-y}

\leavevmode\vadjust pre{\hypertarget{ref-stockton2023}{}}%
Stockton, B., Strange, C. C., \& Harel, O. (2023). Now You See It, Now
You Don{'}t: A Simulation and Illustration of the Importance of Treating
Incomplete Data in Estimating Race Effects in Sentencing. \emph{Journal
of Quantitative Criminology}.
\url{https://doi.org/10.1007/s10940-023-09577-w}

\leavevmode\vadjust pre{\hypertarget{ref-thompsonContextualInfluencesSentencing2020}{}}%
Thompson, L., Rydberg, J., Cassidy, M., \& Socia, K. M. (2020).
Contextual {Influences} on the {Sentencing} of {Individuals Convicted}
of {Sexual Crimes}. \emph{Sexual Abuse}, \emph{32}(7), 778--805.
\url{https://doi.org/10.1177/1079063219852936}

\leavevmode\vadjust pre{\hypertarget{ref-wooldridgeEconometricAnalysisCross2010}{}}%
Wooldridge, J. M. (2010). \emph{Econometric {Analysis} of {Cross
Section} and {Panel Data}, second edition}. {MIT Press}.

\leavevmode\vadjust pre{\hypertarget{ref-wrightRangerFastImplementation2017}{}}%
Wright, M. N., \& Ziegler, A. (2017). Ranger: {A Fast Implementation} of
{Random Forests} for {High Dimensional Data} in {C}++ and {R}.
\emph{Journal of Statistical Software}, \emph{77}, 1--17.
\url{https://doi.org/10.18637/jss.v077.i01}

\end{CSLReferences}

\newpage

\hypertarget{appendix}{%
\section*{Appendix}\label{appendix}}
\addcontentsline{toc}{section}{Appendix}

\begin{figure}

{\centering \includegraphics{pattern-mixture-modeling_files/figure-pdf/fig-incar-cime-1.pdf}

}

\caption{\label{fig-incar-cime}Incarceration decision by most serious
crime type. INCAR == 1 indicates incarceration. INCAR == 0 indicates
parole.}

\end{figure}

\begin{figure}

\begin{minipage}[t]{0.50\linewidth}

{\centering 

\raisebox{-\height}{

\includegraphics{pattern-mixture-modeling_files/figure-pdf/fig-sen-len-qq-1.pdf}

}

}

\subcaption{\label{fig-sen-len-qq-1}Sentence length in days}
\end{minipage}%
%
\begin{minipage}[t]{0.50\linewidth}

{\centering 

\raisebox{-\height}{

\includegraphics{pattern-mixture-modeling_files/figure-pdf/fig-sen-len-qq-2.pdf}

}

}

\subcaption{\label{fig-sen-len-qq-2}Log of sentence length plus one day}
\end{minipage}%

\caption{\label{fig-sen-len-qq}QQ plots of the sentence length (days)}

\end{figure}

\hypertarget{tbl-brms-hurdle-model-summary-2}{}
\begin{longtable}[]{@{}lrrrr@{}}
\caption{\label{tbl-brms-hurdle-model-summary-2}Fixed/population-level
effects for the non-zero part of the lognormal hurdle
model.}\tabularnewline
\toprule\noalign{}
Term & Estimate & SE & LB 95\% CI & UB 95\% CI \\
\midrule\noalign{}
\endfirsthead
\toprule\noalign{}
Term & Estimate & SE & LB 95\% CI & UB 95\% CI \\
\midrule\noalign{}
\endhead
\bottomrule\noalign{}
\endlastfoot
Intercept & 3.87 & 0.03 & 3.82 & 3.92 \\
DOSAGE & 0.04 & 0.00 & 0.04 & 0.05 \\
DOSAGEQ & -0.02 & 0.00 & -0.03 & -0.02 \\
SEXMale & 0.16 & 0.01 & 0.15 & 0.17 \\
OFF\_RACEBLACK & -0.09 & 0.01 & -0.12 & -0.07 \\
OFF\_RACELATINO & 0.03 & 0.05 & -0.07 & 0.13 \\
OFF\_RACEOTHER & 0.15 & 0.06 & 0.03 & 0.27 \\
OGS & 1.28 & 0.00 & 1.27 & 1.29 \\
OGSQ & -0.13 & 0.00 & -0.13 & -0.13 \\
PRVREC1D2D3 & 0.36 & 0.00 & 0.35 & 0.37 \\
PRVREC4D5 & 0.86 & 0.01 & 0.85 & 0.87 \\
PRVRECREVOCDRFEL & 1.30 & 0.01 & 1.28 & 1.32 \\
RECMIN & 0.20 & 0.01 & 0.18 & 0.21 \\
CRIMEDUI & -0.98 & 0.01 & -1.00 & -0.97 \\
CRIMEOther & 0.00 & 0.01 & -0.01 & 0.02 \\
CRIMEPersons & 0.15 & 0.01 & 0.14 & 0.16 \\
CRIMEProperty & 0.14 & 0.01 & 0.13 & 0.15 \\
TRIAL & 0.47 & 0.01 & 0.45 & 0.49 \\
SEXMale:OFF\_RACEBLACK & 0.07 & 0.01 & 0.05 & 0.09 \\
SEXMale:OFF\_RACELATINO & 0.13 & 0.05 & 0.03 & 0.23 \\
SEXMale:OFF\_RACEOTHER & -0.12 & 0.06 & -0.24 & 0.00 \\
OGS:PRVREC1D2D3 & -0.02 & 0.00 & -0.02 & -0.01 \\
OGS:PRVREC4D5 & -0.06 & 0.00 & -0.07 & -0.05 \\
OGS:PRVRECREVOCDRFEL & -0.14 & 0.01 & -0.15 & -0.12 \\
\end{longtable}

\hypertarget{tbl-brms-hurdle-model-summary-2-zero}{}
\begin{longtable}[]{@{}lrrrr@{}}
\caption{\label{tbl-brms-hurdle-model-summary-2-zero}Fixed/population-level
effects for the zero part of the lognormal hurdle model.}\tabularnewline
\toprule\noalign{}
Term & Estimate & SE & LB 95\% CI & UB 95\% CI \\
\midrule\noalign{}
\endfirsthead
\toprule\noalign{}
Term & Estimate & SE & LB 95\% CI & UB 95\% CI \\
\midrule\noalign{}
\endhead
\bottomrule\noalign{}
\endlastfoot
hu\_Intercept & 1.73 & 0.08 & 1.57 & 1.88 \\
hu\_DOSAGE & 0.13 & 0.00 & 0.13 & 0.14 \\
hu\_DOSAGEQ & 0.02 & 0.00 & 0.02 & 0.03 \\
hu\_SEXMale & -0.32 & 0.01 & -0.33 & -0.31 \\
hu\_OFF\_RACEBLACK & 0.03 & 0.01 & 0.01 & 0.06 \\
hu\_OFF\_RACELATINO & -0.29 & 0.08 & -0.44 & -0.13 \\
hu\_OFF\_RACEOTHER & -0.07 & 0.08 & -0.23 & 0.09 \\
hu\_RECMIN & -0.89 & 0.01 & -0.91 & -0.87 \\
hu\_OGS & -0.41 & 0.00 & -0.42 & -0.40 \\
hu\_OGSQ & -0.22 & 0.00 & -0.23 & -0.21 \\
hu\_PRVREC1D2D3 & -0.51 & 0.01 & -0.52 & -0.50 \\
hu\_PRVREC4D5 & -1.11 & 0.01 & -1.13 & -1.09 \\
hu\_PRVRECREVOCDRFEL & -1.47 & 0.02 & -1.51 & -1.43 \\
hu\_CRIMEDUI & -1.68 & 0.01 & -1.69 & -1.66 \\
hu\_CRIMEOther & -0.30 & 0.01 & -0.32 & -0.28 \\
hu\_CRIMEPersons & -0.81 & 0.01 & -0.83 & -0.79 \\
hu\_CRIMEProperty & -0.48 & 0.01 & -0.49 & -0.46 \\
hu\_TRIAL & -0.80 & 0.02 & -0.85 & -0.76 \\
hu\_SEXMale:OFF\_RACEBLACK & -0.31 & 0.02 & -0.34 & -0.28 \\
hu\_SEXMale:OFF\_RACELATINO & -0.39 & 0.08 & -0.56 & -0.22 \\
hu\_SEXMale:OFF\_RACEOTHER & -0.08 & 0.09 & -0.26 & 0.09 \\
\end{longtable}

\begin{figure}

\begin{minipage}[t]{0.33\linewidth}

{\centering 

\raisebox{-\height}{

\includegraphics{pattern-mixture-modeling_files/figure-pdf/fig-cond-eff-brm2-nonzero-1.pdf}

}

}

\subcaption{\label{fig-cond-eff-brm2-nonzero-1}Age}
\end{minipage}%
%
\begin{minipage}[t]{0.33\linewidth}

{\centering 

\raisebox{-\height}{

\includegraphics{pattern-mixture-modeling_files/figure-pdf/fig-cond-eff-brm2-nonzero-2.pdf}

}

}

\subcaption{\label{fig-cond-eff-brm2-nonzero-2}Age Squared}
\end{minipage}%
%
\begin{minipage}[t]{0.33\linewidth}

{\centering 

\raisebox{-\height}{

\includegraphics{pattern-mixture-modeling_files/figure-pdf/fig-cond-eff-brm2-nonzero-3.pdf}

}

}

\subcaption{\label{fig-cond-eff-brm2-nonzero-3}Sex}
\end{minipage}%
\newline
\begin{minipage}[t]{0.33\linewidth}

{\centering 

\raisebox{-\height}{

\includegraphics{pattern-mixture-modeling_files/figure-pdf/fig-cond-eff-brm2-nonzero-4.pdf}

}

}

\subcaption{\label{fig-cond-eff-brm2-nonzero-4}Offender Race}
\end{minipage}%
%
\begin{minipage}[t]{0.33\linewidth}

{\centering 

\raisebox{-\height}{

\includegraphics{pattern-mixture-modeling_files/figure-pdf/fig-cond-eff-brm2-nonzero-5.pdf}

}

}

\subcaption{\label{fig-cond-eff-brm2-nonzero-5}OGS}
\end{minipage}%
%
\begin{minipage}[t]{0.33\linewidth}

{\centering 

\raisebox{-\height}{

\includegraphics{pattern-mixture-modeling_files/figure-pdf/fig-cond-eff-brm2-nonzero-6.pdf}

}

}

\subcaption{\label{fig-cond-eff-brm2-nonzero-6}OGS Squared}
\end{minipage}%
\newline
\begin{minipage}[t]{0.33\linewidth}

{\centering 

\raisebox{-\height}{

\includegraphics{pattern-mixture-modeling_files/figure-pdf/fig-cond-eff-brm2-nonzero-7.pdf}

}

}

\subcaption{\label{fig-cond-eff-brm2-nonzero-7}Prev. Rec.}
\end{minipage}%
%
\begin{minipage}[t]{0.33\linewidth}

{\centering 

\raisebox{-\height}{

\includegraphics{pattern-mixture-modeling_files/figure-pdf/fig-cond-eff-brm2-nonzero-8.pdf}

}

}

\subcaption{\label{fig-cond-eff-brm2-nonzero-8}Rec. Min.}
\end{minipage}%
%
\begin{minipage}[t]{0.33\linewidth}

{\centering 

\raisebox{-\height}{

\includegraphics{pattern-mixture-modeling_files/figure-pdf/fig-cond-eff-brm2-nonzero-9.pdf}

}

}

\subcaption{\label{fig-cond-eff-brm2-nonzero-9}Crime Type}
\end{minipage}%
\newline
\begin{minipage}[t]{0.33\linewidth}

{\centering 

\raisebox{-\height}{

\includegraphics{pattern-mixture-modeling_files/figure-pdf/fig-cond-eff-brm2-nonzero-10.pdf}

}

}

\subcaption{\label{fig-cond-eff-brm2-nonzero-10}Trial}
\end{minipage}%
%
\begin{minipage}[t]{0.33\linewidth}

{\centering 

\raisebox{-\height}{

\includegraphics{pattern-mixture-modeling_files/figure-pdf/fig-cond-eff-brm2-nonzero-11.pdf}

}

}

\subcaption{\label{fig-cond-eff-brm2-nonzero-11}Race by Sex}
\end{minipage}%
%
\begin{minipage}[t]{0.33\linewidth}

{\centering 

\raisebox{-\height}{

\includegraphics{pattern-mixture-modeling_files/figure-pdf/fig-cond-eff-brm2-nonzero-12.pdf}

}

}

\subcaption{\label{fig-cond-eff-brm2-nonzero-12}OGS by Prev. Rec.}
\end{minipage}%

\caption{\label{fig-cond-eff-brm2-nonzero}Posterior estimates of the
conditional effects for the non-zero-part of the lognormal hurdle
model.}

\end{figure}

\begin{figure}

\begin{minipage}[t]{0.33\linewidth}

{\centering 

\raisebox{-\height}{

\includegraphics{pattern-mixture-modeling_files/figure-pdf/fig-cond-eff-brm2-1.pdf}

}

}

\subcaption{\label{fig-cond-eff-brm2-1}Age}
\end{minipage}%
%
\begin{minipage}[t]{0.33\linewidth}

{\centering 

\raisebox{-\height}{

\includegraphics{pattern-mixture-modeling_files/figure-pdf/fig-cond-eff-brm2-2.pdf}

}

}

\subcaption{\label{fig-cond-eff-brm2-2}Age Squared}
\end{minipage}%
%
\begin{minipage}[t]{0.33\linewidth}

{\centering 

\raisebox{-\height}{

\includegraphics{pattern-mixture-modeling_files/figure-pdf/fig-cond-eff-brm2-3.pdf}

}

}

\subcaption{\label{fig-cond-eff-brm2-3}Sex}
\end{minipage}%
\newline
\begin{minipage}[t]{0.33\linewidth}

{\centering 

\raisebox{-\height}{

\includegraphics{pattern-mixture-modeling_files/figure-pdf/fig-cond-eff-brm2-4.pdf}

}

}

\subcaption{\label{fig-cond-eff-brm2-4}Offender Race}
\end{minipage}%
%
\begin{minipage}[t]{0.33\linewidth}

{\centering 

\raisebox{-\height}{

\includegraphics{pattern-mixture-modeling_files/figure-pdf/fig-cond-eff-brm2-5.pdf}

}

}

\subcaption{\label{fig-cond-eff-brm2-5}OGS}
\end{minipage}%
%
\begin{minipage}[t]{0.33\linewidth}

{\centering 

\raisebox{-\height}{

\includegraphics{pattern-mixture-modeling_files/figure-pdf/fig-cond-eff-brm2-6.pdf}

}

}

\subcaption{\label{fig-cond-eff-brm2-6}OGS Squared}
\end{minipage}%
\newline
\begin{minipage}[t]{0.33\linewidth}

{\centering 

\raisebox{-\height}{

\includegraphics{pattern-mixture-modeling_files/figure-pdf/fig-cond-eff-brm2-7.pdf}

}

}

\subcaption{\label{fig-cond-eff-brm2-7}Prev. Rec.}
\end{minipage}%
%
\begin{minipage}[t]{0.33\linewidth}

{\centering 

\raisebox{-\height}{

\includegraphics{pattern-mixture-modeling_files/figure-pdf/fig-cond-eff-brm2-8.pdf}

}

}

\subcaption{\label{fig-cond-eff-brm2-8}Rec. Min.}
\end{minipage}%
%
\begin{minipage}[t]{0.33\linewidth}

{\centering 

\raisebox{-\height}{

\includegraphics{pattern-mixture-modeling_files/figure-pdf/fig-cond-eff-brm2-9.pdf}

}

}

\subcaption{\label{fig-cond-eff-brm2-9}Crime Type}
\end{minipage}%
\newline
\begin{minipage}[t]{0.33\linewidth}

{\centering 

\raisebox{-\height}{

\includegraphics{pattern-mixture-modeling_files/figure-pdf/fig-cond-eff-brm2-10.pdf}

}

}

\subcaption{\label{fig-cond-eff-brm2-10}Trial}
\end{minipage}%
%
\begin{minipage}[t]{0.33\linewidth}

{\centering 

\raisebox{-\height}{

\includegraphics{pattern-mixture-modeling_files/figure-pdf/fig-cond-eff-brm2-11.pdf}

}

}

\subcaption{\label{fig-cond-eff-brm2-11}Race by Sex}
\end{minipage}%
%
\begin{minipage}[t]{0.33\linewidth}

{\centering 

\raisebox{-\height}{

\includegraphics{pattern-mixture-modeling_files/figure-pdf/fig-cond-eff-brm2-12.pdf}

}

}

\subcaption{\label{fig-cond-eff-brm2-12}OGS by Prev. Rec.}
\end{minipage}%

\caption{\label{fig-cond-eff-brm2}Posterior estimates of the conditional
effects for the logistic regression on the zero-part of the lognormal
hurdle model.}

\end{figure}



\end{document}
