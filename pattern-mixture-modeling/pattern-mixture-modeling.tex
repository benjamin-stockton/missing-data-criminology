% Options for packages loaded elsewhere
\PassOptionsToPackage{unicode}{hyperref}
\PassOptionsToPackage{hyphens}{url}
\PassOptionsToPackage{dvipsnames,svgnames,x11names}{xcolor}
%
\documentclass[
  letterpaper,
  DIV=11,
  numbers=noendperiod]{scrartcl}

\usepackage{amsmath,amssymb}
\usepackage{iftex}
\ifPDFTeX
  \usepackage[T1]{fontenc}
  \usepackage[utf8]{inputenc}
  \usepackage{textcomp} % provide euro and other symbols
\else % if luatex or xetex
  \usepackage{unicode-math}
  \defaultfontfeatures{Scale=MatchLowercase}
  \defaultfontfeatures[\rmfamily]{Ligatures=TeX,Scale=1}
\fi
\usepackage{lmodern}
\ifPDFTeX\else  
    % xetex/luatex font selection
\fi
% Use upquote if available, for straight quotes in verbatim environments
\IfFileExists{upquote.sty}{\usepackage{upquote}}{}
\IfFileExists{microtype.sty}{% use microtype if available
  \usepackage[]{microtype}
  \UseMicrotypeSet[protrusion]{basicmath} % disable protrusion for tt fonts
}{}
\makeatletter
\@ifundefined{KOMAClassName}{% if non-KOMA class
  \IfFileExists{parskip.sty}{%
    \usepackage{parskip}
  }{% else
    \setlength{\parindent}{0pt}
    \setlength{\parskip}{6pt plus 2pt minus 1pt}}
}{% if KOMA class
  \KOMAoptions{parskip=half}}
\makeatother
\usepackage{xcolor}
\setlength{\emergencystretch}{3em} % prevent overfull lines
\setcounter{secnumdepth}{2}
% Make \paragraph and \subparagraph free-standing
\ifx\paragraph\undefined\else
  \let\oldparagraph\paragraph
  \renewcommand{\paragraph}[1]{\oldparagraph{#1}\mbox{}}
\fi
\ifx\subparagraph\undefined\else
  \let\oldsubparagraph\subparagraph
  \renewcommand{\subparagraph}[1]{\oldsubparagraph{#1}\mbox{}}
\fi


\providecommand{\tightlist}{%
  \setlength{\itemsep}{0pt}\setlength{\parskip}{0pt}}\usepackage{longtable,booktabs,array}
\usepackage{calc} % for calculating minipage widths
% Correct order of tables after \paragraph or \subparagraph
\usepackage{etoolbox}
\makeatletter
\patchcmd\longtable{\par}{\if@noskipsec\mbox{}\fi\par}{}{}
\makeatother
% Allow footnotes in longtable head/foot
\IfFileExists{footnotehyper.sty}{\usepackage{footnotehyper}}{\usepackage{footnote}}
\makesavenoteenv{longtable}
\usepackage{graphicx}
\makeatletter
\def\maxwidth{\ifdim\Gin@nat@width>\linewidth\linewidth\else\Gin@nat@width\fi}
\def\maxheight{\ifdim\Gin@nat@height>\textheight\textheight\else\Gin@nat@height\fi}
\makeatother
% Scale images if necessary, so that they will not overflow the page
% margins by default, and it is still possible to overwrite the defaults
% using explicit options in \includegraphics[width, height, ...]{}
\setkeys{Gin}{width=\maxwidth,height=\maxheight,keepaspectratio}
% Set default figure placement to htbp
\makeatletter
\def\fps@figure{htbp}
\makeatother
\newlength{\cslhangindent}
\setlength{\cslhangindent}{1.5em}
\newlength{\csllabelwidth}
\setlength{\csllabelwidth}{3em}
\newlength{\cslentryspacingunit} % times entry-spacing
\setlength{\cslentryspacingunit}{\parskip}
\newenvironment{CSLReferences}[2] % #1 hanging-ident, #2 entry spacing
 {% don't indent paragraphs
  \setlength{\parindent}{0pt}
  % turn on hanging indent if param 1 is 1
  \ifodd #1
  \let\oldpar\par
  \def\par{\hangindent=\cslhangindent\oldpar}
  \fi
  % set entry spacing
  \setlength{\parskip}{#2\cslentryspacingunit}
 }%
 {}
\usepackage{calc}
\newcommand{\CSLBlock}[1]{#1\hfill\break}
\newcommand{\CSLLeftMargin}[1]{\parbox[t]{\csllabelwidth}{#1}}
\newcommand{\CSLRightInline}[1]{\parbox[t]{\linewidth - \csllabelwidth}{#1}\break}
\newcommand{\CSLIndent}[1]{\hspace{\cslhangindent}#1}

\usepackage{booktabs}
\usepackage{longtable}
\usepackage{array}
\usepackage{multirow}
\usepackage{wrapfig}
\usepackage{float}
\usepackage{colortbl}
\usepackage{pdflscape}
\usepackage{tabu}
\usepackage{threeparttable}
\usepackage{threeparttablex}
\usepackage[normalem]{ulem}
\usepackage{makecell}
\usepackage{xcolor}
\KOMAoption{captions}{tableheading}
\makeatletter
\makeatother
\makeatletter
\makeatother
\makeatletter
\@ifpackageloaded{caption}{}{\usepackage{caption}}
\AtBeginDocument{%
\ifdefined\contentsname
  \renewcommand*\contentsname{Table of contents}
\else
  \newcommand\contentsname{Table of contents}
\fi
\ifdefined\listfigurename
  \renewcommand*\listfigurename{List of Figures}
\else
  \newcommand\listfigurename{List of Figures}
\fi
\ifdefined\listtablename
  \renewcommand*\listtablename{List of Tables}
\else
  \newcommand\listtablename{List of Tables}
\fi
\ifdefined\figurename
  \renewcommand*\figurename{Figure}
\else
  \newcommand\figurename{Figure}
\fi
\ifdefined\tablename
  \renewcommand*\tablename{Table}
\else
  \newcommand\tablename{Table}
\fi
}
\@ifpackageloaded{float}{}{\usepackage{float}}
\floatstyle{ruled}
\@ifundefined{c@chapter}{\newfloat{codelisting}{h}{lop}}{\newfloat{codelisting}{h}{lop}[chapter]}
\floatname{codelisting}{Listing}
\newcommand*\listoflistings{\listof{codelisting}{List of Listings}}
\makeatother
\makeatletter
\@ifpackageloaded{caption}{}{\usepackage{caption}}
\@ifpackageloaded{subcaption}{}{\usepackage{subcaption}}
\makeatother
\makeatletter
\@ifpackageloaded{tcolorbox}{}{\usepackage[skins,breakable]{tcolorbox}}
\makeatother
\makeatletter
\@ifundefined{shadecolor}{\definecolor{shadecolor}{rgb}{.97, .97, .97}}
\makeatother
\makeatletter
\makeatother
\makeatletter
\makeatother
\ifLuaTeX
  \usepackage{selnolig}  % disable illegal ligatures
\fi
\IfFileExists{bookmark.sty}{\usepackage{bookmark}}{\usepackage{hyperref}}
\IfFileExists{xurl.sty}{\usepackage{xurl}}{} % add URL line breaks if available
\urlstyle{same} % disable monospaced font for URLs
\hypersetup{
  pdftitle={Sentencing Analysis with Pattern Mixture Modeling},
  pdfauthor={C. Clare Strange; Benjamin Stockton; Ofer Harel},
  pdfkeywords={incomplete data, pattern-mixture model},
  colorlinks=true,
  linkcolor={blue},
  filecolor={Maroon},
  citecolor={Blue},
  urlcolor={Blue},
  pdfcreator={LaTeX via pandoc}}

\title{Sentencing Analysis with Pattern Mixture Modeling}
\author{C. Clare Strange \and Benjamin Stockton \and Ofer Harel}
\date{2024-01-08}

\begin{document}
\maketitle
\ifdefined\Shaded\renewenvironment{Shaded}{\begin{tcolorbox}[interior hidden, sharp corners, borderline west={3pt}{0pt}{shadecolor}, frame hidden, boxrule=0pt, enhanced, breakable]}{\end{tcolorbox}}\fi

\hypertarget{methods}{%
\section{Methods}\label{methods}}

We propose using a combination of multiple imputation and
pattern-mixture models to perform sensitivity analyses for the race
effect estimates from a lognormal hurdle model used to predict the
sentence length as a response variable under incompleteness of the race
variable. The lognormal hurdle model can be used to model complex data
where the dependent variable is a combination of true zeros and a
continuous distribution for non-zero observations. Sentence length is
one such instance wherein this phenomena arises with offenders sentenced
to a community sentence or parole receive a sentence of 0 days
(\(Y^* = 0\) months) while offenders sentenced to jail or prison time
receive a sentence of \(Y > 0\) days (\(Y^* = Y/30\) months).

The class of hurdle models fits a logistic regression model to the zero
part of the data and then fits a count or continuous generalized linear
model to the non-zero part of the data. The coefficients from each part
of the model can be interpreted independently or marginally. Predictions
are made from the mixture of the zero and non-zero components.

The incomplete data analysis is performed in the multiple imputation
framework (Rubin 1987). In this set-up the incomplete variables are
imputed or filled in \(M\) times using draws from a predictive
distribution using a regression model. The imputations are then used to
create \(M\) completed data sets that are then analyzed separately using
a standard complete data method, such as a hurdle regression model. The
estimates from each of the \(M\) model fits are combined using Rubin's
rules. In particular, we are going to use random forests
(\textbf{wrightrangerAFastImplementation2017?}) to perform the
imputations in the multiple imputation by chained equations framework
(\textbf{buuren2010mice?}).

Pattern-mixture models can be used in conjunction with multiple
imputation to perform a sensitivity analysis for the model of interest
to particular perturbations of the distribution from which the
imputations are drawn. This allows us to investigate the impacts
nonignorable missingness could potentially have on our analysis.

\hypertarget{analysis}{%
\section{Analysis}\label{analysis}}

\begin{figure}

{\centering \includegraphics{pattern-mixture-modeling_files/figure-pdf/fig-miss-pattern-1.pdf}

}

\caption{\label{fig-miss-pattern}Missing data patterns for the full data
set.}

\end{figure}

\hypertarget{tbl-yearly-summary}{}
\begin{table}
\caption{\label{tbl-yearly-summary}Summary statistics on sentence length (days) and incarceration. }\tabularnewline

\centering
\begin{tabular}{lrrrrrr}
\toprule
Year & Mean & SD & Median & Min. & Max. & P(Incar)\\
\midrule
2010 & 171.74 & 992.32 & 0 & 0 & 230468 & 0.48\\
2011 & 175.87 & 630.95 & 0 & 0 & 30681 & 0.48\\
2012 & 169.65 & 664.13 & 0 & 0 & 38167 & 0.47\\
2013 & 178.44 & 651.66 & 0 & 0 & 30865 & 0.48\\
2014 & 168.79 & 679.93 & 0 & 0 & 73048 & 0.47\\
\addlinespace
2015 & 159.40 & 601.31 & 0 & 0 & 29951 & 0.46\\
2016 & 149.20 & 635.62 & 0 & 0 & 32141 & 0.42\\
2017 & 157.63 & 773.66 & 0 & 0 & 87658 & 0.44\\
2018 & 150.23 & 585.03 & 0 & 0 & 25568 & 0.44\\
2019 & 146.35 & 600.50 & 0 & 0 & 28579 & 0.43\\
\bottomrule
\end{tabular}
\end{table}

\begin{figure}

{\centering \includegraphics{pattern-mixture-modeling_files/figure-pdf/fig-sen-len-yearly-1.pdf}

}

\caption{\label{fig-sen-len-yearly}Density plots for A) Sentence length
(day) by year. B) Log sentence lengths (day) plus one day by year.}

\end{figure}

\begin{figure}

{\centering \includegraphics{pattern-mixture-modeling_files/figure-pdf/fig-sen-len-crime-1.pdf}

}

\caption{\label{fig-sen-len-crime}A) Sentence length (day) by most
serious crime type. B) Log of sentence length plus one day by most
serious crime type.}

\end{figure}

\begin{figure}

{\centering \includegraphics{pattern-mixture-modeling_files/figure-pdf/fig-sen-len-race-1.pdf}

}

\caption{\label{fig-sen-len-race}A) Sentence length by offender race. B)
Log of sentence length (plus one day)}

\end{figure}

\begin{figure}

{\centering \includegraphics{pattern-mixture-modeling_files/figure-pdf/fig-incar-cime-1.pdf}

}

\caption{\label{fig-incar-cime}Incarceration decision by most serious
crime type. INCAR == 1 indicates incarceration. INCAR == 0 indicates
parole.}

\end{figure}

\begin{figure}

\begin{minipage}[t]{0.50\linewidth}

{\centering 

\raisebox{-\height}{

\includegraphics{pattern-mixture-modeling_files/figure-pdf/fig-sen-len-qq-1.pdf}

}

}

\subcaption{\label{fig-sen-len-qq-1}Sentence length in days}
\end{minipage}%
%
\begin{minipage}[t]{0.50\linewidth}

{\centering 

\raisebox{-\height}{

\includegraphics{pattern-mixture-modeling_files/figure-pdf/fig-sen-len-qq-2.pdf}

}

}

\subcaption{\label{fig-sen-len-qq-2}Log of sentence length plus one day}
\end{minipage}%

\caption{\label{fig-sen-len-qq}QQ plots of the sentence length (days)}

\end{figure}

\begin{figure}

{\centering \includegraphics{pattern-mixture-modeling_files/figure-pdf/fig-scatter-pairs-1.pdf}

}

\caption{\label{fig-scatter-pairs}Scatter plot matrix of the numeric
predictors and the log plus one of the sentence length (days).}

\end{figure}

\hypertarget{multiple-imputation}{%
\subsection{Multiple Imputation}\label{multiple-imputation}}

I'll prepare the data for analysis next. First, I'll center and scale
the numeric predictors which are Offense Gravity Score (OGS), defendant
age, and the square of each. Then I'll use MICE to multiply impute the
incomplete variables with random forest. For now I'll use \(M = 10\)
since less than 5\% of all observations are missing (mainly in defendant
race). The number of imputations to use should be informed by the amount
of missing information due to incompleteness for each variable (Harel
2007).

\hypertarget{logistic-regression-on-incarceration}{%
\subsection{Logistic Regression on
Incarceration}\label{logistic-regression-on-incarceration}}

I'll fit a logistic regression like before as a sanity check. The
estimated odds ratio for increased odds of incarceration for a Black
defendant over a White defendant should be roughly 1.25 as we saw in the
complete case analysis in the previous paper.

\hypertarget{tbl-glm-summary}{}
\begin{table}
\caption{\label{tbl-glm-summary}Summary of the logistic regression fit for predicting sentencing
decision (In/Out) excluding the year and county estimates for brevity. }\tabularnewline

\centering
\begin{tabular}{lrrrrl}
\toprule
Term & Estimate & SE & LB 95\% CI & UB 95\% CI & Sig.\\
\midrule
(Intercept) & -3.32 & 0.29 & -3.89 & -2.75 & *\\
DOSAGE & -0.16 & 0.03 & -0.22 & -0.09 & *\\
DOSAGEQ & -0.02 & 0.02 & -0.06 & 0.03 & \\
SEXMale & 0.37 & 0.07 & 0.23 & 0.50 & *\\
OFF\_RACEBLACK & 0.14 & 0.14 & -0.15 & 0.42 & \\
\addlinespace
OFF\_RACELATINO & 0.75 & 0.65 & -0.53 & 2.02 & \\
OFF\_RACEOTHER & 0.59 & 0.66 & -0.70 & 1.88 & \\
OGS & 0.26 & 0.05 & 0.16 & 0.36 & *\\
OGSQ & 0.25 & 0.03 & 0.19 & 0.31 & *\\
PRVREC1/2/3 & 0.49 & 0.06 & 0.37 & 0.61 & *\\
\addlinespace
PRVREC4/5 & 1.27 & 0.10 & 1.08 & 1.46 & *\\
PRVRECREVOC/RFEL & 1.66 & 0.19 & 1.29 & 2.03 & *\\
RECMIN & 0.94 & 0.10 & 0.75 & 1.13 & *\\
CRIMEDUI & 1.79 & 0.08 & 1.64 & 1.95 & *\\
CRIMEOther & 0.24 & 0.09 & 0.06 & 0.42 & *\\
\addlinespace
CRIMEPersons & 0.88 & 0.08 & 0.72 & 1.05 & *\\
CRIMEProperty & 0.46 & 0.07 & 0.31 & 0.60 & *\\
TRIAL & 0.64 & 0.20 & 0.25 & 1.03 & *\\
SEXMale:OFF\_RACEBLACK & 0.16 & 0.16 & -0.15 & 0.48 & \\
SEXMale:OFF\_RACELATINO & -0.11 & 0.69 & -1.47 & 1.25 & \\
\addlinespace
SEXMale:OFF\_RACEOTHER & -0.50 & 0.73 & -1.94 & 0.94 & \\
OGS:PRVREC1/2/3 & 0.07 & 0.07 & -0.06 & 0.20 & \\
OGS:PRVREC4/5 & 0.33 & 0.09 & 0.16 & 0.51 & *\\
OGS:PRVRECREVOC/RFEL & 0.39 & 0.23 & -0.06 & 0.85 & \\
\bottomrule
\end{tabular}
\end{table}

From the logistic regression fit with MI and \(M = 10\) imputations done
using predictive mean matching, we found that a Black defendant is 1.25
(95\% CI of (1.234, 1.265) times more likely to be sentenced to
incarceration than an otherwise similar White defendant.

We re-analyze the data using a generalized linear mixed model and again
taking the binary incarceration decision as the outcome and random
effects for the Year and County with random slopes for the most serious
Crime type by County.

\hypertarget{tbl-glmm-sum}{}
\begin{table}
\caption{\label{tbl-glmm-sum}Summary statistics for the fixed effects from the logistic regression
mixed model. }\tabularnewline

\centering
\begin{tabular}{lrrrrl}
\toprule
Term & Estimate & SE & LB 95\% CI & UB 95\% CI & Sig.\\
\midrule
(Intercept) & -1.82 & 0.14 & -2.09 & -1.55 & *\\
DOSAGE & -0.16 & 0.03 & -0.22 & -0.09 & *\\
DOSAGEQ & -0.02 & 0.02 & -0.06 & 0.03 & \\
SEXMale & 0.37 & 0.07 & 0.23 & 0.50 & *\\
OFF\_RACEBLACK & 0.12 & 0.14 & -0.16 & 0.41 & \\
\addlinespace
OFF\_RACELATINO & 0.75 & 0.65 & -0.53 & 2.02 & \\
OFF\_RACEOTHER & 0.58 & 0.66 & -0.70 & 1.87 & \\
OGS & 0.25 & 0.05 & 0.15 & 0.35 & *\\
OGSQ & 0.25 & 0.03 & 0.19 & 0.31 & *\\
PRVREC1/2/3 & 0.49 & 0.06 & 0.37 & 0.61 & *\\
\addlinespace
PRVREC4/5 & 1.26 & 0.10 & 1.07 & 1.45 & *\\
PRVRECREVOC/RFEL & 1.64 & 0.19 & 1.27 & 2.01 & *\\
RECMIN & 0.94 & 0.10 & 0.75 & 1.13 & *\\
CRIMEDUI & 1.79 & 0.08 & 1.63 & 1.94 & *\\
CRIMEOther & 0.25 & 0.09 & 0.07 & 0.42 & *\\
\addlinespace
CRIMEPersons & 0.88 & 0.08 & 0.72 & 1.05 & *\\
CRIMEProperty & 0.46 & 0.07 & 0.32 & 0.60 & *\\
TRIAL & 0.62 & 0.20 & 0.24 & 1.01 & *\\
SEXMale:OFF\_RACEBLACK & 0.16 & 0.16 & -0.15 & 0.48 & \\
SEXMale:OFF\_RACELATINO & -0.10 & 0.69 & -1.46 & 1.27 & \\
\addlinespace
SEXMale:OFF\_RACEOTHER & -0.51 & 0.73 & -1.94 & 0.93 & \\
OGS:PRVREC1/2/3 & 0.07 & 0.07 & -0.07 & 0.20 & \\
OGS:PRVREC4/5 & 0.34 & 0.09 & 0.16 & 0.51 & *\\
OGS:PRVRECREVOC/RFEL & 0.39 & 0.23 & -0.06 & 0.84 & \\
\bottomrule
\end{tabular}
\end{table}

\hypertarget{hurdle-models}{%
\subsection{Hurdle Models}\label{hurdle-models}}

A hurdle model models data with a high number of zeros (compared to
standard distributions). The model places a probability point mass
\(P(Y = 0) = \theta\) at \(Y = 0\) and uses a truncated (at zero)
probability distribution for the non-zero sample space
\(P(Y \neq 0) = p_{y \neq 0}(y)\). This differs from a zero-inflated
model which is a mixture of two distributions (includes the non-zero
distribution's zero probability) as the hurdle model truncates the
non-zero distribution.

I'll create a GLM with brms and the log normal hurdle distribution.

The model is composed of two components: the hurdle for the zeros and
the GLM for the non-zero part. Let \(\pi_i\) be the probability that the
\(i\)th observation is zero and \(P(Y_i \neq 0) = f_{y\neq 0}(y-i)\)
where \(f_{y\neq 0}\) is a truncated probability mass/density function.

\hypertarget{lognormal-hurdle-glm-with-intercept-only-hurdle}{%
\subsubsection{Lognormal Hurdle GLM with Intercept-only
Hurdle}\label{lognormal-hurdle-glm-with-intercept-only-hurdle}}

Under this first model, we will model the probability of \(Y_i = 0\) as
constant across the observations using an intercept-only model; the
default for \texttt{brms} (\textbf{bürkner2017?}).

\begin{equation}\protect\hypertarget{eq-hurdle-intercept}{}{
\mathrm{logit}^{-1}(P(Y_i = 0)) = \pi_0
}\label{eq-hurdle-intercept}\end{equation}

\begin{equation}\protect\hypertarget{eq-hurdle-glm}{}{
\log(Y_i) = \mathbf{x}_i \boldsymbol{\beta} + \mathbf{z}_i \mathbf{u} + \epsilon_i
}\label{eq-hurdle-glm}\end{equation} where \(\mathbf{u}\) is MVN with
\(E(\mathbf{U}) = 0\) and covariance matrix \(Cov(\mathbf{U}) = G\),
\(\epsilon_i \overset{iid}{\sim} N(0, \sigma^2)\) and \(\mathbf{u}\) and
\(\boldsymbol{\epsilon}\) are mutually independent. \(\mathbf{x}_i\) and
\(\mathbf{z}_i\) are rows from two known design matrices for the
population-level and group-level effects respectively.

The model is specified as usual for a GLMM:

\begin{verbatim}
brm(JP_MIN ~ DOSAGE + DOSAGEQ + SEX*OFF_RACE + OGS + OGSQ +
                     PRVREC + RECMIN + CRIME + TRIAL +
                     (1 | YEAR) + (1 + CRIME |COUNTY),
               data = data,
               family = hurdle_lognormal(link = "identity",
                           link_sigma = "log",
                           link_hu = "logit"))
\end{verbatim}

Here we include group-level effects for the year, the county, and the
crime-type in the county in case the judicial system sentences the
different crime-types differently relative to other counties. Each of
the population-level regression coefficients are given a normal prior
\(\beta_j \sim N(0, 25).\) The group-level effects for the intercepts
and effects for crime-type are given noncentral t-distributions
\(\gamma_k \sim t_{3; 0, 2.5}\) while the correlations between the
county-level crime-type effects and county-level intercepts are given
\(\rho_{i.j} \sim lkj(1)\) priors. The hurdle parameter gets a
\(U(0,1)\) prior. The error term's variance also gets a noncentral t
prior \(\sigma \sim t_{3; 0, 2.5}.\)

The quantity of interest is the regression coefficient for race.

\hypertarget{tbl-brms-hurdle-fe-int-only}{}
\begin{table}
\caption{\label{tbl-brms-hurdle-fe-int-only}Fixed/Population-level effects for the non-zero part of the lognormal
hurdle model with intercept only for the zero-part. }\tabularnewline

\centering
\begin{tabular}{lrrrr}
\toprule
Term & Estimate & SE & LB 95\% CI & UB 95\% CI\\
\midrule
Intercept & 0.62 & 0.08 & 0.47 & 0.77\\
DOSAGE & 0.06 & 0.02 & 0.02 & 0.10\\
DOSAGEQ & -0.04 & 0.01 & -0.07 & -0.01\\
SEXMale & 0.13 & 0.05 & 0.03 & 0.22\\
OFF\_RACEBLACK & -0.15 & 0.10 & -0.35 & 0.04\\
\addlinespace
OFF\_RACELATINO & 0.09 & 0.40 & -0.69 & 0.88\\
OFF\_RACEOTHER & -0.90 & 0.45 & -1.79 & -0.03\\
OGS & 1.36 & 0.03 & 1.29 & 1.43\\
OGSQ & -0.14 & 0.01 & -0.16 & -0.13\\
PRVREC1D2D3 & 0.39 & 0.04 & 0.32 & 0.47\\
\addlinespace
PRVREC4D5 & 0.93 & 0.06 & 0.82 & 1.04\\
PRVRECREVOCDRFEL & 1.44 & 0.09 & 1.26 & 1.63\\
RECMIN & 0.05 & 0.06 & -0.06 & 0.16\\
CRIMEDUI & -0.99 & 0.05 & -1.09 & -0.88\\
CRIMEOther & -0.06 & 0.06 & -0.17 & 0.06\\
\addlinespace
CRIMEPersons & 0.13 & 0.05 & 0.04 & 0.23\\
CRIMEProperty & 0.10 & 0.05 & 0.00 & 0.19\\
TRIAL & 0.34 & 0.08 & 0.17 & 0.50\\
SEXMale:OFF\_RACEBLACK & 0.10 & 0.11 & -0.10 & 0.31\\
SEXMale:OFF\_RACELATINO & 0.19 & 0.41 & -0.62 & 0.98\\
\addlinespace
SEXMale:OFF\_RACEOTHER & 0.71 & 0.49 & -0.24 & 1.69\\
OGS:PRVREC1D2D3 & -0.04 & 0.03 & -0.10 & 0.02\\
OGS:PRVREC4D5 & -0.07 & 0.03 & -0.14 & 0.00\\
OGS:PRVRECREVOCDRFEL & -0.19 & 0.07 & -0.32 & -0.06\\
\bottomrule
\end{tabular}
\end{table}

The standard deviation parameter \(\sigma\) of the lognormal
distribution has a posterior mean of 1.02 (95\% CI: {[}1, 1.04 {]}). The
hurdle parameter has a posterior mean of 0.52 (95\% CI: {[}0.51, 0.53
{]}) which matches the probability of not being incarcerated in the
overall population.

\hypertarget{tbl-brms1-group-eff}{}
\begin{table}
\caption{\label{tbl-brms1-group-eff}Random/group-level effect standard deviation estimates for the
intercept-only hurdle lognormal model. }\tabularnewline

\centering
\begin{tabular}{llrrr}
\toprule
Term & Estimate & SE & LB 95\% CI & UB 95\% CI\\
\midrule
County-level & sd(Intercept) & 0.18 & 0.13 & 0.24\\
Year-level & sd(Intercept & 0.07 & 0.00 & 0.22\\
\bottomrule
\end{tabular}
\end{table}

From the hurdle model, we find that Black defendants do not receive
different sentence lengths compared to White defendants (95\% CI of
(-0.11, 0.03)). In terms of 95\% CIs we also find that there doesn't
seem to be a difference in sentence lengths between male and female
defendants either between or within each racial group.

\begin{figure}

\begin{minipage}[t]{0.33\linewidth}

{\centering 

\raisebox{-\height}{

\includegraphics{pattern-mixture-modeling_files/figure-pdf/fig-cond-eff-brm-nonzero-1.pdf}

}

}

\subcaption{\label{fig-cond-eff-brm-nonzero-1}Age}
\end{minipage}%
%
\begin{minipage}[t]{0.33\linewidth}

{\centering 

\raisebox{-\height}{

\includegraphics{pattern-mixture-modeling_files/figure-pdf/fig-cond-eff-brm-nonzero-2.pdf}

}

}

\subcaption{\label{fig-cond-eff-brm-nonzero-2}Age Squared}
\end{minipage}%
%
\begin{minipage}[t]{0.33\linewidth}

{\centering 

\raisebox{-\height}{

\includegraphics{pattern-mixture-modeling_files/figure-pdf/fig-cond-eff-brm-nonzero-3.pdf}

}

}

\subcaption{\label{fig-cond-eff-brm-nonzero-3}Sex}
\end{minipage}%
\newline
\begin{minipage}[t]{0.33\linewidth}

{\centering 

\raisebox{-\height}{

\includegraphics{pattern-mixture-modeling_files/figure-pdf/fig-cond-eff-brm-nonzero-4.pdf}

}

}

\subcaption{\label{fig-cond-eff-brm-nonzero-4}Offender Race}
\end{minipage}%
%
\begin{minipage}[t]{0.33\linewidth}

{\centering 

\raisebox{-\height}{

\includegraphics{pattern-mixture-modeling_files/figure-pdf/fig-cond-eff-brm-nonzero-5.pdf}

}

}

\subcaption{\label{fig-cond-eff-brm-nonzero-5}OGS}
\end{minipage}%
%
\begin{minipage}[t]{0.33\linewidth}

{\centering 

\raisebox{-\height}{

\includegraphics{pattern-mixture-modeling_files/figure-pdf/fig-cond-eff-brm-nonzero-6.pdf}

}

}

\subcaption{\label{fig-cond-eff-brm-nonzero-6}OGS Squared}
\end{minipage}%
\newline
\begin{minipage}[t]{0.33\linewidth}

{\centering 

\raisebox{-\height}{

\includegraphics{pattern-mixture-modeling_files/figure-pdf/fig-cond-eff-brm-nonzero-7.pdf}

}

}

\subcaption{\label{fig-cond-eff-brm-nonzero-7}Prev. Rec.}
\end{minipage}%
%
\begin{minipage}[t]{0.33\linewidth}

{\centering 

\raisebox{-\height}{

\includegraphics{pattern-mixture-modeling_files/figure-pdf/fig-cond-eff-brm-nonzero-8.pdf}

}

}

\subcaption{\label{fig-cond-eff-brm-nonzero-8}Rec. Min.}
\end{minipage}%
%
\begin{minipage}[t]{0.33\linewidth}

{\centering 

\raisebox{-\height}{

\includegraphics{pattern-mixture-modeling_files/figure-pdf/fig-cond-eff-brm-nonzero-9.pdf}

}

}

\subcaption{\label{fig-cond-eff-brm-nonzero-9}Crime Type}
\end{minipage}%
\newline
\begin{minipage}[t]{0.33\linewidth}

{\centering 

\raisebox{-\height}{

\includegraphics{pattern-mixture-modeling_files/figure-pdf/fig-cond-eff-brm-nonzero-10.pdf}

}

}

\subcaption{\label{fig-cond-eff-brm-nonzero-10}Trial}
\end{minipage}%
%
\begin{minipage}[t]{0.33\linewidth}

{\centering 

\raisebox{-\height}{

\includegraphics{pattern-mixture-modeling_files/figure-pdf/fig-cond-eff-brm-nonzero-11.pdf}

}

}

\subcaption{\label{fig-cond-eff-brm-nonzero-11}Race by Sex}
\end{minipage}%
%
\begin{minipage}[t]{0.33\linewidth}

{\centering 

\raisebox{-\height}{

\includegraphics{pattern-mixture-modeling_files/figure-pdf/fig-cond-eff-brm-nonzero-12.pdf}

}

}

\subcaption{\label{fig-cond-eff-brm-nonzero-12}OGS by Prev. Rec.}
\end{minipage}%

\caption{\label{fig-cond-eff-brm-nonzero}Posterior estimates of the
conditional effects for the lognormal glm for the non-zero-part of the
intercept-only hurdle lognormal model.}

\end{figure}

\begin{figure}

\begin{minipage}[t]{0.33\linewidth}

{\centering 

\raisebox{-\height}{

\includegraphics{pattern-mixture-modeling_files/figure-pdf/fig-cond-eff-brm1-zero-1.pdf}

}

}

\subcaption{\label{fig-cond-eff-brm1-zero-1}Age}
\end{minipage}%
%
\begin{minipage}[t]{0.33\linewidth}

{\centering 

\raisebox{-\height}{

\includegraphics{pattern-mixture-modeling_files/figure-pdf/fig-cond-eff-brm1-zero-2.pdf}

}

}

\subcaption{\label{fig-cond-eff-brm1-zero-2}Age Squared}
\end{minipage}%
%
\begin{minipage}[t]{0.33\linewidth}

{\centering 

\raisebox{-\height}{

\includegraphics{pattern-mixture-modeling_files/figure-pdf/fig-cond-eff-brm1-zero-3.pdf}

}

}

\subcaption{\label{fig-cond-eff-brm1-zero-3}Sex}
\end{minipage}%
\newline
\begin{minipage}[t]{0.33\linewidth}

{\centering 

\raisebox{-\height}{

\includegraphics{pattern-mixture-modeling_files/figure-pdf/fig-cond-eff-brm1-zero-4.pdf}

}

}

\subcaption{\label{fig-cond-eff-brm1-zero-4}Offender Race}
\end{minipage}%
%
\begin{minipage}[t]{0.33\linewidth}

{\centering 

\raisebox{-\height}{

\includegraphics{pattern-mixture-modeling_files/figure-pdf/fig-cond-eff-brm1-zero-5.pdf}

}

}

\subcaption{\label{fig-cond-eff-brm1-zero-5}OGS}
\end{minipage}%
%
\begin{minipage}[t]{0.33\linewidth}

{\centering 

\raisebox{-\height}{

\includegraphics{pattern-mixture-modeling_files/figure-pdf/fig-cond-eff-brm1-zero-6.pdf}

}

}

\subcaption{\label{fig-cond-eff-brm1-zero-6}OGS Squared}
\end{minipage}%
\newline
\begin{minipage}[t]{0.33\linewidth}

{\centering 

\raisebox{-\height}{

\includegraphics{pattern-mixture-modeling_files/figure-pdf/fig-cond-eff-brm1-zero-7.pdf}

}

}

\subcaption{\label{fig-cond-eff-brm1-zero-7}Prev. Rec.}
\end{minipage}%
%
\begin{minipage}[t]{0.33\linewidth}

{\centering 

\raisebox{-\height}{

\includegraphics{pattern-mixture-modeling_files/figure-pdf/fig-cond-eff-brm1-zero-8.pdf}

}

}

\subcaption{\label{fig-cond-eff-brm1-zero-8}Rec. Min.}
\end{minipage}%
%
\begin{minipage}[t]{0.33\linewidth}

{\centering 

\raisebox{-\height}{

\includegraphics{pattern-mixture-modeling_files/figure-pdf/fig-cond-eff-brm1-zero-9.pdf}

}

}

\subcaption{\label{fig-cond-eff-brm1-zero-9}Crime Type}
\end{minipage}%
\newline
\begin{minipage}[t]{0.33\linewidth}

{\centering 

\raisebox{-\height}{

\includegraphics{pattern-mixture-modeling_files/figure-pdf/fig-cond-eff-brm1-zero-10.pdf}

}

}

\subcaption{\label{fig-cond-eff-brm1-zero-10}Trial}
\end{minipage}%
%
\begin{minipage}[t]{0.33\linewidth}

{\centering 

\raisebox{-\height}{

\includegraphics{pattern-mixture-modeling_files/figure-pdf/fig-cond-eff-brm1-zero-11.pdf}

}

}

\subcaption{\label{fig-cond-eff-brm1-zero-11}Race by Sex}
\end{minipage}%
%
\begin{minipage}[t]{0.33\linewidth}

{\centering 

\raisebox{-\height}{

\includegraphics{pattern-mixture-modeling_files/figure-pdf/fig-cond-eff-brm1-zero-12.pdf}

}

}

\subcaption{\label{fig-cond-eff-brm1-zero-12}OGS by Prev. Rec.}
\end{minipage}%

\caption{\label{fig-cond-eff-brm1-zero}Posterior estimates of the
conditional effects for the logistic regression for the zero-part of the
intercept-only hurdle lognormal model.}

\end{figure}

\begin{figure}

\begin{minipage}[t]{0.50\linewidth}

{\centering 

\raisebox{-\height}{

\includegraphics{pattern-mixture-modeling_files/figure-pdf/fig-ppc-brm1-1.pdf}

}

}

\subcaption{\label{fig-ppc-brm1-1}Density for sentence length}
\end{minipage}%
%
\begin{minipage}[t]{0.50\linewidth}

{\centering 

\raisebox{-\height}{

\includegraphics{pattern-mixture-modeling_files/figure-pdf/fig-ppc-brm1-2.pdf}

}

}

\subcaption{\label{fig-ppc-brm1-2}Density for log of sentence length
plus one day}
\end{minipage}%

\caption{\label{fig-ppc-brm1}Posterior Predictive Checks for the
intercept-only hurdle model.}

\end{figure}

\hypertarget{lognormal-glm-hurdle-model-with-predictors-on-hurdle-parameter}{%
\subsubsection{Lognormal GLM Hurdle Model with Predictors on Hurdle
Parameter}\label{lognormal-glm-hurdle-model-with-predictors-on-hurdle-parameter}}

Next, we modify the model from the previous section to include
predictors in the logistic regression part of the hurdle model
Equation~\ref{eq-hurdle-intercept}. The non-zero portion
Equation~\ref{eq-hurdle-glm} of the hurdle model remains the same.

\[
\mathrm{logit}^{-1}(P(Y_i=0)) = \mathbf{x}_i \boldsymbol{\alpha}.
\]

The coefficients \(\boldsymbol{\alpha}\) are given normal priors
\(\alpha_k \overset{iid}{\sim} N(0, 10).\) The priors for the other
parameters remain the same.

The new model is specified with the \texttt{bf()} function.

\begin{verbatim}
bf(JP_MIN ~ DOSAGE + DOSAGEQ + SEX*OFF_RACE + OGS + OGSQ +
                 PRVREC + RECMIN + CRIME + TRIAL +
                 (1 | YEAR) + (1 + CRIME |COUNTY),
    hu ~ 1 + DOSAGE + DOSAGEQ + SEX * OFF_RACE + OGS + OGSQ +
            PRVREC + CRIME + TRIAL + (1 | COUNTY))
\end{verbatim}

Next we fit the model.

\hypertarget{tbl-brms-hurdle-model-summary-2}{}
\begin{table}
\caption{\label{tbl-brms-hurdle-model-summary-2}Fixed/population-level effects for the non-zero part of the full
lognormal hurdle glm. }\tabularnewline

\centering
\begin{tabular}{lrrrr}
\toprule
Term & Estimate & SE & LB 95\% CI & UB 95\% CI\\
\midrule
Intercept & 4.03 & 0.08 & 3.87 & 4.19\\
DOSAGE & 0.06 & 0.02 & 0.02 & 0.09\\
DOSAGEQ & -0.04 & 0.01 & -0.07 & -0.01\\
SEXMale & 0.13 & 0.05 & 0.03 & 0.23\\
OFF\_RACEBLACK & -0.16 & 0.10 & -0.35 & 0.04\\
\addlinespace
OFF\_RACELATINO & 0.09 & 0.39 & -0.68 & 0.86\\
OFF\_RACEOTHER & -0.90 & 0.45 & -1.78 & 0.00\\
OGS & 1.36 & 0.03 & 1.29 & 1.43\\
OGSQ & -0.14 & 0.01 & -0.16 & -0.12\\
PRVREC1D2D3 & 0.39 & 0.04 & 0.31 & 0.47\\
\addlinespace
PRVREC4D5 & 0.93 & 0.05 & 0.83 & 1.04\\
PRVRECREVOCDRFEL & 1.44 & 0.09 & 1.26 & 1.63\\
RECMIN & 0.05 & 0.06 & -0.06 & 0.17\\
CRIMEDUI & -0.99 & 0.05 & -1.09 & -0.88\\
CRIMEOther & -0.05 & 0.06 & -0.17 & 0.06\\
\addlinespace
CRIMEPersons & 0.14 & 0.05 & 0.04 & 0.24\\
CRIMEProperty & 0.10 & 0.05 & 0.00 & 0.19\\
TRIAL & 0.34 & 0.09 & 0.17 & 0.50\\
SEXMale:OFF\_RACEBLACK & 0.11 & 0.11 & -0.10 & 0.31\\
SEXMale:OFF\_RACELATINO & 0.19 & 0.40 & -0.60 & 0.99\\
\addlinespace
SEXMale:OFF\_RACEOTHER & 0.71 & 0.49 & -0.27 & 1.68\\
OGS:PRVREC1D2D3 & -0.04 & 0.03 & -0.09 & 0.02\\
OGS:PRVREC4D5 & -0.07 & 0.03 & -0.14 & -0.01\\
OGS:PRVRECREVOCDRFEL & -0.19 & 0.07 & -0.32 & -0.06\\
\bottomrule
\end{tabular}
\end{table}

\hypertarget{tbl-brms-hurdle-model-summary-2-zero}{}
\begin{table}
\caption{\label{tbl-brms-hurdle-model-summary-2-zero}Fixed/population-level effects for the zero part of the full lognormal
hurdle glm. }\tabularnewline

\centering
\begin{tabular}{lrrrr}
\toprule
Term & Estimate & SE & LB 95\% CI & UB 95\% CI\\
\midrule
hu\_Intercept & 1.77 & 0.14 & 1.50 & 2.04\\
hu\_SEXMale & -0.38 & 0.07 & -0.52 & -0.24\\
hu\_OFF\_RACEBLACK & -0.10 & 0.14 & -0.38 & 0.18\\
hu\_OFF\_RACELATINO & -0.74 & 0.67 & -2.04 & 0.58\\
hu\_OFF\_RACEOTHER & -0.61 & 0.68 & -1.89 & 0.76\\
\addlinespace
hu\_RECMIN & -0.95 & 0.09 & -1.13 & -0.76\\
hu\_OGS & -0.32 & 0.05 & -0.41 & -0.23\\
hu\_OGSQ & -0.26 & 0.03 & -0.31 & -0.20\\
hu\_PRVREC1D2D3 & -0.44 & 0.06 & -0.55 & -0.33\\
hu\_PRVREC4D5 & -1.11 & 0.09 & -1.29 & -0.94\\
\addlinespace
hu\_PRVRECREVOCDRFEL & -1.43 & 0.18 & -1.78 & -1.08\\
hu\_CRIMEDUI & -1.69 & 0.08 & -1.84 & -1.54\\
hu\_CRIMEOther & -0.20 & 0.09 & -0.38 & -0.03\\
hu\_CRIMEPersons & -0.83 & 0.08 & -0.99 & -0.66\\
hu\_CRIMEProperty & -0.43 & 0.07 & -0.57 & -0.29\\
\addlinespace
hu\_TRIAL & -0.59 & 0.20 & -0.99 & -0.21\\
hu\_SEXMale:OFF\_RACEBLACK & -0.21 & 0.16 & -0.52 & 0.10\\
hu\_SEXMale:OFF\_RACELATINO & 0.07 & 0.71 & -1.33 & 1.44\\
hu\_SEXMale:OFF\_RACEOTHER & 0.56 & 0.75 & -0.95 & 1.99\\
\bottomrule
\end{tabular}
\end{table}

The standard deviation parameter \(\sigma\) of the lognormal
distribution has a posterior mean of 1.02 (95\% CI: {[}1, 1.04{]}).

The county-level random/group-level effects and year-level
random/group-level effects are reported in Table~\ref{tbl-brms2-re}.

\hypertarget{tbl-brms2-re}{}
\begin{table}
\caption{\label{tbl-brms2-re}Random/group-level effect standard deviation estimates for the full
hurdle lognormal model. }\tabularnewline

\centering
\begin{tabular}{llrrr}
\toprule
Term & Estimate & SE & LB 95\% CI & UB 95\% CI\\
\midrule
County-level & sd(Intercept) & 0.18 & 0.13 & 0.24\\
County-level & sd(hu\_Intercept) & 0.80 & 0.65 & 0.99\\
Year-level & sd(Intercept & 0.07 & 0.00 & 0.21\\
\bottomrule
\end{tabular}
\end{table}

\begin{figure}

\begin{minipage}[t]{0.33\linewidth}

{\centering 

\raisebox{-\height}{

\includegraphics{pattern-mixture-modeling_files/figure-pdf/fig-cond-eff-brm2-nonzero-1.pdf}

}

}

\subcaption{\label{fig-cond-eff-brm2-nonzero-1}Age}
\end{minipage}%
%
\begin{minipage}[t]{0.33\linewidth}

{\centering 

\raisebox{-\height}{

\includegraphics{pattern-mixture-modeling_files/figure-pdf/fig-cond-eff-brm2-nonzero-2.pdf}

}

}

\subcaption{\label{fig-cond-eff-brm2-nonzero-2}Age Squared}
\end{minipage}%
%
\begin{minipage}[t]{0.33\linewidth}

{\centering 

\raisebox{-\height}{

\includegraphics{pattern-mixture-modeling_files/figure-pdf/fig-cond-eff-brm2-nonzero-3.pdf}

}

}

\subcaption{\label{fig-cond-eff-brm2-nonzero-3}Sex}
\end{minipage}%
\newline
\begin{minipage}[t]{0.33\linewidth}

{\centering 

\raisebox{-\height}{

\includegraphics{pattern-mixture-modeling_files/figure-pdf/fig-cond-eff-brm2-nonzero-4.pdf}

}

}

\subcaption{\label{fig-cond-eff-brm2-nonzero-4}Offender Race}
\end{minipage}%
%
\begin{minipage}[t]{0.33\linewidth}

{\centering 

\raisebox{-\height}{

\includegraphics{pattern-mixture-modeling_files/figure-pdf/fig-cond-eff-brm2-nonzero-5.pdf}

}

}

\subcaption{\label{fig-cond-eff-brm2-nonzero-5}OGS}
\end{minipage}%
%
\begin{minipage}[t]{0.33\linewidth}

{\centering 

\raisebox{-\height}{

\includegraphics{pattern-mixture-modeling_files/figure-pdf/fig-cond-eff-brm2-nonzero-6.pdf}

}

}

\subcaption{\label{fig-cond-eff-brm2-nonzero-6}OGS Squared}
\end{minipage}%
\newline
\begin{minipage}[t]{0.33\linewidth}

{\centering 

\raisebox{-\height}{

\includegraphics{pattern-mixture-modeling_files/figure-pdf/fig-cond-eff-brm2-nonzero-7.pdf}

}

}

\subcaption{\label{fig-cond-eff-brm2-nonzero-7}Prev. Rec.}
\end{minipage}%
%
\begin{minipage}[t]{0.33\linewidth}

{\centering 

\raisebox{-\height}{

\includegraphics{pattern-mixture-modeling_files/figure-pdf/fig-cond-eff-brm2-nonzero-8.pdf}

}

}

\subcaption{\label{fig-cond-eff-brm2-nonzero-8}Rec. Min.}
\end{minipage}%
%
\begin{minipage}[t]{0.33\linewidth}

{\centering 

\raisebox{-\height}{

\includegraphics{pattern-mixture-modeling_files/figure-pdf/fig-cond-eff-brm2-nonzero-9.pdf}

}

}

\subcaption{\label{fig-cond-eff-brm2-nonzero-9}Crime Type}
\end{minipage}%
\newline
\begin{minipage}[t]{0.33\linewidth}

{\centering 

\raisebox{-\height}{

\includegraphics{pattern-mixture-modeling_files/figure-pdf/fig-cond-eff-brm2-nonzero-10.pdf}

}

}

\subcaption{\label{fig-cond-eff-brm2-nonzero-10}Trial}
\end{minipage}%
%
\begin{minipage}[t]{0.33\linewidth}

{\centering 

\raisebox{-\height}{

\includegraphics{pattern-mixture-modeling_files/figure-pdf/fig-cond-eff-brm2-nonzero-11.pdf}

}

}

\subcaption{\label{fig-cond-eff-brm2-nonzero-11}Race by Sex}
\end{minipage}%
%
\begin{minipage}[t]{0.33\linewidth}

{\centering 

\raisebox{-\height}{

\includegraphics{pattern-mixture-modeling_files/figure-pdf/fig-cond-eff-brm2-nonzero-12.pdf}

}

}

\subcaption{\label{fig-cond-eff-brm2-nonzero-12}OGS by Prev. Rec.}
\end{minipage}%

\caption{\label{fig-cond-eff-brm2-nonzero}Posterior estimates of the
conditional effects for the lognormal glm for the non-zero-part of the
full hurdle lognormal model.}

\end{figure}

\begin{figure}

\begin{minipage}[t]{0.33\linewidth}

{\centering 

\raisebox{-\height}{

\includegraphics{pattern-mixture-modeling_files/figure-pdf/fig-cond-eff-brm2-1.pdf}

}

}

\subcaption{\label{fig-cond-eff-brm2-1}Age}
\end{minipage}%
%
\begin{minipage}[t]{0.33\linewidth}

{\centering 

\raisebox{-\height}{

\includegraphics{pattern-mixture-modeling_files/figure-pdf/fig-cond-eff-brm2-2.pdf}

}

}

\subcaption{\label{fig-cond-eff-brm2-2}Age Squared}
\end{minipage}%
%
\begin{minipage}[t]{0.33\linewidth}

{\centering 

\raisebox{-\height}{

\includegraphics{pattern-mixture-modeling_files/figure-pdf/fig-cond-eff-brm2-3.pdf}

}

}

\subcaption{\label{fig-cond-eff-brm2-3}Sex}
\end{minipage}%
\newline
\begin{minipage}[t]{0.33\linewidth}

{\centering 

\raisebox{-\height}{

\includegraphics{pattern-mixture-modeling_files/figure-pdf/fig-cond-eff-brm2-4.pdf}

}

}

\subcaption{\label{fig-cond-eff-brm2-4}Offender Race}
\end{minipage}%
%
\begin{minipage}[t]{0.33\linewidth}

{\centering 

\raisebox{-\height}{

\includegraphics{pattern-mixture-modeling_files/figure-pdf/fig-cond-eff-brm2-5.pdf}

}

}

\subcaption{\label{fig-cond-eff-brm2-5}OGS}
\end{minipage}%
%
\begin{minipage}[t]{0.33\linewidth}

{\centering 

\raisebox{-\height}{

\includegraphics{pattern-mixture-modeling_files/figure-pdf/fig-cond-eff-brm2-6.pdf}

}

}

\subcaption{\label{fig-cond-eff-brm2-6}OGS Squared}
\end{minipage}%
\newline
\begin{minipage}[t]{0.33\linewidth}

{\centering 

\raisebox{-\height}{

\includegraphics{pattern-mixture-modeling_files/figure-pdf/fig-cond-eff-brm2-7.pdf}

}

}

\subcaption{\label{fig-cond-eff-brm2-7}Prev. Rec.}
\end{minipage}%
%
\begin{minipage}[t]{0.33\linewidth}

{\centering 

\raisebox{-\height}{

\includegraphics{pattern-mixture-modeling_files/figure-pdf/fig-cond-eff-brm2-8.pdf}

}

}

\subcaption{\label{fig-cond-eff-brm2-8}Rec. Min.}
\end{minipage}%
%
\begin{minipage}[t]{0.33\linewidth}

{\centering 

\raisebox{-\height}{

\includegraphics{pattern-mixture-modeling_files/figure-pdf/fig-cond-eff-brm2-9.pdf}

}

}

\subcaption{\label{fig-cond-eff-brm2-9}Crime Type}
\end{minipage}%
\newline
\begin{minipage}[t]{0.33\linewidth}

{\centering 

\raisebox{-\height}{

\includegraphics{pattern-mixture-modeling_files/figure-pdf/fig-cond-eff-brm2-10.pdf}

}

}

\subcaption{\label{fig-cond-eff-brm2-10}Trial}
\end{minipage}%
%
\begin{minipage}[t]{0.33\linewidth}

{\centering 

\raisebox{-\height}{

\includegraphics{pattern-mixture-modeling_files/figure-pdf/fig-cond-eff-brm2-11.pdf}

}

}

\subcaption{\label{fig-cond-eff-brm2-11}Race by Sex}
\end{minipage}%
%
\begin{minipage}[t]{0.33\linewidth}

{\centering 

\raisebox{-\height}{

\includegraphics{pattern-mixture-modeling_files/figure-pdf/fig-cond-eff-brm2-12.pdf}

}

}

\subcaption{\label{fig-cond-eff-brm2-12}OGS by Prev. Rec.}
\end{minipage}%

\caption{\label{fig-cond-eff-brm2}Posterior estimates of the conditional
effects for the logistic regression on the zero-part of the full hurdle
lognormal model.}

\end{figure}

\begin{figure}

\begin{minipage}[t]{0.33\linewidth}

{\centering 

\raisebox{-\height}{

\includegraphics{pattern-mixture-modeling_files/figure-pdf/fig-ppc-brm2-1.pdf}

}

}

\subcaption{\label{fig-ppc-brm2-1}PPC Density Sentence Length (days)}
\end{minipage}%
%
\begin{minipage}[t]{0.33\linewidth}

{\centering 

\raisebox{-\height}{

\includegraphics{pattern-mixture-modeling_files/figure-pdf/fig-ppc-brm2-2.pdf}

}

}

\subcaption{\label{fig-ppc-brm2-2}PPC Density Log of Sentence Length
plus 1}
\end{minipage}%
%
\begin{minipage}[t]{0.33\linewidth}

{\centering 

\raisebox{-\height}{

\includegraphics{pattern-mixture-modeling_files/figure-pdf/fig-ppc-brm2-3.pdf}

}

}

\subcaption{\label{fig-ppc-brm2-3}PPC Scatter of average error by age}
\end{minipage}%

\caption{\label{fig-ppc-brm2}Posterior Predictive Checks for the full
lognormal hurdle model.}

\end{figure}

\hypertarget{glmmadaptive-lognormal-hurdle-model}{%
\subsubsection{GLMMadaptive Lognormal Hurdle
Model}\label{glmmadaptive-lognormal-hurdle-model}}

Complete case analysis to see if the function works and how quickly it
runs.

Evaluation of predictions

\begin{figure}

{\centering \includegraphics{pattern-mixture-modeling_files/figure-pdf/fig-cdf-glmm-lnh-pred-1.pdf}

}

\caption{\label{fig-cdf-glmm-lnh-pred}The empirical CDF of the observed
sentence lengths (days, log scale) with fitted cdfs simulated from the
model fit overlaid in light gray.}

\end{figure}

Fitting with MI.

\hypertarget{tbl-coef-comp}{}
\begin{table}
\caption{\label{tbl-coef-comp}Coefficients from the non-zero portions of the full lognormal hurdle
models fit with brms and glmmadaptive after multiple imputation. }\tabularnewline

\centering
\begin{tabular}{lrrrr}
\toprule
Term & brms - Est. & glmma - Est. & brms - Hurdle Est. & glmma - Hurdle Est.\\
\midrule
Intercept & 4.0333 & -0.0299 & 1.8038 & 1.9404\\
DOSAGE & 0.0557 & 0.0612 & 0.1593 & 0.1409\\
DOSAGEQ & -0.0398 & -0.0406 & 0.0151 & 0.0030\\
SEXMale & 0.1288 & 0.1392 & -0.3607 & -0.3682\\
OFF\_RACEBLACK & -0.1558 & -0.1632 & -0.1087 & 0.0583\\
\addlinespace
OFF\_RACELATINO & 0.0892 & 0.1131 & -0.7168 & -0.8419\\
OFF\_RACEOTHER & -0.9008 & -0.4966 & -0.5718 & -0.5529\\
OGS & 1.3618 & 1.3638 & -0.3214 & -0.1963\\
OGSQ & -0.1441 & 0.3823 & -0.2516 & -0.2628\\
PRVREC1D2D3 & 0.3895 & 0.9116 & -0.4814 & -0.4153\\
\addlinespace
PRVREC4D5 & 0.9319 & 1.3744 & -1.2312 & -1.0177\\
PRVRECREVOCDRFEL & 1.4442 & -0.1482 & -1.6192 & -1.1936\\
RECMIN & 0.0542 & 0.0608 & -0.9510 & -0.8402\\
CRIMEDUI & -0.9868 & -0.9812 & -1.7938 & -1.6576\\
CRIMEOther & -0.0544 & -0.0384 & -0.2276 & -0.2364\\
\addlinespace
CRIMEPersons & 0.1362 & 0.1462 & -0.8663 & -0.8366\\
CRIMEProperty & 0.0998 & 0.1083 & -0.4473 & -0.4826\\
TRIAL & 0.3353 & 0.3266 & -0.6359 & -0.5492\\
SEXMale:OFF\_RACEBLACK & 0.1056 & 0.0910 & -0.1811 & -0.1562\\
SEXMale:OFF\_RACELATINO & 0.1885 & 0.1911 & 0.0642 & -0.0577\\
\addlinespace
SEXMale:OFF\_RACEOTHER & 0.7088 & 0.2618 & 0.4684 & 0.4947\\
OGS:PRVREC1D2D3 & -0.0357 & -0.0385 & NA & NA\\
OGS:PRVREC4D5 & -0.0717 & -0.0769 & NA & NA\\
OGS:PRVRECREVOCDRFEL & -0.1906 & -0.1915 & NA & NA\\
\bottomrule
\end{tabular}
\end{table}

\hypertarget{sensitivity-analysis}{%
\section{Sensitivity Analysis}\label{sensitivity-analysis}}

Evaluating the impacts of various nonignorable missingness mechanisms
can be accomplished with pattern-mixture models. The values of the
incomplete numeric data can be scaled or shifted. The same cannot be
done with categorical imputations, instead the proportion of each
category can be varied within the imputations compared to the observed
distribution or the MAR imputed distribution.

While there are very few missing sentence lengths, I could also modify
the imputed sentence lengths with a scale \(c\) or shift \(\delta\).
Scaling maintains that offenders that are non-incarcerated will remain
non-incarcerated while the sentences of incarcerated defenders would
shift. A positive shift would make all offenders incarcerated while a
negative shift would impose negative sentence lengths that would need to
be corrected to use Poisson or lognormal hurdle glms.

To perturb the imputations, we modify the vector of probabilities
\(\mathbf{p} = (p_1, p_2, p_3, p_4)'\) of class membership for each
racial/ethnic group under consideration (White, Black, Latino, or
Other). A vector of scale parameters
\(\mathbf{c} = (c_1, c_2, c_3, c_4)'\) is chosen such that the new
racial/ethnic group label will be drawn from the normalized vector
\(\mathbf{p}^* = \frac{1}{\sum_{j=1}^4 c_j p_j} (c_1 p_1, c_2 p_2, c_3 p_3, c_4 p_4)'\)
so that the probability vectors remains the same if \(c_j = 1\) for each
\(j = 1,\dots,4.\) For simplicity, we will focus on varying the
probability of assigning a White label by changing only \(c_1\) while
holding \(c_2 = c_3 = c_4 = 1.\)

The vector of probabilities \(\mathbf{p}_i\) is obtained for each
incomplete observation \(i\) by re-fitting the same random forest used
in the initial mulitple imputation step. This vector is then scaled and
normalized using the same scaling vector \(\mathbf{c}\) for all
incomplete observations. The probability vectors \(\mathbf{p}^*\) are
used to draw new imputed racial/ethnic group labels for the incomplete
observations. This procedure is repeated, including the model fitting,
across each of the \(M\) completed data sets resulting in a new set of
\(M\) completed data sets with modified imputations. The new set of
completed data sets is then analyzed as before using an lognormal hurdle
model. Estimates from each variation of \(\mathbf{c}\) that is chosen
are then compared graphically in Figure~\ref{fig-sens-analysis-res}.

We choose to vary \(c_1\) along the sequence
\(\{0.1, 0.33, 0.75, 0.9, 1, 1.1, 1.25, 1.5, 2\}.\) When \(c_1 = 1\),
the imputations correspond to the original imputation model under the
MAR assumption. For \(c_1 < 1\), fewer observations are imputed with a
White label than under the MAR assumption reflecting a more diverse
population of offenders who have unknown or unreported race labels. The
opposite is true for \(c_1 > 1\), reflecting a more White population of
offenders. We include cases such as \(c_1 = 0.1\) and \(c_1 = 0.33\) as
well as \(c_1 = 1.5\) and \(c_1 = 2\) as rather implausible extreme
conditions to investigate what could happen in the most extreme
circumstances.

\begin{figure}

{\centering \includegraphics{pattern-mixture-modeling_files/figure-pdf/fig-sens-analysis-res-1.pdf}

}

\caption{\label{fig-sens-analysis-res}Coefficient estimates for the
logistic regression model predicting incarceration.}

\end{figure}

\newpage

\hypertarget{references}{%
\section*{References}\label{references}}
\addcontentsline{toc}{section}{References}

\hypertarget{refs}{}
\begin{CSLReferences}{1}{0}
\leavevmode\vadjust pre{\hypertarget{ref-harelInferencesMissingInformation2007}{}}%
Harel, Ofer. 2007. {``Inferences on Missing Information Under Multiple
Imputation and Two-Stage Multiple Imputation.''} \emph{Statistical
Methodology} 4 (1): 75--89.
\url{https://doi.org/10.1016/j.stamet.2006.03.002}.

\leavevmode\vadjust pre{\hypertarget{ref-rubinMultipleImputationNonresponse1987}{}}%
Rubin, Donald B. 1987. \emph{{Multiple Imputation for Nonresponse in
Surveys \textbar{} Wiley Series in Probability and Statistics}}. {Wiley
series in probability and mathematical statistics : Applied probability
and statistics}. {New York}: {Wiley}.

\end{CSLReferences}



\end{document}
