% Options for packages loaded elsewhere
\PassOptionsToPackage{unicode}{hyperref}
\PassOptionsToPackage{hyphens}{url}
\PassOptionsToPackage{dvipsnames,svgnames,x11names}{xcolor}
%
\documentclass[
  letterpaper,
  DIV=11,
  numbers=noendperiod]{scrartcl}

\usepackage{amsmath,amssymb}
\usepackage{iftex}
\ifPDFTeX
  \usepackage[T1]{fontenc}
  \usepackage[utf8]{inputenc}
  \usepackage{textcomp} % provide euro and other symbols
\else % if luatex or xetex
  \usepackage{unicode-math}
  \defaultfontfeatures{Scale=MatchLowercase}
  \defaultfontfeatures[\rmfamily]{Ligatures=TeX,Scale=1}
\fi
\usepackage{lmodern}
\ifPDFTeX\else  
    % xetex/luatex font selection
\fi
% Use upquote if available, for straight quotes in verbatim environments
\IfFileExists{upquote.sty}{\usepackage{upquote}}{}
\IfFileExists{microtype.sty}{% use microtype if available
  \usepackage[]{microtype}
  \UseMicrotypeSet[protrusion]{basicmath} % disable protrusion for tt fonts
}{}
\makeatletter
\@ifundefined{KOMAClassName}{% if non-KOMA class
  \IfFileExists{parskip.sty}{%
    \usepackage{parskip}
  }{% else
    \setlength{\parindent}{0pt}
    \setlength{\parskip}{6pt plus 2pt minus 1pt}}
}{% if KOMA class
  \KOMAoptions{parskip=half}}
\makeatother
\usepackage{xcolor}
\setlength{\emergencystretch}{3em} % prevent overfull lines
\setcounter{secnumdepth}{2}
% Make \paragraph and \subparagraph free-standing
\ifx\paragraph\undefined\else
  \let\oldparagraph\paragraph
  \renewcommand{\paragraph}[1]{\oldparagraph{#1}\mbox{}}
\fi
\ifx\subparagraph\undefined\else
  \let\oldsubparagraph\subparagraph
  \renewcommand{\subparagraph}[1]{\oldsubparagraph{#1}\mbox{}}
\fi


\providecommand{\tightlist}{%
  \setlength{\itemsep}{0pt}\setlength{\parskip}{0pt}}\usepackage{longtable,booktabs,array}
\usepackage{calc} % for calculating minipage widths
% Correct order of tables after \paragraph or \subparagraph
\usepackage{etoolbox}
\makeatletter
\patchcmd\longtable{\par}{\if@noskipsec\mbox{}\fi\par}{}{}
\makeatother
% Allow footnotes in longtable head/foot
\IfFileExists{footnotehyper.sty}{\usepackage{footnotehyper}}{\usepackage{footnote}}
\makesavenoteenv{longtable}
\usepackage{graphicx}
\makeatletter
\def\maxwidth{\ifdim\Gin@nat@width>\linewidth\linewidth\else\Gin@nat@width\fi}
\def\maxheight{\ifdim\Gin@nat@height>\textheight\textheight\else\Gin@nat@height\fi}
\makeatother
% Scale images if necessary, so that they will not overflow the page
% margins by default, and it is still possible to overwrite the defaults
% using explicit options in \includegraphics[width, height, ...]{}
\setkeys{Gin}{width=\maxwidth,height=\maxheight,keepaspectratio}
% Set default figure placement to htbp
\makeatletter
\def\fps@figure{htbp}
\makeatother
\newlength{\cslhangindent}
\setlength{\cslhangindent}{1.5em}
\newlength{\csllabelwidth}
\setlength{\csllabelwidth}{3em}
\newlength{\cslentryspacingunit} % times entry-spacing
\setlength{\cslentryspacingunit}{\parskip}
\newenvironment{CSLReferences}[2] % #1 hanging-ident, #2 entry spacing
 {% don't indent paragraphs
  \setlength{\parindent}{0pt}
  % turn on hanging indent if param 1 is 1
  \ifodd #1
  \let\oldpar\par
  \def\par{\hangindent=\cslhangindent\oldpar}
  \fi
  % set entry spacing
  \setlength{\parskip}{#2\cslentryspacingunit}
 }%
 {}
\usepackage{calc}
\newcommand{\CSLBlock}[1]{#1\hfill\break}
\newcommand{\CSLLeftMargin}[1]{\parbox[t]{\csllabelwidth}{#1}}
\newcommand{\CSLRightInline}[1]{\parbox[t]{\linewidth - \csllabelwidth}{#1}\break}
\newcommand{\CSLIndent}[1]{\hspace{\cslhangindent}#1}

\KOMAoption{captions}{tableheading}
\makeatletter
\makeatother
\makeatletter
\makeatother
\makeatletter
\@ifpackageloaded{caption}{}{\usepackage{caption}}
\AtBeginDocument{%
\ifdefined\contentsname
  \renewcommand*\contentsname{Table of contents}
\else
  \newcommand\contentsname{Table of contents}
\fi
\ifdefined\listfigurename
  \renewcommand*\listfigurename{List of Figures}
\else
  \newcommand\listfigurename{List of Figures}
\fi
\ifdefined\listtablename
  \renewcommand*\listtablename{List of Tables}
\else
  \newcommand\listtablename{List of Tables}
\fi
\ifdefined\figurename
  \renewcommand*\figurename{Figure}
\else
  \newcommand\figurename{Figure}
\fi
\ifdefined\tablename
  \renewcommand*\tablename{Table}
\else
  \newcommand\tablename{Table}
\fi
}
\@ifpackageloaded{float}{}{\usepackage{float}}
\floatstyle{ruled}
\@ifundefined{c@chapter}{\newfloat{codelisting}{h}{lop}}{\newfloat{codelisting}{h}{lop}[chapter]}
\floatname{codelisting}{Listing}
\newcommand*\listoflistings{\listof{codelisting}{List of Listings}}
\makeatother
\makeatletter
\@ifpackageloaded{caption}{}{\usepackage{caption}}
\@ifpackageloaded{subcaption}{}{\usepackage{subcaption}}
\makeatother
\makeatletter
\@ifpackageloaded{tcolorbox}{}{\usepackage[skins,breakable]{tcolorbox}}
\makeatother
\makeatletter
\@ifundefined{shadecolor}{\definecolor{shadecolor}{rgb}{.97, .97, .97}}
\makeatother
\makeatletter
\makeatother
\makeatletter
\makeatother
\ifLuaTeX
  \usepackage{selnolig}  % disable illegal ligatures
\fi
\IfFileExists{bookmark.sty}{\usepackage{bookmark}}{\usepackage{hyperref}}
\IfFileExists{xurl.sty}{\usepackage{xurl}}{} % add URL line breaks if available
\urlstyle{same} % disable monospaced font for URLs
\hypersetup{
  pdftitle={Sentencing Analysis with Pattern Mixture Modeling},
  pdfauthor={C. Clare Strange; Benjamin Stockton; Ofer Harel},
  pdfkeywords={incomplete data, pattern-mixture model},
  colorlinks=true,
  linkcolor={blue},
  filecolor={Maroon},
  citecolor={Blue},
  urlcolor={Blue},
  pdfcreator={LaTeX via pandoc}}

\title{Sentencing Analysis with Pattern Mixture Modeling}
\author{C. Clare Strange \and Benjamin Stockton \and Ofer Harel}
\date{2024-01-03}

\begin{document}
\maketitle
\ifdefined\Shaded\renewenvironment{Shaded}{\begin{tcolorbox}[breakable, enhanced, interior hidden, sharp corners, frame hidden, boxrule=0pt, borderline west={3pt}{0pt}{shadecolor}]}{\end{tcolorbox}}\fi

\hypertarget{methods}{%
\section{Methods}\label{methods}}

\hypertarget{analysis}{%
\section{Analysis}\label{analysis}}

\begin{figure}

{\centering \includegraphics{pattern-mixture-modeling_files/figure-pdf/fig-miss-pattern-1.pdf}

}

\caption{\label{fig-miss-pattern}Missing data patterns for the full data
set.}

\end{figure}

\hypertarget{tbl-yearly-summary}{}
\begin{table}
\caption{\label{tbl-yearly-summary}Summary statistics on sentence length (months) and incarceration. }\tabularnewline

\centering
\begin{tabular}{l|r|r|r|r|r|r}
\hline
Year & Mean & SD & Median & Min. & Max. & P(Incar)\\
\hline
2010 & 5.64 & 32.60 & 0 & 0 & 7571.97 & 0.48\\
\hline
2011 & 5.78 & 20.73 & 0 & 0 & 1008.02 & 0.48\\
\hline
2012 & 5.57 & 21.82 & 0 & 0 & 1253.97 & 0.47\\
\hline
2013 & 5.86 & 21.41 & 0 & 0 & 1014.06 & 0.48\\
\hline
2014 & 5.55 & 22.34 & 0 & 0 & 2399.97 & 0.47\\
\hline
2015 & 5.24 & 19.76 & 0 & 0 & 984.03 & 0.46\\
\hline
2016 & 4.90 & 20.88 & 0 & 0 & 1055.98 & 0.42\\
\hline
2017 & 5.18 & 25.42 & 0 & 0 & 2879.98 & 0.44\\
\hline
2018 & 4.94 & 19.22 & 0 & 0 & 840.03 & 0.44\\
\hline
2019 & 4.81 & 19.73 & 0 & 0 & 938.96 & 0.43\\
\hline
\end{tabular}
\end{table}

\begin{figure}

{\centering \includegraphics{pattern-mixture-modeling_files/figure-pdf/fig-sen-len-yearly-1.pdf}

}

\caption{\label{fig-sen-len-yearly}Density plots for A) Sentence length
(mon.) by year. B) Log sentence lengths (mon.) plus one month by year.}

\end{figure}

\begin{figure}

{\centering \includegraphics{pattern-mixture-modeling_files/figure-pdf/fig-sen-len-crime-1.pdf}

}

\caption{\label{fig-sen-len-crime}A) Sentence length (mon.) by most
serious crime type. B) Log of sentence length plus one month by most
serious crime type.}

\end{figure}

\begin{figure}

{\centering \includegraphics{pattern-mixture-modeling_files/figure-pdf/fig-sen-len-race-1.pdf}

}

\caption{\label{fig-sen-len-race}A) Sentence length by offender race. B)
Log of sentence length (plus one month)}

\end{figure}

\begin{figure}

{\centering \includegraphics{pattern-mixture-modeling_files/figure-pdf/fig-incar-cime-1.pdf}

}

\caption{\label{fig-incar-cime}Incarceration decision by most serious
crime type. INCAR == 1 indicates incarceration. INCAR == 0 indicates
parole.}

\end{figure}

\begin{figure}

\begin{minipage}[t]{0.50\linewidth}

{\centering 

\raisebox{-\height}{

\includegraphics{pattern-mixture-modeling_files/figure-pdf/fig-sen-len-qq-1.pdf}

}

}

\subcaption{\label{fig-sen-len-qq-1}Sentence length in months}
\end{minipage}%
%
\begin{minipage}[t]{0.50\linewidth}

{\centering 

\raisebox{-\height}{

\includegraphics{pattern-mixture-modeling_files/figure-pdf/fig-sen-len-qq-2.pdf}

}

}

\subcaption{\label{fig-sen-len-qq-2}Log of sentence length plus one
month}
\end{minipage}%

\caption{\label{fig-sen-len-qq}QQ plots of the sentence length (months)}

\end{figure}

\hypertarget{multiple-imputation}{%
\subsection{Multiple Imputation}\label{multiple-imputation}}

I'll prepare the data for analysis next. First, I'll center and scale
the numeric predictors which are Offense Gravity Score (OGS), defendant
age, and the square of each. Then I'll use MICE to multiply impute the
incomplete variables with random forest. For now I'll use \(M = 10\)
since less than 5\% of all observations are missing (mainly in defendant
race).

\hypertarget{logistic-regression-on-incarceration}{%
\subsection{Logistic Regression on
Incarceration}\label{logistic-regression-on-incarceration}}

I'll fit a logistic regression like before as a sanity check. The
estimated odds ratio for increased odds of incarceration for a Black
defendant over a White defendant should be roughly 1.25 as we saw in the
complete case analysis in the previous paper.

\hypertarget{tbl-glm-summary}{}
\begin{table}
\caption{\label{tbl-glm-summary}Summary of the logistic regression fit for predicting sentencing
decision (In/Out) excluding the year and county estimates for brevity. }\tabularnewline

\centering
\begin{tabular}{l|r|r|r|r|l}
\hline
Term & Estimate & SE & LB 95\% CI & UB 95\% CI & Sig.\\
\hline
(Intercept) & -2.82 & 0.27 & -3.34 & -2.30 & *\\
\hline
DOSAGE & -0.16 & 0.15 & -0.46 & 0.14 & \\
\hline
DOSAGEQ & 0.02 & 0.15 & -0.27 & 0.32 & \\
\hline
SEXMale & 0.26 & 0.07 & 0.13 & 0.40 & *\\
\hline
OFF\_RACEBLACK & -0.11 & 0.13 & -0.37 & 0.15 & \\
\hline
OFF\_RACELATINO & -0.03 & 0.68 & -1.37 & 1.32 & \\
\hline
OFF\_RACEOTHER & -0.17 & 0.62 & -1.38 & 1.04 & \\
\hline
OGS & -0.66 & 0.11 & -0.87 & -0.45 & *\\
\hline
OGSQ & 1.38 & 0.13 & 1.12 & 1.63 & *\\
\hline
PRVREC1/2/3 & 0.60 & 0.06 & 0.48 & 0.72 & *\\
\hline
PRVREC4/5 & 1.22 & 0.10 & 1.03 & 1.42 & *\\
\hline
PRVRECREVOC/RFEL & 1.53 & 0.20 & 1.14 & 1.92 & *\\
\hline
RECMIN & 0.99 & 0.10 & 0.80 & 1.18 & *\\
\hline
CRIMEDUI & 1.76 & 0.08 & 1.60 & 1.91 & *\\
\hline
CRIMEOther & 0.30 & 0.09 & 0.12 & 0.48 & *\\
\hline
CRIMEPersons & 0.84 & 0.08 & 0.68 & 1.00 & *\\
\hline
CRIMEProperty & 0.53 & 0.07 & 0.39 & 0.67 & *\\
\hline
TRIAL & 0.83 & 0.19 & 0.45 & 1.20 & *\\
\hline
SEXMale:OFF\_RACEBLACK & 0.50 & 0.14 & 0.22 & 0.78 & *\\
\hline
SEXMale:OFF\_RACELATINO & 0.69 & 0.71 & -0.71 & 2.09 & \\
\hline
SEXMale:OFF\_RACEOTHER & -0.08 & 0.69 & -1.43 & 1.28 & \\
\hline
OGS:PRVREC1/2/3 & 0.10 & 0.07 & -0.03 & 0.24 & \\
\hline
OGS:PRVREC4/5 & 0.39 & 0.09 & 0.21 & 0.57 & *\\
\hline
OGS:PRVRECREVOC/RFEL & 0.65 & 0.25 & 0.16 & 1.15 & *\\
\hline
\end{tabular}
\end{table}

From the logistic regression fit with MI and \(M = 10\) imputations done
using predictive mean matching, we found that a Black defendant is 1.25
(95\% CI of (1.234, 1.265)times more likely to be sentenced to
incarceration than an otherwise similar White defendant.

We re-analyze the data using a generalized linear mixed model and again
taking the binary incarceration decision as the outcome and random
effects for the Year and County with random slopes for the most serious
Crime type by County.

\hypertarget{tbl-glmm-sum}{}
\begin{table}
\caption{\label{tbl-glmm-sum}Summary statistics for the fixed effects from the logistic regression
mixed model. }\tabularnewline

\centering
\begin{tabular}{l|r|r|r|r|l}
\hline
Term & Estimate & SE & LB 95\% CI & UB 95\% CI & Sig.\\
\hline
(Intercept) & -1.50 & 0.12 & -1.74 & -1.25 & *\\
\hline
DOSAGE & -0.15 & 0.15 & -0.45 & 0.14 & \\
\hline
DOSAGEQ & 0.02 & 0.15 & -0.28 & 0.31 & \\
\hline
SEXMale & 0.27 & 0.07 & 0.13 & 0.40 & *\\
\hline
OFF\_RACEBLACK & -0.12 & 0.13 & -0.38 & 0.14 & \\
\hline
OFF\_RACELATINO & -0.01 & 0.68 & -1.34 & 1.32 & \\
\hline
OFF\_RACEOTHER & -0.16 & 0.62 & -1.38 & 1.06 & \\
\hline
OGS & -0.67 & 0.11 & -0.88 & -0.45 & *\\
\hline
OGSQ & 1.38 & 0.13 & 1.12 & 1.63 & *\\
\hline
PRVREC1/2/3 & 0.60 & 0.06 & 0.48 & 0.72 & *\\
\hline
PRVREC4/5 & 1.21 & 0.10 & 1.02 & 1.41 & *\\
\hline
PRVRECREVOC/RFEL & 1.52 & 0.20 & 1.13 & 1.91 & *\\
\hline
RECMIN & 0.97 & 0.10 & 0.78 & 1.16 & *\\
\hline
CRIMEDUI & 1.74 & 0.08 & 1.59 & 1.90 & *\\
\hline
CRIMEOther & 0.30 & 0.09 & 0.12 & 0.48 & *\\
\hline
CRIMEPersons & 0.84 & 0.08 & 0.67 & 1.00 & *\\
\hline
CRIMEProperty & 0.53 & 0.07 & 0.39 & 0.67 & *\\
\hline
TRIAL & 0.82 & 0.19 & 0.45 & 1.20 & *\\
\hline
SEXMale:OFF\_RACEBLACK & 0.50 & 0.14 & 0.21 & 0.78 & *\\
\hline
SEXMale:OFF\_RACELATINO & 0.69 & 0.71 & -0.70 & 2.08 & \\
\hline
SEXMale:OFF\_RACEOTHER & -0.09 & 0.69 & -1.46 & 1.27 & \\
\hline
OGS:PRVREC1/2/3 & 0.10 & 0.07 & -0.03 & 0.24 & \\
\hline
OGS:PRVREC4/5 & 0.39 & 0.09 & 0.21 & 0.57 & *\\
\hline
OGS:PRVRECREVOC/RFEL & 0.65 & 0.25 & 0.16 & 1.15 & *\\
\hline
\end{tabular}
\end{table}

\hypertarget{hurdle-models}{%
\subsection{Hurdle Models}\label{hurdle-models}}

A hurdle model models data with a high number of zeros (compared to
standard distributions). The model places a probability point mass
\(P(Y = 0) = \theta\) at \(Y = 0\) and uses a truncated (at zero)
probability distribution for the non-zero sample space
\(P(Y \neq 0) = p_{y \neq 0}(y)\). This differs from a zero-inflated
model which is a mixture of two distributions (includes the non-zero
distribution's zero probability) as the hurdle model truncates the
non-zero distribution.

I'll create a GLM with brms and the log normal hurdle distribution.

The model is composed of two components: the hurdle for the zeros and
the GLM for the non-zero part. Let \(\pi_i\) be the probability that the
\(i\)th observation is zero and \(P(Y_i \neq 0) = f_{y\neq 0}(y-i)\)
where \(f_{y\neq 0}\) is a truncated probability mass/density function.

\hypertarget{lognormal-hurdle-glm-with-intercept-only-hurdle}{%
\subsubsection{Lognormal Hurdle GLM with Intercept-only
Hurdle}\label{lognormal-hurdle-glm-with-intercept-only-hurdle}}

Under this first model, we will model the probability of \(Y_i = 0\) as
constant across the observations using an intercept-only model; the
default for \texttt{brms} (Bürkner 2017).

\begin{equation}\protect\hypertarget{eq-hurdle-intercept}{}{
\mathrm{logit}^{-1}(P(Y_i = 0)) = \pi_0
}\label{eq-hurdle-intercept}\end{equation}

\begin{equation}\protect\hypertarget{eq-hurdle-glm}{}{
\log(Y_i) = \mathbf{x}_i \boldsymbol{\beta} + \mathbf{z}_i \mathbf{u} + \epsilon_i
}\label{eq-hurdle-glm}\end{equation} where \(\mathbf{u}\) is MVN with
\(E(\mathbf{U}) = 0\) and covariance matrix \(Cov(\mathbf{U}) = G\),
\(\epsilon_i \overset{iid}{\sim} N(0, \sigma^2)\) and \(\mathbf{u}\) and
\(\boldsymbol{\epsilon}\) are mutually independent. \(\mathbf{x}_i\) and
\(\mathbf{z}_i\) are rows from two known design matrices for the
population-level and group-level effects respectively.

The model is specified as usual for a GLMM:

\begin{verbatim}
brm(JP_MIN_MON ~ DOSAGE + DOSAGEQ + SEX*OFF_RACE + OGS + OGSQ +
                     PRVREC + RECMIN + CRIME + TRIAL +
                     (1 | YEAR) + (1 + CRIME |COUNTY),
               data = data,
               family = hurdle_lognormal(link = "identity",
                           link_sigma = "log",
                           link_hu = "logit"))
\end{verbatim}

Here we include group-level effects for the year, the county, and the
crime-type in the county in case the judicial system sentences the
different crime-types differently relative to other counties. Each of
the population-level regression coefficients are given a normal prior
\(\beta_j \sim N(0, 25).\) The group-level effects for the intercepts
and effects for crime-type are given noncentral t-distributions
\(\gamma_k \sim t_{3; 0, 2.5}\) while the correlations between the
county-level crime-type effects and county-level intercepts are given
\(\rho_{i.j} \sim lkj(1)\) priors. The hurdle parameter gets a
\(U(0,1)\) prior. The error term's variance also gets a noncentral t
prior \(\sigma \sim t_{3; 0, 2.5}.\)

The quantity of interest is the regression coefficient for race.

\hypertarget{tbl-brms-hurdle-fe-int-only}{}
\begin{table}
\caption{\label{tbl-brms-hurdle-fe-int-only}Fixed/Population-level effects for the non-zero part of the lognormal
hurdle model with intercept only for the zero-part. }\tabularnewline

\centering
\begin{tabular}{l|r|r|r|r}
\hline
Term & Estimate & SE & LB 95\% CI & UB 95\% CI\\
\hline
Intercept & 0.37 & 0.07 & 0.23 & 0.51\\
\hline
DOSAGE & 0.12 & 0.10 & -0.07 & 0.31\\
\hline
DOSAGEQ & -0.09 & 0.09 & -0.28 & 0.09\\
\hline
SEXMale & 0.16 & 0.05 & 0.07 & 0.25\\
\hline
OFF\_RACEBLACK & -0.17 & 0.09 & -0.36 & 0.01\\
\hline
OFF\_RACELATINO & -0.94 & 0.44 & -1.80 & -0.07\\
\hline
OFF\_RACEOTHER & 0.23 & 0.40 & -0.56 & 1.00\\
\hline
OGS & 1.80 & 0.05 & 1.69 & 1.91\\
\hline
OGSQ & -0.64 & 0.04 & -0.72 & -0.56\\
\hline
PRVREC1D2D3 & 0.40 & 0.04 & 0.33 & 0.47\\
\hline
PRVREC4D5 & 0.90 & 0.05 & 0.79 & 1.01\\
\hline
PRVRECREVOCDRFEL & 1.40 & 0.09 & 1.21 & 1.58\\
\hline
RECMIN & 0.19 & 0.06 & 0.08 & 0.30\\
\hline
CRIMEDUI & -0.94 & 0.05 & -1.04 & -0.83\\
\hline
CRIMEOther & -0.04 & 0.06 & -0.15 & 0.08\\
\hline
CRIMEPersons & 0.05 & 0.05 & -0.05 & 0.15\\
\hline
CRIMEProperty & 0.07 & 0.05 & -0.02 & 0.16\\
\hline
TRIAL & 0.50 & 0.08 & 0.34 & 0.66\\
\hline
SEXMale:OFF\_RACEBLACK & 0.13 & 0.10 & -0.06 & 0.33\\
\hline
SEXMale:OFF\_RACELATINO & 0.98 & 0.46 & 0.09 & 1.87\\
\hline
SEXMale:OFF\_RACEOTHER & -0.06 & 0.44 & -0.92 & 0.79\\
\hline
OGS:PRVREC1D2D3 & -0.08 & 0.03 & -0.14 & -0.02\\
\hline
OGS:PRVREC4D5 & -0.14 & 0.03 & -0.21 & -0.08\\
\hline
OGS:PRVRECREVOCDRFEL & -0.20 & 0.06 & -0.33 & -0.08\\
\hline
\end{tabular}
\end{table}

The standard deviation parameter \(\sigma\) of the lognormal
distribution has a posterior mean of 1 (95\% CI: {[}0.98, 1.02 {]}). The
hurdle parameter has a posterior mean of 0.52 (95\% CI: {[}0.51, 0.53
{]}) which matches the probability of not being incarcerated in the
overall population.

\hypertarget{tbl-brms1-group-eff}{}
\begin{table}
\caption{\label{tbl-brms1-group-eff}Random/group-level effect standard deviation estimates for the
intercept-only hurdle lognormal model. }\tabularnewline

\centering
\begin{tabular}{l|l|r|r|r}
\hline
Term & Estimate & SE & LB 95\% CI & UB 95\% CI\\
\hline
County-level & sd(Intercept) & 0.22 & 0.16 & 0.28\\
\hline
Year-level & sd(Intercept & 0.05 & 0.00 & 0.16\\
\hline
\end{tabular}
\end{table}

From the hurdle model, we find that Black defendants do not receive
different sentence lengths compared to White defendants (95\% CI of
(-0.11, 0.03)). In terms of 95\% CIs we also find that there doesn't
seem to be a difference in sentence lengths between male and female
defendants either between or within each racial group.

\begin{figure}

\begin{minipage}[t]{0.33\linewidth}

{\centering 

\raisebox{-\height}{

\includegraphics{pattern-mixture-modeling_files/figure-pdf/fig-cond-eff-brm-nonzero-1.pdf}

}

}

\subcaption{\label{fig-cond-eff-brm-nonzero-1}Age}
\end{minipage}%
%
\begin{minipage}[t]{0.33\linewidth}

{\centering 

\raisebox{-\height}{

\includegraphics{pattern-mixture-modeling_files/figure-pdf/fig-cond-eff-brm-nonzero-2.pdf}

}

}

\subcaption{\label{fig-cond-eff-brm-nonzero-2}Age Squared}
\end{minipage}%
%
\begin{minipage}[t]{0.33\linewidth}

{\centering 

\raisebox{-\height}{

\includegraphics{pattern-mixture-modeling_files/figure-pdf/fig-cond-eff-brm-nonzero-3.pdf}

}

}

\subcaption{\label{fig-cond-eff-brm-nonzero-3}Sex}
\end{minipage}%
\newline
\begin{minipage}[t]{0.33\linewidth}

{\centering 

\raisebox{-\height}{

\includegraphics{pattern-mixture-modeling_files/figure-pdf/fig-cond-eff-brm-nonzero-4.pdf}

}

}

\subcaption{\label{fig-cond-eff-brm-nonzero-4}Offender Race}
\end{minipage}%
%
\begin{minipage}[t]{0.33\linewidth}

{\centering 

\raisebox{-\height}{

\includegraphics{pattern-mixture-modeling_files/figure-pdf/fig-cond-eff-brm-nonzero-5.pdf}

}

}

\subcaption{\label{fig-cond-eff-brm-nonzero-5}OGS}
\end{minipage}%
%
\begin{minipage}[t]{0.33\linewidth}

{\centering 

\raisebox{-\height}{

\includegraphics{pattern-mixture-modeling_files/figure-pdf/fig-cond-eff-brm-nonzero-6.pdf}

}

}

\subcaption{\label{fig-cond-eff-brm-nonzero-6}OGS Squared}
\end{minipage}%
\newline
\begin{minipage}[t]{0.33\linewidth}

{\centering 

\raisebox{-\height}{

\includegraphics{pattern-mixture-modeling_files/figure-pdf/fig-cond-eff-brm-nonzero-7.pdf}

}

}

\subcaption{\label{fig-cond-eff-brm-nonzero-7}Prev. Rec.}
\end{minipage}%
%
\begin{minipage}[t]{0.33\linewidth}

{\centering 

\raisebox{-\height}{

\includegraphics{pattern-mixture-modeling_files/figure-pdf/fig-cond-eff-brm-nonzero-8.pdf}

}

}

\subcaption{\label{fig-cond-eff-brm-nonzero-8}Rec. Min.}
\end{minipage}%
%
\begin{minipage}[t]{0.33\linewidth}

{\centering 

\raisebox{-\height}{

\includegraphics{pattern-mixture-modeling_files/figure-pdf/fig-cond-eff-brm-nonzero-9.pdf}

}

}

\subcaption{\label{fig-cond-eff-brm-nonzero-9}Crime Type}
\end{minipage}%
\newline
\begin{minipage}[t]{0.33\linewidth}

{\centering 

\raisebox{-\height}{

\includegraphics{pattern-mixture-modeling_files/figure-pdf/fig-cond-eff-brm-nonzero-10.pdf}

}

}

\subcaption{\label{fig-cond-eff-brm-nonzero-10}Trial}
\end{minipage}%
%
\begin{minipage}[t]{0.33\linewidth}

{\centering 

\raisebox{-\height}{

\includegraphics{pattern-mixture-modeling_files/figure-pdf/fig-cond-eff-brm-nonzero-11.pdf}

}

}

\subcaption{\label{fig-cond-eff-brm-nonzero-11}Race by Sex}
\end{minipage}%
%
\begin{minipage}[t]{0.33\linewidth}

{\centering 

\raisebox{-\height}{

\includegraphics{pattern-mixture-modeling_files/figure-pdf/fig-cond-eff-brm-nonzero-12.pdf}

}

}

\subcaption{\label{fig-cond-eff-brm-nonzero-12}OGS by Prev. Rec.}
\end{minipage}%

\caption{\label{fig-cond-eff-brm-nonzero}Posterior estimates of the
conditional effects for the lognormal glm for the non-zero-part of the
intercept-only hurdle lognormal model.}

\end{figure}

\begin{figure}

\begin{minipage}[t]{0.33\linewidth}

{\centering 

\raisebox{-\height}{

\includegraphics{pattern-mixture-modeling_files/figure-pdf/fig-cond-eff-brm1-zero-1.pdf}

}

}

\subcaption{\label{fig-cond-eff-brm1-zero-1}Age}
\end{minipage}%
%
\begin{minipage}[t]{0.33\linewidth}

{\centering 

\raisebox{-\height}{

\includegraphics{pattern-mixture-modeling_files/figure-pdf/fig-cond-eff-brm1-zero-2.pdf}

}

}

\subcaption{\label{fig-cond-eff-brm1-zero-2}Age Squared}
\end{minipage}%
%
\begin{minipage}[t]{0.33\linewidth}

{\centering 

\raisebox{-\height}{

\includegraphics{pattern-mixture-modeling_files/figure-pdf/fig-cond-eff-brm1-zero-3.pdf}

}

}

\subcaption{\label{fig-cond-eff-brm1-zero-3}Sex}
\end{minipage}%
\newline
\begin{minipage}[t]{0.33\linewidth}

{\centering 

\raisebox{-\height}{

\includegraphics{pattern-mixture-modeling_files/figure-pdf/fig-cond-eff-brm1-zero-4.pdf}

}

}

\subcaption{\label{fig-cond-eff-brm1-zero-4}Offender Race}
\end{minipage}%
%
\begin{minipage}[t]{0.33\linewidth}

{\centering 

\raisebox{-\height}{

\includegraphics{pattern-mixture-modeling_files/figure-pdf/fig-cond-eff-brm1-zero-5.pdf}

}

}

\subcaption{\label{fig-cond-eff-brm1-zero-5}OGS}
\end{minipage}%
%
\begin{minipage}[t]{0.33\linewidth}

{\centering 

\raisebox{-\height}{

\includegraphics{pattern-mixture-modeling_files/figure-pdf/fig-cond-eff-brm1-zero-6.pdf}

}

}

\subcaption{\label{fig-cond-eff-brm1-zero-6}OGS Squared}
\end{minipage}%
\newline
\begin{minipage}[t]{0.33\linewidth}

{\centering 

\raisebox{-\height}{

\includegraphics{pattern-mixture-modeling_files/figure-pdf/fig-cond-eff-brm1-zero-7.pdf}

}

}

\subcaption{\label{fig-cond-eff-brm1-zero-7}Prev. Rec.}
\end{minipage}%
%
\begin{minipage}[t]{0.33\linewidth}

{\centering 

\raisebox{-\height}{

\includegraphics{pattern-mixture-modeling_files/figure-pdf/fig-cond-eff-brm1-zero-8.pdf}

}

}

\subcaption{\label{fig-cond-eff-brm1-zero-8}Rec. Min.}
\end{minipage}%
%
\begin{minipage}[t]{0.33\linewidth}

{\centering 

\raisebox{-\height}{

\includegraphics{pattern-mixture-modeling_files/figure-pdf/fig-cond-eff-brm1-zero-9.pdf}

}

}

\subcaption{\label{fig-cond-eff-brm1-zero-9}Crime Type}
\end{minipage}%
\newline
\begin{minipage}[t]{0.33\linewidth}

{\centering 

\raisebox{-\height}{

\includegraphics{pattern-mixture-modeling_files/figure-pdf/fig-cond-eff-brm1-zero-10.pdf}

}

}

\subcaption{\label{fig-cond-eff-brm1-zero-10}Trial}
\end{minipage}%
%
\begin{minipage}[t]{0.33\linewidth}

{\centering 

\raisebox{-\height}{

\includegraphics{pattern-mixture-modeling_files/figure-pdf/fig-cond-eff-brm1-zero-11.pdf}

}

}

\subcaption{\label{fig-cond-eff-brm1-zero-11}Race by Sex}
\end{minipage}%
%
\begin{minipage}[t]{0.33\linewidth}

{\centering 

\raisebox{-\height}{

\includegraphics{pattern-mixture-modeling_files/figure-pdf/fig-cond-eff-brm1-zero-12.pdf}

}

}

\subcaption{\label{fig-cond-eff-brm1-zero-12}OGS by Prev. Rec.}
\end{minipage}%

\caption{\label{fig-cond-eff-brm1-zero}Posterior estimates of the
conditional effects for the logistic regression for the zero-part of the
intercept-only hurdle lognormal model.}

\end{figure}

\begin{figure}

\begin{minipage}[t]{0.50\linewidth}

{\centering 

\raisebox{-\height}{

\includegraphics{pattern-mixture-modeling_files/figure-pdf/fig-ppc-brm1-1.pdf}

}

}

\subcaption{\label{fig-ppc-brm1-1}Density for sentence length}
\end{minipage}%
%
\begin{minipage}[t]{0.50\linewidth}

{\centering 

\raisebox{-\height}{

\includegraphics{pattern-mixture-modeling_files/figure-pdf/fig-ppc-brm1-2.pdf}

}

}

\subcaption{\label{fig-ppc-brm1-2}Density for log of sentence length
plus one month}
\end{minipage}%

\caption{\label{fig-ppc-brm1}Posterior Predictive Checks for the
intercept-only hurdle model.}

\end{figure}

\hypertarget{lognormal-glm-hurdle-model-with-predictors-on-hurdle-parameter}{%
\subsubsection{Lognormal GLM Hurdle Model with Predictors on Hurdle
Parameter}\label{lognormal-glm-hurdle-model-with-predictors-on-hurdle-parameter}}

Next, we modify the model from the previous section to include
predictors in the logistic regression part of the hurdle model
Equation~\ref{eq-hurdle-intercept}. The non-zero portion
Equation~\ref{eq-hurdle-glm} of the hurdle model remains the same.

\[
\mathrm{logit}^{-1}(P(Y_i=0)) = \mathbf{x}_i \boldsymbol{\alpha}.
\]

The coefficients \(\boldsymbol{\alpha}\) are given normal priors
\(\alpha_k \overset{iid}{\sim} N(0, 10).\) The priors for the other
parameters remain the same.

The new model is specified with the \texttt{bf()} function.

\begin{verbatim}
bf(JP_MIN_MON ~ DOSAGE + DOSAGEQ + SEX*OFF_RACE + OGS + OGSQ +
                 PRVREC + RECMIN + CRIME + TRIAL +
                 (1 | YEAR) + (1 + CRIME |COUNTY),
    hu ~ 1 + SEX * OFF_RACE + DOSAGE + DOSAGEQ + OGS + OGSQ +
            PRVREC + CRIME + TRIAL + (1 | COUNTY))
\end{verbatim}

Next we fit the model.

\hypertarget{tbl-brms-hurdle-model-summary-2}{}
\begin{table}
\caption{\label{tbl-brms-hurdle-model-summary-2}Fixed/population-level effects for the non-zero part of the full
lognormal hurdle glm. }\tabularnewline

\centering
\begin{tabular}{l|r|r|r|r}
\hline
Term & Estimate & SE & LB 95\% CI & UB 95\% CI\\
\hline
Intercept & 3.79 & 0.07 & 3.64 & 3.93\\
\hline
DOSAGE & 0.12 & 0.09 & -0.07 & 0.30\\
\hline
DOSAGEQ & -0.09 & 0.09 & -0.27 & 0.09\\
\hline
SEXMale & 0.16 & 0.05 & 0.07 & 0.26\\
\hline
OFF\_RACEBLACK & -0.17 & 0.09 & -0.36 & 0.02\\
\hline
OFF\_RACELATINO & -0.96 & 0.44 & -1.82 & -0.10\\
\hline
OFF\_RACEOTHER & 0.23 & 0.41 & -0.58 & 1.02\\
\hline
OGS & 1.80 & 0.05 & 1.70 & 1.91\\
\hline
OGSQ & -0.64 & 0.04 & -0.72 & -0.56\\
\hline
PRVREC1D2D3 & 0.40 & 0.04 & 0.33 & 0.47\\
\hline
PRVREC4D5 & 0.90 & 0.05 & 0.80 & 1.01\\
\hline
PRVRECREVOCDRFEL & 1.40 & 0.09 & 1.21 & 1.57\\
\hline
RECMIN & 0.19 & 0.06 & 0.08 & 0.30\\
\hline
CRIMEDUI & -0.94 & 0.05 & -1.04 & -0.83\\
\hline
CRIMEOther & -0.04 & 0.06 & -0.15 & 0.08\\
\hline
CRIMEPersons & 0.05 & 0.05 & -0.05 & 0.15\\
\hline
CRIMEProperty & 0.07 & 0.05 & -0.02 & 0.17\\
\hline
TRIAL & 0.50 & 0.08 & 0.34 & 0.66\\
\hline
SEXMale:OFF\_RACEBLACK & 0.13 & 0.10 & -0.07 & 0.33\\
\hline
SEXMale:OFF\_RACELATINO & 0.99 & 0.45 & 0.12 & 1.88\\
\hline
SEXMale:OFF\_RACEOTHER & -0.06 & 0.44 & -0.92 & 0.83\\
\hline
OGS:PRVREC1D2D3 & -0.08 & 0.03 & -0.14 & -0.02\\
\hline
OGS:PRVREC4D5 & -0.14 & 0.03 & -0.21 & -0.08\\
\hline
OGS:PRVRECREVOCDRFEL & -0.20 & 0.06 & -0.33 & -0.08\\
\hline
\end{tabular}
\end{table}

\hypertarget{tbl-brms-hurdle-model-summary-2-zero}{}
\begin{table}
\caption{\label{tbl-brms-hurdle-model-summary-2-zero}Fixed/population-level effects for the zero part of the full lognormal
hurdle glm. }\tabularnewline

\centering
\begin{tabular}{l|r|r|r|r}
\hline
Term & Estimate & SE & LB 95\% CI & UB 95\% CI\\
\hline
hu\_Intercept & 1.44 & 0.13 & 1.20 & 1.70\\
\hline
hu\_SEXMale & -0.29 & 0.07 & -0.43 & -0.16\\
\hline
hu\_OFF\_RACEBLACK & 0.10 & 0.13 & -0.16 & 0.36\\
\hline
hu\_OFF\_RACELATINO & 0.08 & 0.68 & -1.22 & 1.44\\
\hline
hu\_OFF\_RACEOTHER & 0.19 & 0.63 & -1.01 & 1.47\\
\hline
hu\_RECMIN & -0.99 & 0.09 & -1.17 & -0.80\\
\hline
hu\_OGS & 0.59 & 0.11 & 0.38 & 0.79\\
\hline
hu\_OGSQ & -1.38 & 0.13 & -1.64 & -1.13\\
\hline
hu\_PRVREC1D2D3 & -0.54 & 0.06 & -0.65 & -0.43\\
\hline
hu\_PRVREC4D5 & -1.06 & 0.09 & -1.24 & -0.88\\
\hline
hu\_PRVRECREVOCDRFEL & -1.28 & 0.19 & -1.65 & -0.92\\
\hline
hu\_CRIMEDUI & -1.67 & 0.08 & -1.82 & -1.52\\
\hline
hu\_CRIMEOther & -0.27 & 0.09 & -0.45 & -0.09\\
\hline
hu\_CRIMEPersons & -0.81 & 0.08 & -0.97 & -0.64\\
\hline
hu\_CRIMEProperty & -0.51 & 0.07 & -0.65 & -0.37\\
\hline
hu\_TRIAL & -0.80 & 0.19 & -1.18 & -0.43\\
\hline
hu\_SEXMale:OFF\_RACEBLACK & -0.50 & 0.14 & -0.78 & -0.22\\
\hline
hu\_SEXMale:OFF\_RACELATINO & -0.79 & 0.71 & -2.21 & 0.55\\
\hline
hu\_SEXMale:OFF\_RACEOTHER & 0.03 & 0.70 & -1.38 & 1.37\\
\hline
\end{tabular}
\end{table}

The standard deviation parameter \(\sigma\) of the lognormal
distribution has a posterior mean of 1 (95\% CI: {[}0.98, 1.02{]}).

The county-level random/group-level effects and year-level
random/group-level effects are reported in Table~\ref{tbl-brms2-re}.

\hypertarget{tbl-brms2-re}{}
\begin{table}
\caption{\label{tbl-brms2-re}Random/group-level effect standard deviation estimates for the full
hurdle lognormal model. }\tabularnewline

\centering
\begin{tabular}{l|l|r|r|r}
\hline
Term & Estimate & SE & LB 95\% CI & UB 95\% CI\\
\hline
County-level & sd(Intercept) & 0.22 & 0.16 & 0.28\\
\hline
County-level & sd(hu\_Intercept) & 0.72 & 0.58 & 0.89\\
\hline
Year-level & sd(Intercept & 0.05 & 0.00 & 0.15\\
\hline
\end{tabular}
\end{table}

\begin{figure}

\begin{minipage}[t]{0.33\linewidth}

{\centering 

\raisebox{-\height}{

\includegraphics{pattern-mixture-modeling_files/figure-pdf/fig-cond-eff-brm2-nonzero-1.pdf}

}

}

\subcaption{\label{fig-cond-eff-brm2-nonzero-1}Age}
\end{minipage}%
%
\begin{minipage}[t]{0.33\linewidth}

{\centering 

\raisebox{-\height}{

\includegraphics{pattern-mixture-modeling_files/figure-pdf/fig-cond-eff-brm2-nonzero-2.pdf}

}

}

\subcaption{\label{fig-cond-eff-brm2-nonzero-2}Age Squared}
\end{minipage}%
%
\begin{minipage}[t]{0.33\linewidth}

{\centering 

\raisebox{-\height}{

\includegraphics{pattern-mixture-modeling_files/figure-pdf/fig-cond-eff-brm2-nonzero-3.pdf}

}

}

\subcaption{\label{fig-cond-eff-brm2-nonzero-3}Sex}
\end{minipage}%
\newline
\begin{minipage}[t]{0.33\linewidth}

{\centering 

\raisebox{-\height}{

\includegraphics{pattern-mixture-modeling_files/figure-pdf/fig-cond-eff-brm2-nonzero-4.pdf}

}

}

\subcaption{\label{fig-cond-eff-brm2-nonzero-4}Offender Race}
\end{minipage}%
%
\begin{minipage}[t]{0.33\linewidth}

{\centering 

\raisebox{-\height}{

\includegraphics{pattern-mixture-modeling_files/figure-pdf/fig-cond-eff-brm2-nonzero-5.pdf}

}

}

\subcaption{\label{fig-cond-eff-brm2-nonzero-5}OGS}
\end{minipage}%
%
\begin{minipage}[t]{0.33\linewidth}

{\centering 

\raisebox{-\height}{

\includegraphics{pattern-mixture-modeling_files/figure-pdf/fig-cond-eff-brm2-nonzero-6.pdf}

}

}

\subcaption{\label{fig-cond-eff-brm2-nonzero-6}OGS Squared}
\end{minipage}%
\newline
\begin{minipage}[t]{0.33\linewidth}

{\centering 

\raisebox{-\height}{

\includegraphics{pattern-mixture-modeling_files/figure-pdf/fig-cond-eff-brm2-nonzero-7.pdf}

}

}

\subcaption{\label{fig-cond-eff-brm2-nonzero-7}Prev. Rec.}
\end{minipage}%
%
\begin{minipage}[t]{0.33\linewidth}

{\centering 

\raisebox{-\height}{

\includegraphics{pattern-mixture-modeling_files/figure-pdf/fig-cond-eff-brm2-nonzero-8.pdf}

}

}

\subcaption{\label{fig-cond-eff-brm2-nonzero-8}Rec. Min.}
\end{minipage}%
%
\begin{minipage}[t]{0.33\linewidth}

{\centering 

\raisebox{-\height}{

\includegraphics{pattern-mixture-modeling_files/figure-pdf/fig-cond-eff-brm2-nonzero-9.pdf}

}

}

\subcaption{\label{fig-cond-eff-brm2-nonzero-9}Crime Type}
\end{minipage}%
\newline
\begin{minipage}[t]{0.33\linewidth}

{\centering 

\raisebox{-\height}{

\includegraphics{pattern-mixture-modeling_files/figure-pdf/fig-cond-eff-brm2-nonzero-10.pdf}

}

}

\subcaption{\label{fig-cond-eff-brm2-nonzero-10}Trial}
\end{minipage}%
%
\begin{minipage}[t]{0.33\linewidth}

{\centering 

\raisebox{-\height}{

\includegraphics{pattern-mixture-modeling_files/figure-pdf/fig-cond-eff-brm2-nonzero-11.pdf}

}

}

\subcaption{\label{fig-cond-eff-brm2-nonzero-11}Race by Sex}
\end{minipage}%
%
\begin{minipage}[t]{0.33\linewidth}

{\centering 

\raisebox{-\height}{

\includegraphics{pattern-mixture-modeling_files/figure-pdf/fig-cond-eff-brm2-nonzero-12.pdf}

}

}

\subcaption{\label{fig-cond-eff-brm2-nonzero-12}OGS by Prev. Rec.}
\end{minipage}%

\caption{\label{fig-cond-eff-brm2-nonzero}Posterior estimates of the
conditional effects for the lognormal glm for the non-zero-part of the
full hurdle lognormal model.}

\end{figure}

\begin{figure}

\begin{minipage}[t]{0.33\linewidth}

{\centering 

\raisebox{-\height}{

\includegraphics{pattern-mixture-modeling_files/figure-pdf/fig-cond-eff-brm2-1.pdf}

}

}

\subcaption{\label{fig-cond-eff-brm2-1}Age}
\end{minipage}%
%
\begin{minipage}[t]{0.33\linewidth}

{\centering 

\raisebox{-\height}{

\includegraphics{pattern-mixture-modeling_files/figure-pdf/fig-cond-eff-brm2-2.pdf}

}

}

\subcaption{\label{fig-cond-eff-brm2-2}Age Squared}
\end{minipage}%
%
\begin{minipage}[t]{0.33\linewidth}

{\centering 

\raisebox{-\height}{

\includegraphics{pattern-mixture-modeling_files/figure-pdf/fig-cond-eff-brm2-3.pdf}

}

}

\subcaption{\label{fig-cond-eff-brm2-3}Sex}
\end{minipage}%
\newline
\begin{minipage}[t]{0.33\linewidth}

{\centering 

\raisebox{-\height}{

\includegraphics{pattern-mixture-modeling_files/figure-pdf/fig-cond-eff-brm2-4.pdf}

}

}

\subcaption{\label{fig-cond-eff-brm2-4}Offender Race}
\end{minipage}%
%
\begin{minipage}[t]{0.33\linewidth}

{\centering 

\raisebox{-\height}{

\includegraphics{pattern-mixture-modeling_files/figure-pdf/fig-cond-eff-brm2-5.pdf}

}

}

\subcaption{\label{fig-cond-eff-brm2-5}OGS}
\end{minipage}%
%
\begin{minipage}[t]{0.33\linewidth}

{\centering 

\raisebox{-\height}{

\includegraphics{pattern-mixture-modeling_files/figure-pdf/fig-cond-eff-brm2-6.pdf}

}

}

\subcaption{\label{fig-cond-eff-brm2-6}OGS Squared}
\end{minipage}%
\newline
\begin{minipage}[t]{0.33\linewidth}

{\centering 

\raisebox{-\height}{

\includegraphics{pattern-mixture-modeling_files/figure-pdf/fig-cond-eff-brm2-7.pdf}

}

}

\subcaption{\label{fig-cond-eff-brm2-7}Prev. Rec.}
\end{minipage}%
%
\begin{minipage}[t]{0.33\linewidth}

{\centering 

\raisebox{-\height}{

\includegraphics{pattern-mixture-modeling_files/figure-pdf/fig-cond-eff-brm2-8.pdf}

}

}

\subcaption{\label{fig-cond-eff-brm2-8}Rec. Min.}
\end{minipage}%
%
\begin{minipage}[t]{0.33\linewidth}

{\centering 

\raisebox{-\height}{

\includegraphics{pattern-mixture-modeling_files/figure-pdf/fig-cond-eff-brm2-9.pdf}

}

}

\subcaption{\label{fig-cond-eff-brm2-9}Crime Type}
\end{minipage}%
\newline
\begin{minipage}[t]{0.33\linewidth}

{\centering 

\raisebox{-\height}{

\includegraphics{pattern-mixture-modeling_files/figure-pdf/fig-cond-eff-brm2-10.pdf}

}

}

\subcaption{\label{fig-cond-eff-brm2-10}Trial}
\end{minipage}%
%
\begin{minipage}[t]{0.33\linewidth}

{\centering 

\raisebox{-\height}{

\includegraphics{pattern-mixture-modeling_files/figure-pdf/fig-cond-eff-brm2-11.pdf}

}

}

\subcaption{\label{fig-cond-eff-brm2-11}Race by Sex}
\end{minipage}%
%
\begin{minipage}[t]{0.33\linewidth}

{\centering 

\raisebox{-\height}{

\includegraphics{pattern-mixture-modeling_files/figure-pdf/fig-cond-eff-brm2-12.pdf}

}

}

\subcaption{\label{fig-cond-eff-brm2-12}OGS by Prev. Rec.}
\end{minipage}%

\caption{\label{fig-cond-eff-brm2}Posterior estimates of the conditional
effects for the logistic regression on the zero-part of the full hurdle
lognormal model.}

\end{figure}

\begin{figure}

\begin{minipage}[t]{0.33\linewidth}

{\centering 

\raisebox{-\height}{

\includegraphics{pattern-mixture-modeling_files/figure-pdf/fig-ppc-brm2-1.pdf}

}

}

\subcaption{\label{fig-ppc-brm2-1}PPC Density Sentence Length (months)}
\end{minipage}%
%
\begin{minipage}[t]{0.33\linewidth}

{\centering 

\raisebox{-\height}{

\includegraphics{pattern-mixture-modeling_files/figure-pdf/fig-ppc-brm2-2.pdf}

}

}

\subcaption{\label{fig-ppc-brm2-2}PPC Density Log of Sentence Length
plus 1}
\end{minipage}%
%
\begin{minipage}[t]{0.33\linewidth}

{\centering 

\raisebox{-\height}{

\includegraphics{pattern-mixture-modeling_files/figure-pdf/fig-ppc-brm2-3.pdf}

}

}

\subcaption{\label{fig-ppc-brm2-3}PPC Scatter of average error by age}
\end{minipage}%

\caption{\label{fig-ppc-brm2}Posterior Predictive Checks for the full
lognormal hurdle model.}

\end{figure}

\hypertarget{section}{%
\subsubsection{}\label{section}}

\hypertarget{glmmadaptive-poisson-hurdle-model}{%
\subsubsection{GLMMadaptive Poisson Hurdle
Model}\label{glmmadaptive-poisson-hurdle-model}}

\hypertarget{sensitivity-analysis}{%
\section{Sensitivity Analysis}\label{sensitivity-analysis}}

Evaluating the impacts of various nonignorable missingness mechanisms
can be accomplished with pattern-mixture models. The values of the
incomplete numeric data can be scaled or shifted. The same cannot be
done with categorical imputations, instead the proportion of each
category can be varied within the imputations compared to the observed
distribution or the MAR imputed distribution.

While there are very few missing sentence lengths, I could also modify
the imputed sentence lengths with a scale \(c\) or shift \(\delta\).
Scaling maintains that offenders that are non-incarcerated will remain
non-incarcerated while the sentences of incarcerated defenders would
shift. A positive shift would make all offenders incarcerated while a
negative shift would impose negative sentence lengths that would need to
be corrected to use Poisson or lognormal hurdle glms.

\newpage

\hypertarget{references}{%
\section*{References}\label{references}}
\addcontentsline{toc}{section}{References}

\hypertarget{refs}{}
\begin{CSLReferences}{1}{0}
\leavevmode\vadjust pre{\hypertarget{ref-buxfcrkner2017}{}}%
Bürkner, Paul-Christian. 2017. {``Brms: An R Package for Bayesian
Multilevel Models Using Stan.''} \emph{Journal of Statistical Software}
80 (August): 1--28. \url{https://doi.org/10.18637/jss.v080.i01}.

\end{CSLReferences}



\end{document}
