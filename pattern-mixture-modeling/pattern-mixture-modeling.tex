% Options for packages loaded elsewhere
\PassOptionsToPackage{unicode}{hyperref}
\PassOptionsToPackage{hyphens}{url}
\PassOptionsToPackage{dvipsnames,svgnames,x11names}{xcolor}
%
\documentclass[
  letterpaper,
  DIV=11,
  numbers=noendperiod]{scrartcl}

\usepackage{amsmath,amssymb}
\usepackage{iftex}
\ifPDFTeX
  \usepackage[T1]{fontenc}
  \usepackage[utf8]{inputenc}
  \usepackage{textcomp} % provide euro and other symbols
\else % if luatex or xetex
  \usepackage{unicode-math}
  \defaultfontfeatures{Scale=MatchLowercase}
  \defaultfontfeatures[\rmfamily]{Ligatures=TeX,Scale=1}
\fi
\usepackage{lmodern}
\ifPDFTeX\else  
    % xetex/luatex font selection
\fi
% Use upquote if available, for straight quotes in verbatim environments
\IfFileExists{upquote.sty}{\usepackage{upquote}}{}
\IfFileExists{microtype.sty}{% use microtype if available
  \usepackage[]{microtype}
  \UseMicrotypeSet[protrusion]{basicmath} % disable protrusion for tt fonts
}{}
\makeatletter
\@ifundefined{KOMAClassName}{% if non-KOMA class
  \IfFileExists{parskip.sty}{%
    \usepackage{parskip}
  }{% else
    \setlength{\parindent}{0pt}
    \setlength{\parskip}{6pt plus 2pt minus 1pt}}
}{% if KOMA class
  \KOMAoptions{parskip=half}}
\makeatother
\usepackage{xcolor}
\setlength{\emergencystretch}{3em} % prevent overfull lines
\setcounter{secnumdepth}{2}
% Make \paragraph and \subparagraph free-standing
\ifx\paragraph\undefined\else
  \let\oldparagraph\paragraph
  \renewcommand{\paragraph}[1]{\oldparagraph{#1}\mbox{}}
\fi
\ifx\subparagraph\undefined\else
  \let\oldsubparagraph\subparagraph
  \renewcommand{\subparagraph}[1]{\oldsubparagraph{#1}\mbox{}}
\fi


\providecommand{\tightlist}{%
  \setlength{\itemsep}{0pt}\setlength{\parskip}{0pt}}\usepackage{longtable,booktabs,array}
\usepackage{calc} % for calculating minipage widths
% Correct order of tables after \paragraph or \subparagraph
\usepackage{etoolbox}
\makeatletter
\patchcmd\longtable{\par}{\if@noskipsec\mbox{}\fi\par}{}{}
\makeatother
% Allow footnotes in longtable head/foot
\IfFileExists{footnotehyper.sty}{\usepackage{footnotehyper}}{\usepackage{footnote}}
\makesavenoteenv{longtable}
\usepackage{graphicx}
\makeatletter
\def\maxwidth{\ifdim\Gin@nat@width>\linewidth\linewidth\else\Gin@nat@width\fi}
\def\maxheight{\ifdim\Gin@nat@height>\textheight\textheight\else\Gin@nat@height\fi}
\makeatother
% Scale images if necessary, so that they will not overflow the page
% margins by default, and it is still possible to overwrite the defaults
% using explicit options in \includegraphics[width, height, ...]{}
\setkeys{Gin}{width=\maxwidth,height=\maxheight,keepaspectratio}
% Set default figure placement to htbp
\makeatletter
\def\fps@figure{htbp}
\makeatother
\newlength{\cslhangindent}
\setlength{\cslhangindent}{1.5em}
\newlength{\csllabelwidth}
\setlength{\csllabelwidth}{3em}
\newlength{\cslentryspacingunit} % times entry-spacing
\setlength{\cslentryspacingunit}{\parskip}
\newenvironment{CSLReferences}[2] % #1 hanging-ident, #2 entry spacing
 {% don't indent paragraphs
  \setlength{\parindent}{0pt}
  % turn on hanging indent if param 1 is 1
  \ifodd #1
  \let\oldpar\par
  \def\par{\hangindent=\cslhangindent\oldpar}
  \fi
  % set entry spacing
  \setlength{\parskip}{#2\cslentryspacingunit}
 }%
 {}
\usepackage{calc}
\newcommand{\CSLBlock}[1]{#1\hfill\break}
\newcommand{\CSLLeftMargin}[1]{\parbox[t]{\csllabelwidth}{#1}}
\newcommand{\CSLRightInline}[1]{\parbox[t]{\linewidth - \csllabelwidth}{#1}\break}
\newcommand{\CSLIndent}[1]{\hspace{\cslhangindent}#1}

\usepackage{booktabs}
\usepackage{longtable}
\usepackage{array}
\usepackage{multirow}
\usepackage{wrapfig}
\usepackage{float}
\usepackage{colortbl}
\usepackage{pdflscape}
\usepackage{tabu}
\usepackage{threeparttable}
\usepackage{threeparttablex}
\usepackage[normalem]{ulem}
\usepackage{makecell}
\usepackage{xcolor}
\KOMAoption{captions}{tableheading}
\makeatletter
\makeatother
\makeatletter
\makeatother
\makeatletter
\@ifpackageloaded{caption}{}{\usepackage{caption}}
\AtBeginDocument{%
\ifdefined\contentsname
  \renewcommand*\contentsname{Table of contents}
\else
  \newcommand\contentsname{Table of contents}
\fi
\ifdefined\listfigurename
  \renewcommand*\listfigurename{List of Figures}
\else
  \newcommand\listfigurename{List of Figures}
\fi
\ifdefined\listtablename
  \renewcommand*\listtablename{List of Tables}
\else
  \newcommand\listtablename{List of Tables}
\fi
\ifdefined\figurename
  \renewcommand*\figurename{Figure}
\else
  \newcommand\figurename{Figure}
\fi
\ifdefined\tablename
  \renewcommand*\tablename{Table}
\else
  \newcommand\tablename{Table}
\fi
}
\@ifpackageloaded{float}{}{\usepackage{float}}
\floatstyle{ruled}
\@ifundefined{c@chapter}{\newfloat{codelisting}{h}{lop}}{\newfloat{codelisting}{h}{lop}[chapter]}
\floatname{codelisting}{Listing}
\newcommand*\listoflistings{\listof{codelisting}{List of Listings}}
\makeatother
\makeatletter
\@ifpackageloaded{caption}{}{\usepackage{caption}}
\@ifpackageloaded{subcaption}{}{\usepackage{subcaption}}
\makeatother
\makeatletter
\@ifpackageloaded{tcolorbox}{}{\usepackage[skins,breakable]{tcolorbox}}
\makeatother
\makeatletter
\@ifundefined{shadecolor}{\definecolor{shadecolor}{rgb}{.97, .97, .97}}
\makeatother
\makeatletter
\makeatother
\makeatletter
\makeatother
\ifLuaTeX
  \usepackage{selnolig}  % disable illegal ligatures
\fi
\IfFileExists{bookmark.sty}{\usepackage{bookmark}}{\usepackage{hyperref}}
\IfFileExists{xurl.sty}{\usepackage{xurl}}{} % add URL line breaks if available
\urlstyle{same} % disable monospaced font for URLs
\hypersetup{
  pdftitle={Sentencing Analysis with Pattern Mixture Modeling},
  pdfauthor={C. Clare Strange; Benjamin Stockton; Ofer Harel},
  pdfkeywords={incomplete data, pattern-mixture model},
  colorlinks=true,
  linkcolor={blue},
  filecolor={Maroon},
  citecolor={Blue},
  urlcolor={Blue},
  pdfcreator={LaTeX via pandoc}}

\title{Sentencing Analysis with Pattern Mixture Modeling}
\author{C. Clare Strange \and Benjamin Stockton \and Ofer Harel}
\date{2024-01-15}

\begin{document}
\maketitle
\ifdefined\Shaded\renewenvironment{Shaded}{\begin{tcolorbox}[breakable, interior hidden, enhanced, sharp corners, boxrule=0pt, borderline west={3pt}{0pt}{shadecolor}, frame hidden]}{\end{tcolorbox}}\fi

\textbf{Note: Results in this document are for example. All analyses
were run on a subset of the full data.}

\hypertarget{methods}{%
\section{Methods}\label{methods}}

We propose using multiple imputation and pattern-mixture models to
perform sensitivity analyses for the race effect estimates on the
jail/prison sentence length under incompleteness of the race variable.
Hurdle models with a lognormal generalized linear model for the non-zero
sentence lengths was chosen as it can be used to model complex data
where the dependent variable is a combination of true zeros
(non-incarcerated offenders' zero day jail/prison sentences) and a
continuous distribution for non-zero observations (incarcerated
offenders' jail/prison sentences). Sentence length is one such instance
wherein this phenomena arises with offenders sentenced to a community
sentence receive a jail/prison sentence of 0 days (\(Y^* = 0\) months)
while offenders sentenced to jail or prison time receive a sentence of
\(Y > 0\) days (\(Y^* = Y/30\) months). Note the distinction between
sentence length which would be inclusive of community and jail/prison
sentence lengths which are in general not comparable and our dependent
variable which is solely jail/prison sentence length. When we refer to
sentence length in this manuscript, we are referring specifically to
jail/prison sentence length.

The class of hurdle models fits a logistic regression model to the zero
part of the data and then fits a count or continuous generalized linear
model to the non-zero part of the data. The coefficients from each part
of the model can be interpreted independently or marginally. Predictions
are made from the mixture of the zero and non-zero components. In other
words, we can evaluate the race effect on sentence length either
conditional on the offender being incarcerated, or as the combination of
the probability of receiving a community sentence (zero-days in
jail/prison) and the expected sentence length given the incarceration
decision.

We are defining the race effect measured by the hurdle model as the
change in number of jail/prison days sentenced to between Black and
White offenders. This effect considers the zeros corresponding to
non-incarceration as true zeros, i.e.~the offenders are considered to
have served zero days in jail/prison.

The incomplete data analysis is performed in the multiple imputation
framework (Rubin 1987). In this set-up the incomplete variables sentence
length, race, age, recommended minimum, and previous record are imputed
or filled in \(M\) times using draws from a predictive distribution
using a regression model. See Figure~\ref{fig-miss-pattern} for further
details on the patterns of missing data. Less than 5\% of observations
are incomplete, with race being incomplete on 3\% of cases and the other
variables incomplete on less than 0.5\% of cases each.

The imputations are then used to create \(M\) completed data sets that
are then analyzed separately using a standard complete data method, in
this case a hurdle model. The estimates from each of the \(M\) model
fits are combined using Rubin's rules. In particular, we are going to
use random forests (Wright and Ziegler 2017) to perform the imputations
in the multiple imputation by chained equations framework (Raghunathan
et al. 2001; S. van Buuren and Groothuis-Oudshoorn 2010).

Pattern-mixture models can be used in conjunction with multiple
imputation to perform a sensitivity analysis for the model of interest
to particular perturbations of the distribution from which the
imputations are drawn (S. van Buuren 2018, sec. 3.8). This allows us to
investigate the impacts nonignorable missingness could potentially have
on our analysis.

The pattern-mixture model approach allows the analyst to specify the
exact assumptions of the missingness model and assumptions of how the
distribution varies over different patterns. In our analysis, the two
patterns of interest are the offenders who have a reported race and
those who do not. It is plausible that these two groups have different
characteristics and that the latter group may not have a reported race
because of their true racial/ethnic identity (Stockton, Strange, and
Harel 2023). This is a form of what's known as non-ignorable
missingness.

To assess the impacts of the non-ignorable missingness on the race
effects, we can use slight perturbations of the imputed race labels to
get a broader view of the range of potential estimates under different
distributional assumptions. Varying the distribution of imputations
based on the missing data pattern is what allows us to bring
pattern-mixture modeling into the fold.

\hypertarget{analysis}{%
\section{Analysis}\label{analysis}}

To begin the analysis, I'll create several plots to provide a clearer
context of the data. First, we'll take a look at the distribution of
sentence lengths by year in Figure~\ref{fig-sen-len-yearly} and
Table~\ref{tbl-yearly-summary} to see if and how they differ over time.
Figure~\ref{fig-sen-len-yearly} shows very little difference in the
distributions, displayed as violin plots over time; while
Table~\ref{tbl-yearly-summary} mostly confirms this, there is a slight
trend to shorter sentences and greater probability of community
sentences over time. Maximum sentence lengths in particular also tend to
decrease over time.

\begin{figure}

{\centering \includegraphics{pattern-mixture-modeling_files/figure-pdf/fig-miss-pattern-1.pdf}

}

\caption{\label{fig-miss-pattern}Missing data patterns for the full data
set. In the right panel, red cells indicate missingness.}

\end{figure}

\hypertarget{tbl-yearly-summary}{}
\begin{table}
\caption{\label{tbl-yearly-summary}Summary statistics on sentence length (days) and incarceration. }\tabularnewline

\centering
\begin{tabular}{lrrrrrr}
\toprule
Year & Mean & SD & Median & Min. & Max. & P(Incar)\\
\midrule
2010 & 164.44 & 1284.39 & 0 & 2 & 230468 & 0.51\\
2011 & 164.53 & 591.00 & 0 & 1 & 29220 & 0.50\\
2012 & 161.57 & 669.73 & 0 & 0 & 38167 & 0.49\\
2013 & 165.57 & 612.68 & 0 & 0 & 30865 & 0.49\\
2014 & 160.26 & 689.77 & 0 & 0 & 73048 & 0.49\\
\addlinespace
2015 & 147.45 & 553.58 & 0 & 0 & 29951 & 0.48\\
2016 & 139.62 & 641.51 & 0 & 0 & 32141 & 0.43\\
2017 & 146.53 & 694.47 & 0 & 0 & 87658 & 0.45\\
2018 & 148.29 & 588.65 & 0 & 0 & 25568 & 0.45\\
2019 & 145.81 & 587.12 & 0 & 0 & 18263 & 0.45\\
\bottomrule
\end{tabular}
\end{table}

\begin{figure}

{\centering \includegraphics{pattern-mixture-modeling_files/figure-pdf/fig-sen-len-yearly-1.pdf}

}

\caption{\label{fig-sen-len-yearly}Density plots for A) Sentence length
(day) by year. B) Log sentence lengths (day) plus one day by year.}

\end{figure}

Evaluating the sentence length by most serious crime type in
Figure~\ref{fig-sen-len-crime} shows that the most serious crime type
also has a strong impact on the sentence length with DUIs and Other
offenses resulting in relatively short and often community sentences.
Persons offenses can result in very long offenses but otherwise are
similar in distribution to Drug offenses as seen in the distribution of
the log sentence lengths.

\begin{figure}

{\centering \includegraphics{pattern-mixture-modeling_files/figure-pdf/fig-sen-len-crime-1.pdf}

}

\caption{\label{fig-sen-len-crime}A) Sentence length (day) by most
serious crime type. B) Log of sentence length plus one day by most
serious crime type.}

\end{figure}

The comparison of sentence lengths by offender race in
Figure~\ref{fig-sen-len-race} show that there is little visual
difference in the centers of the distribution for each racial group.
White offenders receive the longest sentences as the result of four
outliers, but otherwise have similar distributions to Black offenders.

\begin{figure}

{\centering \includegraphics{pattern-mixture-modeling_files/figure-pdf/fig-sen-len-race-1.pdf}

}

\caption{\label{fig-sen-len-race}A) Sentence length by offender race. B)
Log of sentence length (plus one day)}

\end{figure}

\hypertarget{multiple-imputation}{%
\subsection{Multiple Imputation}\label{multiple-imputation}}

I'll prepare the data for analysis next. First, I'll center and scale
the numeric predictors which are Offense Gravity Score (OGS), defendant
age, and the square of each. Then I'll use MICE to multiply impute the
incomplete variables with random forest. For now I'll use \(M = 5\)
since less than 5\% of all observations are missing (mainly in defendant
race). The number of imputations to use should be informed by the amount
of missing information due to incompleteness for each variable (Harel
2007). Imputations are made using random forests, a machine learning
technique for modeling categorical and continuous outcomes.

\hypertarget{logistic-regression-on-incarceration}{%
\subsection{Logistic Regression on
Incarceration}\label{logistic-regression-on-incarceration}}

Before modeling the whole sentencing decision, I'll fit a logistic
regression on the incarceration decision as a sanity check. The
estimated odds ratio for increased odds of incarceration for a Black
defendant over a White defendant should be roughly 1.25 as we saw in the
complete case analysis in the previous paper.

\begin{table}

\end{table}

We re-analyze the data using a generalized linear mixed model and again
taking the binary incarceration decision as the outcome and random
effects for the Year and County with random slopes for the most serious
Crime type by County.

\hypertarget{tbl-glmm-sum}{}
\begin{table}
\caption{\label{tbl-glmm-sum}Summary statistics for the fixed effects from the logistic regression
and logistic regression mixed model. }\tabularnewline

\centering
\begin{tabular}{llrrrrl}
\toprule
Model & Term & Estimate & SE & LB 95\% CI & UB 95\% CI & Sig.\\
\midrule
GLM - Logistic & OFF\_RACEBLACK & 0.10 & 0.06 & -0.02 & 0.22 & \\
GLM - Logistic & OFF\_RACELATINO & 0.56 & 0.27 & 0.03 & 1.09 & *\\
GLM - Logistic & OFF\_RACEOTHER & 0.03 & 0.32 & -0.59 & 0.65 & \\
GLMM - Logistic MM & OFF\_RACEBLACK & 0.09 & 0.06 & -0.03 & 0.21 & \\
GLMM - Logistic MM & OFF\_RACELATINO & 0.59 & 0.27 & 0.06 & 1.12 & *\\
\addlinespace
GLMM - Logistic MM & OFF\_RACEOTHER & 0.02 & 0.31 & -0.59 & 0.64 & \\
\bottomrule
\end{tabular}
\end{table}

Race effects for the incarceration decision are reported in
Table~\ref{tbl-glmm-sum}. From the logistic regression fit with MI and
\(M = 5\) imputations done using predictive mean matching, we found that
a Black defendant is 1.28 (95\% CI of (1.133, 1.436) times more likely
to be sentenced to incarceration than an otherwise similar White
defendant. The mixed model reports similar estimated race effect of 1.26
(95\% CI of (1.12, 1.418).

\hypertarget{hurdle-models}{%
\subsection{Hurdle Models}\label{hurdle-models}}

A hurdle model models data with a high number of zeros (compared to
standard distributions). The model is composed of two components: the
hurdle for the zeros and the GLM for the non-zero part. Let
\(\pi_i = P(Y_i = 0)\) be the probability that the \(i\)th observation
is zero and \(P(Y_i \neq 0) = f_{y\neq 0}(y_i)\) where \(f_{y\neq 0}\)
is a truncated probability density function (Cragg 1971). We are first
presenting the analysis of the data with multiple imputations under the
MAR assumption to demonstrate how the model fits the data and how to
interpret the race effect estimates.

\hypertarget{lognormal-glm-hurdle-model-with-predictors-on-hurdle-parameter}{%
\subsubsection{Lognormal GLM Hurdle Model with Predictors on Hurdle
Parameter}\label{lognormal-glm-hurdle-model-with-predictors-on-hurdle-parameter}}

Under this first model, we will model the probability of \(Y_i = 0\) as
a logistic regression on the independent and legally relevant variables
with the R package \texttt{brms} (Bürkner 2017).

\begin{equation}\protect\hypertarget{eq-hurdle-logistic}{}{
\mathrm{logit}^{-1}(P(Y_i = 0)) = \mathbf{x}_i \boldsymbol{\alpha} + \mathbf{z}_i \mathbf{v}.
}\label{eq-hurdle-logistic}\end{equation}

\begin{equation}\protect\hypertarget{eq-hurdle-glm}{}{
\log(Y_i) = \mathbf{x}_i \boldsymbol{\beta} + \mathbf{z}_i \mathbf{u} + \epsilon_i
}\label{eq-hurdle-glm}\end{equation} where \(\mathbf{v}\) and
\(\mathbf{u}\) are independent MVN with
\(E(\mathbf{v}) = E(\mathbf{u}) = 0\) and covariance matrices
\(Cov(\mathbf{v}) = G_v\) and \(Cov(\mathbf{u}) = G_u\),
\(\epsilon_i \overset{iid}{\sim} N(0, \sigma^2)\) and
\(\mathbf{v}, \mathbf{u}\) and \(\boldsymbol{\epsilon}\) are mutually
independent. \(\mathbf{x}_i\) and \(\mathbf{z}_i\) are rows from two
known design matrices for the population-level and group-level effects
respectively.

Here we include group-level effects for the year, the county, and the
crime-type in the county in case the judicial system sentences the
different crime-types differently relative to other counties. Each of
the population-level regression coefficients are given a
weakly-informative normal prior \(\beta_j \sim N(0, 100).\) The
group-level effects for the intercepts and effects for crime-type are
given noncentral t-distributions \(v_k \sim t_{3; 0, 2.5}\) and
\(u_k \sim t_{3; 0, 2.5}\) while the correlations between the
county-level slopes for proportion of cases and county-average OGS and
county-level intercepts are given \(\rho_{i.j} \sim lkj(1)\) priors. The
error term's variance also gets a noncentral t prior
\(\sigma \sim t_{3; 0, 2.5}.\) The coefficients \(\boldsymbol{\alpha}\)
are given normal priors \(\alpha_k \overset{iid}{\sim} N(0, 100).\)

Under this model we are assuming the incomplete data are MAR and the
missingness can be modeled entirely by the multiple imputation
procedure.

\textbf{Dependent Variable}

The length of the jail/prison sentence modeled as a dependent variable
by our hurdle models. The length is measured in days and is zero for
offenders sentenced to a community sentence.

\textbf{Independent Variables}

The independent variables of interest are the offender's race and sex.
Race is coded as White, Black, Latino or Other. Sex is coded as male or
female.

\textbf{Legally Relevant Variables}

We also include legally relevant variables including the crime type,
whether the minimum sentence was recommended, if there was a trial or
plea, the offender's previous record, and the offense gravity score
(OGS). Additionally, the offender's age is included in the model. OGS
and age were centered and scaled before inclusion in the model.

\textbf{County-level Variable}

County-level random effects are included for both the lognormal model
for non-zero lengths, and the logistic model for the incarceration
decision. For each county, the average of the offense gravity score was
included to provide context for the typical cases in the county as well
as the proportion of cases that each county processes out of the total
number of cases processed in the state of Pennsylvania.

\hypertarget{tbl-brms-hurdle-model-summary-2-racesex}{}
\begin{table}
\caption{\label{tbl-brms-hurdle-model-summary-2-racesex}Fixed/population-level effects for the non-zero part of the full
lognormal hurdle glm. }\tabularnewline

\centering
\begin{tabular}{lrrrr}
\toprule
Term & Estimate & SE & LB 95\% CI & UB 95\% CI\\
\midrule
OFF\_RACEBLACK & -0.10 & 0.10 & -0.29 & 0.09\\
OFF\_RACELATINO & -0.86 & 0.60 & -2.03 & 0.32\\
OFF\_RACEOTHER & -1.49 & 0.76 & -3.02 & -0.07\\
SEXMale:OFF\_RACEBLACK & 0.06 & 0.10 & -0.14 & 0.27\\
SEXMale:OFF\_RACELATINO & 1.14 & 0.62 & -0.08 & 2.35\\
\addlinespace
SEXMale:OFF\_RACEOTHER & 1.41 & 0.78 & -0.06 & 3.00\\
\bottomrule
\end{tabular}
\end{table}

\hypertarget{tbl-brms-hurdle-model-summary-2-zero-racesex}{}
\begin{table}
\caption{\label{tbl-brms-hurdle-model-summary-2-zero-racesex}Fixed/population-level effects for the zero part of the full lognormal
hurdle glm. }\tabularnewline

\centering
\begin{tabular}{lrrrr}
\toprule
Term & Estimate & SE & LB 95\% CI & UB 95\% CI\\
\midrule
hu\_OFF\_RACEBLACK & -0.04 & 0.13 & -0.29 & 0.22\\
hu\_OFF\_RACELATINO & 0.57 & 0.83 & -0.98 & 2.27\\
hu\_OFF\_RACEOTHER & 1.15 & 0.92 & -0.42 & 3.18\\
hu\_SEXMale:OFF\_RACEBLACK & -0.15 & 0.14 & -0.43 & 0.13\\
hu\_SEXMale:OFF\_RACELATINO & -1.22 & 0.89 & -3.06 & 0.47\\
\addlinespace
hu\_SEXMale:OFF\_RACEOTHER & -1.28 & 1.00 & -3.44 & 0.48\\
\bottomrule
\end{tabular}
\end{table}

The standard deviation parameter \(\sigma\) of the lognormal
distribution has a posterior mean of 1.04 (95\% CI: {[}1.01, 1.06{]}).

The county-level random/group-level effects and year-level
random/group-level effects are reported in Table~\ref{tbl-brms2-re}.

\hypertarget{tbl-brms2-re}{}
\begin{table}
\caption{\label{tbl-brms2-re}Random/group-level effect standard deviation estimates for the full
hurdle lognormal model. }\tabularnewline

\centering
\begin{tabular}{lrrrr}
\toprule
Term & Estimate & SE & LB 95\% CI & UB 95\% CI\\
\midrule
sd(Intercept) & 0.14 & 0.07 & 0.01 & 0.25\\
sd(COUNTY\_OGS) & 0.04 & 0.02 & 0.00 & 0.08\\
sd(PCASES) & 1.49 & 1.13 & 0.06 & 4.19\\
sd(hu\_Intercept) & 0.39 & 0.19 & 0.03 & 0.69\\
sd(hu\_COUNTY\_OGS) & 0.12 & 0.06 & 0.01 & 0.22\\
\addlinespace
sd(hu\_PCASES) & 3.52 & 2.85 & 0.14 & 10.58\\
\bottomrule
\end{tabular}
\end{table}

In Figure~\ref{fig-cond-eff-brm2-racesex}, we can visualize the impacts
of race and sex on the sentence length in aggregate (including
zero-length sentences) (Figure~\ref{fig-cond-eff-brm2-racesex-1}) and on
the incarceration decision (Figure~\ref{fig-cond-eff-brm2-racesex-2}).
Numeric estimates are reported in Table~\ref{tbl-race-sex-effects}.
Under this model and the MAR missingness assumption, we find that there
is no significant difference in the sentence lengths between the
different racial groups within each sex. There is overlap between all
four credible intervals for both male offenders and female offenders.
Between the sexes, we do see significant differences. Female offenders
receive shorter sentences and are more likely to receive community
sentences regardless of race. There is a large amount of uncertainty in
the effect estimate for Latino offenders. The Other group also has a
relatively high amount of uncertainty and the male Other race offenders'
CI overlaps with all three other groups' CIs for both sexes, although
the mean posterior estimate is lower for sentence length.

\begin{figure}

\begin{minipage}[t]{0.50\linewidth}

{\centering 

\raisebox{-\height}{

\includegraphics{pattern-mixture-modeling_files/figure-pdf/fig-cond-eff-brm2-racesex-1.pdf}

}

}

\subcaption{\label{fig-cond-eff-brm2-racesex-1}Estimates on marginal
sentence length (includes zeros).}
\end{minipage}%
%
\begin{minipage}[t]{0.50\linewidth}

{\centering 

\raisebox{-\height}{

\includegraphics{pattern-mixture-modeling_files/figure-pdf/fig-cond-eff-brm2-racesex-2.pdf}

}

}

\subcaption{\label{fig-cond-eff-brm2-racesex-2}Estimates on probability
of incarceration from the logistic regression on the zero-part of the
full hurdle lognormal model.}
\end{minipage}%

\caption{\label{fig-cond-eff-brm2-racesex}Posterior estimates of the
interaction between race and sex effects on the sentence length in
days.}

\end{figure}

Results for the other terms in the model are reported in the Appendix.

\hypertarget{tbl-race-sex-effects}{}
\begin{table}
\caption{\label{tbl-race-sex-effects}Estimates for sentence length combining the non-zero and zero
predictions for the eight combinations of race and sex. }\tabularnewline

\centering
\begin{tabular}{llrrrr}
\toprule
Sex & Race & Est & SE & LB 95\% CI & UB 95\% CI\\
\midrule
Female & WHITE & 15.97 & 1.79 & 12.76 & 19.83\\
Female & BLACK & 14.82 & 2.32 & 10.78 & 20.07\\
Female & LATINO & 4.23 & 3.52 & 0.59 & 22.09\\
Female & OTHER & 1.42 & 1.45 & 0.10 & 10.53\\
Male & WHITE & 24.96 & 2.40 & 20.57 & 30.12\\
\addlinespace
Male & BLACK & 27.21 & 2.84 & 22.10 & 33.15\\
Male & LATINO & 49.46 & 12.19 & 29.97 & 78.34\\
Male & OTHER & 25.01 & 8.14 & 12.90 & 45.92\\
\bottomrule
\end{tabular}
\end{table}

\hypertarget{sensitivity-analysis}{%
\section{Sensitivity Analysis}\label{sensitivity-analysis}}

Evaluating the impacts of various nonignorable missingness mechanisms
can be accomplished with pattern-mixture models. The values of the
incomplete numeric data can be scaled or shifted. The same cannot be
done with categorical imputations, instead the proportion of each
category can be varied within the imputations compared to the observed
distribution or the MAR imputed distribution.

While there are very few missing sentence lengths, I could also modify
the imputed sentence lengths with a scale \(c\) or shift \(\delta\).
Scaling maintains that offenders that are non-incarcerated will remain
non-incarcerated while the sentences of incarcerated defenders would
shift. A positive shift would make all offenders incarcerated while a
negative shift would impose negative sentence lengths that would need to
be corrected to use Poisson or lognormal hurdle glms.

To perturb the imputations, we modify the vector of probabilities
\(\mathbf{p} = (p_1, p_2, p_3, p_4)'\) of class membership for each
racial/ethnic group under consideration (White, Black, Latino, or
Other). A vector of scale parameters
\(\mathbf{c} = (c_1, c_2, c_3, c_4)'\) is chosen such that the new
racial/ethnic group label will be drawn from the normalized vector
\(\mathbf{p}^* = \frac{1}{\sum_{j=1}^4 c_j p_j} (c_1 p_1, c_2 p_2, c_3 p_3, c_4 p_4)'\)
so that the probability vectors remains the same if \(c_j = 1\) for each
\(j = 1,\dots,4.\) For simplicity, we will focus on varying the
probability of assigning a White label by changing only \(c_1\) while
holding \(c_2 = c_3 = c_4 = 1.\)

The vector of probabilities \(\mathbf{p}_i\) is obtained for each
incomplete observation \(i\) by re-fitting the same random forest used
in the initial mulitple imputation step. This vector is then scaled and
normalized using the same scaling vector \(\mathbf{c}\) for all
incomplete observations. The probability vectors \(\mathbf{p}^*\) are
used to draw new imputed racial/ethnic group labels for the incomplete
observations. This procedure is repeated, including the model fitting,
across each of the \(M\) completed data sets resulting in a new set of
\(M\) completed data sets with modified imputations. The new set of
completed data sets is then analyzed as before using an lognormal hurdle
model. Estimates from each variation of \(\mathbf{c}\) that is chosen
are then compared graphically in Figure~\ref{fig-sens-analysis-res}.

We choose to vary \(c_1\) along the sequence
\(\{0.1, 0.33, 0.75, 0.9, 1, 1.1, 1.25, 1.5, 2\}.\) When \(c_1 = 1\),
the imputations correspond to the original imputation model under the
MAR assumption. For \(c_1 < 1\), fewer observations are imputed with a
White label than under the MAR assumption reflecting a more diverse
population of offenders who have unknown or unreported race labels. The
opposite is true for \(c_1 > 1\), reflecting a more White population of
offenders. We include cases such as \(c_1 = 0.1\) and \(c_1 = 0.33\) as
well as \(c_1 = 1.5\) and \(c_1 = 2\) as rather implausible extreme
conditions to investigate what could happen in the most extreme
circumstances.

\begin{figure}

\begin{minipage}[t]{0.50\linewidth}

{\centering 

\raisebox{-\height}{

\includegraphics{pattern-mixture-modeling_files/figure-pdf/fig-sens-analysis-res-1.pdf}

}

}

\subcaption{\label{fig-sens-analysis-res-1}White offenders}
\end{minipage}%
%
\begin{minipage}[t]{0.50\linewidth}

{\centering 

\raisebox{-\height}{

\includegraphics{pattern-mixture-modeling_files/figure-pdf/fig-sens-analysis-res-2.pdf}

}

}

\subcaption{\label{fig-sens-analysis-res-2}Black offenders}
\end{minipage}%
\newline
\begin{minipage}[t]{0.50\linewidth}

{\centering 

\raisebox{-\height}{

\includegraphics{pattern-mixture-modeling_files/figure-pdf/fig-sens-analysis-res-3.pdf}

}

}

\subcaption{\label{fig-sens-analysis-res-3}Latino offenders}
\end{minipage}%
%
\begin{minipage}[t]{0.50\linewidth}

{\centering 

\raisebox{-\height}{

\includegraphics{pattern-mixture-modeling_files/figure-pdf/fig-sens-analysis-res-4.pdf}

}

}

\subcaption{\label{fig-sens-analysis-res-4}Other ethnicity offenders}
\end{minipage}%

\caption{\label{fig-sens-analysis-res}Coefficient estimates for the
logistic regression model predicting incarceration.}

\end{figure}

Figure~\ref{fig-sens-analysis-res} displays the race effects on sentence
length for the White, Black, Latino, and Other racial/ethnic groups. In
general, we see that estimates tend to be very stable for White and
Black offenders, while there's more variation in the estimates for
Latino offenders and offenders with an Other race/ethnicity reported.

\newpage

\hypertarget{references}{%
\section*{References}\label{references}}
\addcontentsline{toc}{section}{References}

\hypertarget{refs}{}
\begin{CSLReferences}{1}{0}
\leavevmode\vadjust pre{\hypertarget{ref-buxfcrkner2017}{}}%
Bürkner, Paul-Christian. 2017. {``Brms: An R Package for Bayesian
Multilevel Models Using Stan.''} \emph{Journal of Statistical Software}
80 (August): 1--28. \url{https://doi.org/10.18637/jss.v080.i01}.

\leavevmode\vadjust pre{\hypertarget{ref-buuren2010mice}{}}%
Buuren, S van, and Karin Groothuis-Oudshoorn. 2010. {``Mice:
Multivariate Imputation by Chained Equations in r.''} \emph{Journal of
Statistical Software}, 168.

\leavevmode\vadjust pre{\hypertarget{ref-vanbuuren2018}{}}%
Buuren, Stef van. 2018. \emph{Flexible Imputation of Missing Data}. 2nd
ed. Interdisciplinary Statistics Series. Chapman; Hall/CRC.
\url{https://stefvanbuuren.name/fimd/}.

\leavevmode\vadjust pre{\hypertarget{ref-craggStatisticalModelsLimited1971}{}}%
Cragg, John G. 1971. {``Some {Statistical Models} for {Limited Dependent
Variables} with {Application} to the {Demand} for {Durable Goods}.''}
\emph{Econometrica} 39 (5): 829--44.
\url{https://doi.org/10.2307/1909582}.

\leavevmode\vadjust pre{\hypertarget{ref-harelInferencesMissingInformation2007}{}}%
Harel, Ofer. 2007. {``Inferences on Missing Information Under Multiple
Imputation and Two-Stage Multiple Imputation.''} \emph{Statistical
Methodology} 4 (1): 75--89.
\url{https://doi.org/10.1016/j.stamet.2006.03.002}.

\leavevmode\vadjust pre{\hypertarget{ref-raghunathanMultivariateTechniqueMultiply2001}{}}%
Raghunathan, Trivellore E, James M Lepkowski, John Van Hoewyk, and Peter
Solenberger. 2001. {``A {Multivariate Technique} for {Multiply Imputing
Missing Values Using} a {Sequence} of {Regression Models}.''}
\emph{Statistics Canada} 27 (1): 85--95.

\leavevmode\vadjust pre{\hypertarget{ref-rubinMultipleImputationNonresponse1987}{}}%
Rubin, Donald B. 1987. \emph{{Multiple Imputation for Nonresponse in
Surveys \textbar{} Wiley Series in Probability and Statistics}}. {Wiley
series in probability and mathematical statistics : Applied probability
and statistics}. {New York}: {Wiley}.

\leavevmode\vadjust pre{\hypertarget{ref-stockton2023}{}}%
Stockton, Benjamin, C. Clare Strange, and Ofer Harel. 2023. {``Now You
See It, Now You Don{'}t: A Simulation and Illustration of the Importance
of Treating Incomplete Data in Estimating Race Effects in Sentencing.''}
\emph{Journal of Quantitative Criminology}, September.
\url{https://doi.org/10.1007/s10940-023-09577-w}.

\leavevmode\vadjust pre{\hypertarget{ref-wrightRangerFastImplementation2017}{}}%
Wright, Marvin N., and Andreas Ziegler. 2017. {``Ranger: {A Fast
Implementation} of {Random Forests} for {High Dimensional Data} in {C}++
and {R}.''} \emph{Journal of Statistical Software} 77 (March): 1--17.
\url{https://doi.org/10.18637/jss.v077.i01}.

\end{CSLReferences}

\newpage

\hypertarget{appendix}{%
\section*{Appendix}\label{appendix}}
\addcontentsline{toc}{section}{Appendix}

\begin{figure}

{\centering \includegraphics{pattern-mixture-modeling_files/figure-pdf/fig-incar-cime-1.pdf}

}

\caption{\label{fig-incar-cime}Incarceration decision by most serious
crime type. INCAR == 1 indicates incarceration. INCAR == 0 indicates
parole.}

\end{figure}

\begin{figure}

\begin{minipage}[t]{0.50\linewidth}

{\centering 

\raisebox{-\height}{

\includegraphics{pattern-mixture-modeling_files/figure-pdf/fig-sen-len-qq-1.pdf}

}

}

\subcaption{\label{fig-sen-len-qq-1}Sentence length in days}
\end{minipage}%
%
\begin{minipage}[t]{0.50\linewidth}

{\centering 

\raisebox{-\height}{

\includegraphics{pattern-mixture-modeling_files/figure-pdf/fig-sen-len-qq-2.pdf}

}

}

\subcaption{\label{fig-sen-len-qq-2}Log of sentence length plus one day}
\end{minipage}%

\caption{\label{fig-sen-len-qq}QQ plots of the sentence length (days)}

\end{figure}

\hypertarget{tbl-brms-hurdle-model-summary-2}{}
\begin{table}
\caption{\label{tbl-brms-hurdle-model-summary-2}Fixed/population-level effects for the non-zero part of the full
lognormal hurdle glm. }\tabularnewline

\centering
\begin{tabular}{lrrrr}
\toprule
Term & Estimate & SE & LB 95\% CI & UB 95\% CI\\
\midrule
Intercept & 3.80 & 0.07 & 3.66 & 3.93\\
DOSAGE & 0.03 & 0.02 & -0.01 & 0.07\\
DOSAGEQ & -0.03 & 0.01 & -0.05 & 0.00\\
SEXMale & 0.19 & 0.05 & 0.09 & 0.28\\
OFF\_RACEBLACK & -0.10 & 0.10 & -0.29 & 0.09\\
\addlinespace
OFF\_RACELATINO & -0.86 & 0.60 & -2.03 & 0.32\\
OFF\_RACEOTHER & -1.49 & 0.76 & -3.02 & -0.07\\
OGS & 1.31 & 0.04 & 1.24 & 1.38\\
OGSQ & -0.13 & 0.01 & -0.15 & -0.11\\
PRVREC1D2D3 & 0.32 & 0.04 & 0.24 & 0.39\\
\addlinespace
PRVREC4D5 & 0.78 & 0.06 & 0.67 & 0.89\\
PRVRECREVOCDRFEL & 1.30 & 0.10 & 1.10 & 1.50\\
RECMIN & 0.20 & 0.06 & 0.08 & 0.32\\
CRIMEDUI & -0.88 & 0.05 & -0.99 & -0.78\\
CRIMEOther & 0.11 & 0.06 & 0.00 & 0.23\\
\addlinespace
CRIMEPersons & 0.15 & 0.05 & 0.05 & 0.26\\
CRIMEProperty & 0.25 & 0.05 & 0.16 & 0.35\\
TRIAL & 0.37 & 0.09 & 0.19 & 0.55\\
SEXMale:OFF\_RACEBLACK & 0.06 & 0.10 & -0.14 & 0.27\\
SEXMale:OFF\_RACELATINO & 1.14 & 0.62 & -0.08 & 2.35\\
\addlinespace
SEXMale:OFF\_RACEOTHER & 1.41 & 0.78 & -0.06 & 3.00\\
OGS:PRVREC1D2D3 & 0.02 & 0.03 & -0.04 & 0.08\\
OGS:PRVREC4D5 & 0.00 & 0.03 & -0.07 & 0.07\\
OGS:PRVRECREVOCDRFEL & -0.18 & 0.06 & -0.30 & -0.05\\
\bottomrule
\end{tabular}
\end{table}

\hypertarget{tbl-brms-hurdle-model-summary-2-zero}{}
\begin{table}
\caption{\label{tbl-brms-hurdle-model-summary-2-zero}Fixed/population-level effects for the zero part of the full lognormal
hurdle glm. }\tabularnewline

\centering
\begin{tabular}{lrrrr}
\toprule
Term & Estimate & SE & LB 95\% CI & UB 95\% CI\\
\midrule
hu\_Intercept & 1.71 & 0.12 & 1.47 & 1.95\\
hu\_DOSAGE & 0.13 & 0.03 & 0.06 & 0.19\\
hu\_DOSAGEQ & 0.04 & 0.02 & 0.00 & 0.08\\
hu\_SEXMale & -0.35 & 0.07 & -0.48 & -0.22\\
hu\_OFF\_RACEBLACK & -0.04 & 0.13 & -0.29 & 0.22\\
\addlinespace
hu\_OFF\_RACELATINO & 0.57 & 0.83 & -0.98 & 2.27\\
hu\_OFF\_RACEOTHER & 1.15 & 0.92 & -0.42 & 3.18\\
hu\_RECMIN & -0.84 & 0.09 & -1.02 & -0.67\\
hu\_OGS & -0.42 & 0.05 & -0.51 & -0.33\\
hu\_OGSQ & -0.24 & 0.03 & -0.29 & -0.18\\
\addlinespace
hu\_PRVREC1D2D3 & -0.50 & 0.06 & -0.61 & -0.38\\
hu\_PRVREC4D5 & -1.08 & 0.09 & -1.26 & -0.91\\
hu\_PRVRECREVOCDRFEL & -1.50 & 0.18 & -1.85 & -1.16\\
hu\_CRIMEDUI & -1.62 & 0.08 & -1.77 & -1.47\\
hu\_CRIMEOther & -0.19 & 0.09 & -0.36 & -0.02\\
\addlinespace
hu\_CRIMEPersons & -0.71 & 0.08 & -0.87 & -0.55\\
hu\_CRIMEProperty & -0.48 & 0.07 & -0.62 & -0.34\\
hu\_TRIAL & -0.54 & 0.20 & -0.94 & -0.15\\
hu\_SEXMale:OFF\_RACEBLACK & -0.15 & 0.14 & -0.43 & 0.13\\
hu\_SEXMale:OFF\_RACELATINO & -1.22 & 0.89 & -3.06 & 0.47\\
\addlinespace
hu\_SEXMale:OFF\_RACEOTHER & -1.28 & 1.00 & -3.44 & 0.48\\
\bottomrule
\end{tabular}
\end{table}

\begin{figure}

\begin{minipage}[t]{0.33\linewidth}

{\centering 

\raisebox{-\height}{

\includegraphics{pattern-mixture-modeling_files/figure-pdf/fig-cond-eff-brm2-nonzero-1.pdf}

}

}

\subcaption{\label{fig-cond-eff-brm2-nonzero-1}Age}
\end{minipage}%
%
\begin{minipage}[t]{0.33\linewidth}

{\centering 

\raisebox{-\height}{

\includegraphics{pattern-mixture-modeling_files/figure-pdf/fig-cond-eff-brm2-nonzero-2.pdf}

}

}

\subcaption{\label{fig-cond-eff-brm2-nonzero-2}Age Squared}
\end{minipage}%
%
\begin{minipage}[t]{0.33\linewidth}

{\centering 

\raisebox{-\height}{

\includegraphics{pattern-mixture-modeling_files/figure-pdf/fig-cond-eff-brm2-nonzero-3.pdf}

}

}

\subcaption{\label{fig-cond-eff-brm2-nonzero-3}Sex}
\end{minipage}%
\newline
\begin{minipage}[t]{0.33\linewidth}

{\centering 

\raisebox{-\height}{

\includegraphics{pattern-mixture-modeling_files/figure-pdf/fig-cond-eff-brm2-nonzero-4.pdf}

}

}

\subcaption{\label{fig-cond-eff-brm2-nonzero-4}Offender Race}
\end{minipage}%
%
\begin{minipage}[t]{0.33\linewidth}

{\centering 

\raisebox{-\height}{

\includegraphics{pattern-mixture-modeling_files/figure-pdf/fig-cond-eff-brm2-nonzero-5.pdf}

}

}

\subcaption{\label{fig-cond-eff-brm2-nonzero-5}OGS}
\end{minipage}%
%
\begin{minipage}[t]{0.33\linewidth}

{\centering 

\raisebox{-\height}{

\includegraphics{pattern-mixture-modeling_files/figure-pdf/fig-cond-eff-brm2-nonzero-6.pdf}

}

}

\subcaption{\label{fig-cond-eff-brm2-nonzero-6}OGS Squared}
\end{minipage}%
\newline
\begin{minipage}[t]{0.33\linewidth}

{\centering 

\raisebox{-\height}{

\includegraphics{pattern-mixture-modeling_files/figure-pdf/fig-cond-eff-brm2-nonzero-7.pdf}

}

}

\subcaption{\label{fig-cond-eff-brm2-nonzero-7}Prev. Rec.}
\end{minipage}%
%
\begin{minipage}[t]{0.33\linewidth}

{\centering 

\raisebox{-\height}{

\includegraphics{pattern-mixture-modeling_files/figure-pdf/fig-cond-eff-brm2-nonzero-8.pdf}

}

}

\subcaption{\label{fig-cond-eff-brm2-nonzero-8}Rec. Min.}
\end{minipage}%
%
\begin{minipage}[t]{0.33\linewidth}

{\centering 

\raisebox{-\height}{

\includegraphics{pattern-mixture-modeling_files/figure-pdf/fig-cond-eff-brm2-nonzero-9.pdf}

}

}

\subcaption{\label{fig-cond-eff-brm2-nonzero-9}Crime Type}
\end{minipage}%
\newline
\begin{minipage}[t]{0.33\linewidth}

{\centering 

\raisebox{-\height}{

\includegraphics{pattern-mixture-modeling_files/figure-pdf/fig-cond-eff-brm2-nonzero-10.pdf}

}

}

\subcaption{\label{fig-cond-eff-brm2-nonzero-10}Trial}
\end{minipage}%
%
\begin{minipage}[t]{0.33\linewidth}

{\centering 

\raisebox{-\height}{

\includegraphics{pattern-mixture-modeling_files/figure-pdf/fig-cond-eff-brm2-nonzero-11.pdf}

}

}

\subcaption{\label{fig-cond-eff-brm2-nonzero-11}Race by Sex}
\end{minipage}%
%
\begin{minipage}[t]{0.33\linewidth}

{\centering 

\raisebox{-\height}{

\includegraphics{pattern-mixture-modeling_files/figure-pdf/fig-cond-eff-brm2-nonzero-12.pdf}

}

}

\subcaption{\label{fig-cond-eff-brm2-nonzero-12}OGS by Prev. Rec.}
\end{minipage}%

\caption{\label{fig-cond-eff-brm2-nonzero}Posterior estimates of the
conditional effects for the lognormal glm for the non-zero-part of the
full hurdle lognormal model.}

\end{figure}

\begin{figure}

\begin{minipage}[t]{0.33\linewidth}

{\centering 

\raisebox{-\height}{

\includegraphics{pattern-mixture-modeling_files/figure-pdf/fig-cond-eff-brm2-1.pdf}

}

}

\subcaption{\label{fig-cond-eff-brm2-1}Age}
\end{minipage}%
%
\begin{minipage}[t]{0.33\linewidth}

{\centering 

\raisebox{-\height}{

\includegraphics{pattern-mixture-modeling_files/figure-pdf/fig-cond-eff-brm2-2.pdf}

}

}

\subcaption{\label{fig-cond-eff-brm2-2}Age Squared}
\end{minipage}%
%
\begin{minipage}[t]{0.33\linewidth}

{\centering 

\raisebox{-\height}{

\includegraphics{pattern-mixture-modeling_files/figure-pdf/fig-cond-eff-brm2-3.pdf}

}

}

\subcaption{\label{fig-cond-eff-brm2-3}Sex}
\end{minipage}%
\newline
\begin{minipage}[t]{0.33\linewidth}

{\centering 

\raisebox{-\height}{

\includegraphics{pattern-mixture-modeling_files/figure-pdf/fig-cond-eff-brm2-4.pdf}

}

}

\subcaption{\label{fig-cond-eff-brm2-4}Offender Race}
\end{minipage}%
%
\begin{minipage}[t]{0.33\linewidth}

{\centering 

\raisebox{-\height}{

\includegraphics{pattern-mixture-modeling_files/figure-pdf/fig-cond-eff-brm2-5.pdf}

}

}

\subcaption{\label{fig-cond-eff-brm2-5}OGS}
\end{minipage}%
%
\begin{minipage}[t]{0.33\linewidth}

{\centering 

\raisebox{-\height}{

\includegraphics{pattern-mixture-modeling_files/figure-pdf/fig-cond-eff-brm2-6.pdf}

}

}

\subcaption{\label{fig-cond-eff-brm2-6}OGS Squared}
\end{minipage}%
\newline
\begin{minipage}[t]{0.33\linewidth}

{\centering 

\raisebox{-\height}{

\includegraphics{pattern-mixture-modeling_files/figure-pdf/fig-cond-eff-brm2-7.pdf}

}

}

\subcaption{\label{fig-cond-eff-brm2-7}Prev. Rec.}
\end{minipage}%
%
\begin{minipage}[t]{0.33\linewidth}

{\centering 

\raisebox{-\height}{

\includegraphics{pattern-mixture-modeling_files/figure-pdf/fig-cond-eff-brm2-8.pdf}

}

}

\subcaption{\label{fig-cond-eff-brm2-8}Rec. Min.}
\end{minipage}%
%
\begin{minipage}[t]{0.33\linewidth}

{\centering 

\raisebox{-\height}{

\includegraphics{pattern-mixture-modeling_files/figure-pdf/fig-cond-eff-brm2-9.pdf}

}

}

\subcaption{\label{fig-cond-eff-brm2-9}Crime Type}
\end{minipage}%
\newline
\begin{minipage}[t]{0.33\linewidth}

{\centering 

\raisebox{-\height}{

\includegraphics{pattern-mixture-modeling_files/figure-pdf/fig-cond-eff-brm2-10.pdf}

}

}

\subcaption{\label{fig-cond-eff-brm2-10}Trial}
\end{minipage}%
%
\begin{minipage}[t]{0.33\linewidth}

{\centering 

\raisebox{-\height}{

\includegraphics{pattern-mixture-modeling_files/figure-pdf/fig-cond-eff-brm2-11.pdf}

}

}

\subcaption{\label{fig-cond-eff-brm2-11}Race by Sex}
\end{minipage}%
%
\begin{minipage}[t]{0.33\linewidth}

{\centering 

\raisebox{-\height}{

\includegraphics{pattern-mixture-modeling_files/figure-pdf/fig-cond-eff-brm2-12.pdf}

}

}

\subcaption{\label{fig-cond-eff-brm2-12}OGS by Prev. Rec.}
\end{minipage}%

\caption{\label{fig-cond-eff-brm2}Posterior estimates of the conditional
effects for the logistic regression on the zero-part of the full hurdle
lognormal model.}

\end{figure}

\begin{figure}

\begin{minipage}[t]{0.33\linewidth}

{\centering 

\raisebox{-\height}{

\includegraphics{pattern-mixture-modeling_files/figure-pdf/fig-ppc-brm2-1.pdf}

}

}

\subcaption{\label{fig-ppc-brm2-1}PPC Density Sentence Length (days)}
\end{minipage}%
%
\begin{minipage}[t]{0.33\linewidth}

{\centering 

\raisebox{-\height}{

\includegraphics{pattern-mixture-modeling_files/figure-pdf/fig-ppc-brm2-2.pdf}

}

}

\subcaption{\label{fig-ppc-brm2-2}PPC Density Log of Sentence Length
plus 1}
\end{minipage}%
%
\begin{minipage}[t]{0.33\linewidth}

{\centering 

\raisebox{-\height}{

\includegraphics{pattern-mixture-modeling_files/figure-pdf/fig-ppc-brm2-3.pdf}

}

}

\subcaption{\label{fig-ppc-brm2-3}PPC Scatter of average error by age}
\end{minipage}%

\caption{\label{fig-ppc-brm2}Posterior Predictive Checks for the full
lognormal hurdle model.}

\end{figure}



\end{document}
