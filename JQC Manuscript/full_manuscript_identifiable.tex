%% Full length research paper template
%% Created by Simon Hengchen and Nilo Pedrazzini for the Journal of Open Humanities Data (https://openhumanitiesdata.metajnl.com)

\documentclass[titlepage]{article}
\usepackage[english]{babel}
\usepackage[utf8]{inputenc}
\usepackage{amsfonts}
\usepackage{authblk}
\usepackage{fancyhdr}
\usepackage[skip=10pt plus1pt, indent=0pt]{parskip}
\usepackage[paperheight=11in,paperwidth=8.5in,textwidth=6.5in,textheight=9in, margin = 1in, headheight=0.5in, headsep=.25in]{geometry}

\renewcommand{\thefootnote}{\fnsymbol{footnote}}
\title{Now you see it, now you don't: A simulation and illustration of the importance of treating incomplete data in estimating race effects in criminology}

\author[1]{Benjamin Stockton}
\author[2]{C. Clare Strange \thanks{Corresponding author: \\ 3401 Market Street, Suite 110, Philadelphia, PA 19104 \\ P: (215)571-4628 \\ E: \tt{cs3846@drexel.edu} \\ \\ Acknowledgement: The authors would like to thank the Pennsylvania Commission on Sentencing for providing the data that were used to construct the simulated data set. }}
\author[1]{Ofer Harel}

\affil[1]{Department of Statistics, University of Connecticut, Storrs, CT}
\affil[2]{Department of Criminology and Justice Studies, Drexel University, Philadelphia, PA}


\date{} %leave blank

\begin{document}
\pagestyle{fancy}
\fancyhf{}
\fancyhead[L]{ESTIMATING RACE EFFECTS IN CRIMINOLOGY}
\fancyfoot[R]{\thepage}

\maketitle

\renewcommand{\thefootnote}{\arabic{footnote}}

\newpage

\begin{abstract} 
\noindent \textbf{Objectives:} Evaluate the impact of incomplete data (i.e., data with missing observations) on observed racial disparities in the likelihood of an incarceration sentence.

\noindent \textbf{Methods:} Using a simulation study with data based on cases sentenced in the Court of Common Pleas in Pennsylvania between 2010 and 2019, we assess the differences in the likelihood of incarceration between similarly situated White and Black defendants based on varying sample sizes and patterns of data missingness.

\noindent \textbf{Results:} Complete case analysis (CCA) of incomplete data (i.e., the modal analytic approach in criminological research) can fail to provide unbiased estimates of the race effect, even with less than 10\% of cases missing. Multiple imputation provides an alternative, valid methodology for the unbiased estimation of race effects (and other factors) when data are missing at random (MAR). These results largely hold across sample sizes and number of imputations.

\noindent \textbf{Conclusions:} The existence and magnitude of race effects on the likelihood of an incarceration sentence can vary greatly based on the degree and treatment of missing data. Limitations include that missing data mechanisms cannot be truly known outside of a data simulation. Future research should prioritize the identification, treatment, and reporting of missing data prior to isolating race effects, in line with calls from the field for more open science practices.

\noindent \textbf{Keywords:} incomplete data; missing data mechanisms; race effects; complete case analysis; multiple imputation \\
 \end{abstract}



% \noindent\authorroles{For determining author roles, please use following taxonomy: \url{https://credit.niso.org/}. Please list the roles for each author.} 

\section{Introduction}

"Race effects," or the empirical contributions of one’s racial identity to a given outcome, have long been a subject of interest in criminological research (Green, 1961;1964; Johnson, 1957; Steffensmeier, Ulmer, \& Kramer, 1998, Light 2022; Mitchell et al., 2022). While the general consensus is that the direct influence of race has declined (see King \& Light, 2019; Light, 2022; Mitchell et al., 2022), still decades of research across global contexts reveals significant race effects with regard to arrest (Hepburn, 1978; Owusu-Bempah \& Luscombe, 2021; Stevens \& Morash, 2015), court and sentencing (Steffensmeier et al., 1998; Wooldredge et al., 2015), correctional (Mears et al., 2021; Snowball \& Weatherburn, 2007), and community supervision and rehabilitation outcomes (Skeem \& Lowenkamp, 2016; Steinmetz \& Henderson, 2016).

Race effects typically vary within and between studies, however, fueling debates among scholars about the sources of this variation (Albonetti, 1991; Franklin, 2018; Mitchell, 2005; Pratt, 1999; Spohn, 2000; 2015). Much research has explored how the race effect changes depending on organizational, community, and temporal contexts (Gaston, 2019; Johnson \& Lee, 2013; Feldmeyer et al., 2015; Light, 2022), as well as individuals’ "constellations" of characteristics (i.e., race in conjunction with ethnicity, age, gender, employment or immigration status, crime type, or criminal history, for example) (Franklin \& Henry, 2020; Jordan \& McNeal, 2016; Spohn \& Holleran, 2000; Steffensmeier et al., 1998). These explorations highlight the complicated task of isolating race effects and determining their theoretical drivers.

We offer an additional potential explanation for variations in race effects in criminology: the treatment of incomplete data. Few studies consider how methodological decisions about missing data impact the race effect. To fill this gap, we employ a data simulation to illustrate whether different types of missingness assumptions can significantly alter the existence and magnitude of observed race effects on an outcome of interest. This exploration has important implications at a time when issues of racial justice and criminal legal reform are ubiquitous in the public discourse, and racial impact statements regularly accompany policy proposals\footnote{See https://www.sentencingproject.org/publications/racial-impact-statements/}.  More than ever, it is important to consider how the treatment of missing data impacts observed race effects in criminological research.

\section{BACKGROUND}

\subsection{The "Race Effect" in Courts and Sentencing}
Race effects have been measured across many criminological contexts; however, courts provide one clear and well-researched example for the sake of illustration. Scholars have found that Black defendants tend to suffer harsher outcomes than their White counterparts at each stage of the court process, even after controlling for legal factors like criminal history, current offense type and severity, and policies relating to sentencing guidelines and mandatory minimums. Black defendants are often charged with more crimes or with crimes that carry more severe punishment implications (Rehavi \& Starr, 2014; Wu, 2016), held more often in pretrial detention (Martinez et al., 2020; Wooldredge, 2012), and receive higher bail amounts (Schlesinger, 2008) and more disadvantageous plea bargains (Kutateladze et al., 2016; Metcalfe \& Chiricos, 2018).

Race effects may be small or statistically non-significant at any one decision-point but tend to accumulate and produce larger disparities later in the legal process (Hickert et al., 2022; Martinez et al., 2020; Wooldredge et al., 2015). This "cumulative disadvantage" (see Kurlychek \& Johnson [2019] for a detailed review) is most readily observed at sentencing, the final step in the legal process prior to the imposition of punishment. Compounded by earlier decision-points, Black defendants most consistently receive the harshest sanctions over any other racial group (Mitchell, 2005; Omori \& Johnson, 2019; Ridgeway et al., 2020; Spohn, 2000). For example, a Black defendant is typically more likely than their White counterpart to receive an incarceration sentence (i.e., jail or prison) instead of a community-based sentence (e.g., probation), even in situations where a community sentence is justifiable. 

While race effects are often present in sentencing research, their magnitudes vary. In their seminal study on race, age, gender, and punishment in Pennsylvania, Steffensmeier et al. (1998) found Black defendants were ten percent more likely than their White counterparts to be incarcerated. Conversely, Ridgeway and colleagues (2020) found that Black defendants in New York were only slightly more likely than White defendants to receive a prison sentence (a three-percentage point difference). Omori and Petersen (2019) found larger and more varied effects when studying felony cases in Miami-Dade County. They assessed a five to six percentage point difference in the likelihood of a prison sentence, and a four to ten percentage point difference in the likelihood of a jail sentence when comparing White, Non-Latinos to Black Latinos and Black, non-Latinos. At the Federal level, Doerner and Demuth (2010) found the difference in the odds of incarceration between Black and White defendants was too small to reach statistical significance (Black OR = 1.060). In a systematic assessment of the body of sentencing research, Baumer (2013) emphasized his (and other sentencing scholars’) perceptions that this variation in race effects may stem from both theoretical and methodological sources.

\subsection{Explaining Differences in the "Race Effect" in Courts and Sentencing}

The potential theoretical explanations for differing race effects in sentencing research are many. While not the focus of our study, scholars have attributed race effects to court actor discretion in individual cases (Albonetti, 1991, Bushway \& Piehl, 2001; Steffensmeier et al., 1998; Ulmer, Light, Kramer, \& Eisenstein, 2011), the differential impacts of sentencing policy across racial groups (Baumer, 2013; Pratt, 1998), and the sociopolitical contexts of courts (Blumer, 1958; Feldmeyer et al., 2015; Hester \& Sevigny, 2016; Ulmer \& Johnson, 2004), for example. Scholars continue to evaluate these as the theoretical drivers of sentencing disparities between legally similar defendants of different races.

Relevant to the current study, scholars have also highlighted potential methodological explanations for differences in race effects across studies. There is no shortage of methodological critiques of research on race and sentencing (see, e.g., Baumer, 2013; Bushway et al., 2007; Lynch, 2019; MacDonald \& Donnelly, 2019; Mitchell, 2005; Spohn, 2015; Ulmer, 2012), and this literature has encouraged a more rigorous line of study. Several methodological concerns are common across these critiques, including measurement decisions, sample selection bias, and choice of modeling strategies.

Measurement decisions, like the inclusion of (and degree of specificity in) legally relevant variables, are considered one avenue through which race effects may vary (Baumer, 2013; Mitchell, 2005). Mitchell (2005) found that "less precise measures of criminal history and offense seriousness produced larger estimates of unwarranted racial disparity" in sentencing (p. 457). Mitchell went on to qualify, though, that "even after taking into account precision of measurement and presence of key variables, the race effect persists; however, the average estimate of racial disadvantage drops considerably" (p. 462).

Another avenue for disparate race effects in sentencing is sample selection bias. In this context, bias stems from the fact that only those who are sentenced to incarceration can be included in studies of sentence length disparities. This concern dates to the 1970s and has traditionally been addressed through a two-step estimator first developed by Heckman (1976). More recently, Bushway and colleagues (2007) found that, though bias was lesser for race estimates relative to other factors like gender and ethnicity, sample selection bias (even when employing the "Heckman correction") remains a concern in race and sentencing research that is specific to sentence length. Baumer (2013) reiterated that true race effects will remain evasive so long as sample selection bias remains unaddressed in studies of sentence length disparities.

MacDonald and Donnelly (2019) more recently highlighted that choice of modeling strategy may help explain any differences in race effects. Their study comparing regression, propensity score, and entropy weighting methods showed that the former two underestimated racial disparities in sentencing and that the latter (entropy weighting) was a more rigorous approach that "may lead to alternative conclusions about the size and presence of racial disparities in sentencing" (p.656). 

While these methodological concerns remain legitimate potential sources of the variation in measured race effects, a discussion of incomplete data and their potential impacts is notably absent in the sentencing literature and in criminological research more broadly. We argue the treatment of incomplete data could be a key methodological consideration that impacts the existence and magnitude of race effects. 

\subsection{Incomplete Data Treatment in Research on Race and Courts/Sentencing}

Incomplete data are often ignored or mishandled in research on race and courts/sentencing. When incomplete data are acknowledged--a relatively rare occurrence--scholars tend to remove the affected observations from the sample and perform complete case analysis, treating the missingness mechanism as non-problematic without first evaluating this assumption. Such a strategy is common, even in research published within the top criminological outlets (see Bales \& Piquero, 2012; Cassidy \& Rydberg, 2019; Kutateladze et al., 2014; Light, 2022; Zane et al., 2022). What is perhaps more common (and more insidious) are studies that fail to acknowledge incomplete data altogether. Such studies comprise the majority of published sentencing research.

Scholars will often suggest that the amount of missing data is not problematic to the degree that it would substantively change the results. Methodological experts in the field have suggested that missing data are probably "not a major problem in studies where the proportion of cases with missing data are small" (Brame \& Paternoster, 2003, p.74). They have argued, "In such situations, the departures from complete case analysis assumptions would often have to be quite large for them to have a meaningful impact on one’s results'' (p.74). Our study asserts otherwise; that even relatively small amounts of missing data can have substantive impacts on observed race effects if not treated properly. Furthermore, we suggest that incomplete data may even be a key determinant of variable race effects measured across studies and over time and is therefore in need of further evaluation. Along with their previous statement, Brame and Paternoster (2003) note that "there is some value in the development of methods that will allow researchers to investigate the robustness of their conclusions to different assumptions about missing data mechanisms" (p.76). Our study directly addresses this call and moves forward the study of race effects in criminological research.

\subsection{Current Study}

Using the Pennsylvania Commission on Sentencing’s (PCS) case-level sentencing data for individuals convicted in a Court of Common Pleas between January 1, 2010 and December 31, 2019 \((N = 864,422)\), we employ a data simulation to illustrate how patterns of incomplete data may impact differences in the likelihood of an incarceration sentence between White, non-Latino   (hereafter “White”) defendants and Black, non-Latino (hereafter “Black”) defendants. Similar methods have proven useful in other fields (Belin, 2009; Perkins et al., 2018; Sidi \& Harel, 2018). All methodological and measurement decisions (described in detail below) stem from the intention to present a clear and straightforward simulation with conventions that are familiar to criminological researchers.

\section{METHOD}

\subsection{Simulated Data Set}

Using the PCS administrative data set as a guide, we created a simulated data set of the same size \((N = 864,422)\) to act as a hypothetical population of records\footnote{The datasets generated during and/or analyzed during the current study are available from the corresponding author on reasonable request.}. Associations within the independent and control variables may confound the analysis and bias estimates (Bartlett et al. 2015). To avoid this complication, we draw from the simulated population which was generated so that only the relationships between the dependent variable (the incarceration decision) and each independent and control variable remain. For all of the models in this paper we follow the conventions of the “modal approach” to sentencing research (see Baumer, 2013), controlling for known legal and extralegal case and defendant characteristics to model the binary sentencing decision (“in/out”). These include defendant race (White = reference), gender (male = reference), linear and quadratic measures of age, linear and quadratic measures of offense severity (known in Pennsylvania as the offense gravity score [OGS]), a categorical measure of criminal history (known in Pennsylvania as the prior record score [PRS], “None” = reference category), crime type (property = reference category), recommended minimum sentence of incarceration (1 = yes), disposition type (1 = trial), and dummy variables for county and year (1 = yes). These measures have varying levels of evidence of being meaningful for modeling the sentencing decision, therefore we include as many as possible in the simulation models to create a richer data set.

To construct the simulated data set, we collected summaries for each of the independent and control variables and fit a logistic regression model to the observed PCS administrative data (a complete case analysis). Then for each case, we used the marginal probabilities for the categorical demographic and legal variables to randomly and independently select levels for each variable. For the quantitative variables, we used the sample means and variances to randomly and independently select values for each case from the normal distribution (or gamma distribution for strictly positive variables). We used the estimated coefficients from the fitted logistic regression model as the “true” parameter values for our simulated population. The predicted probability of incarceration for each simulated case is used to flip a weighted coin to obtain the simulated sentencing decision. The simulated data set has statistically independent predictor variables and causal relationships with the dependent variable determined by the observed relationships in the observed PCS data.

A logistic regression model predicting the sentencing decision given the independent and control variables was then used to illustrate the potential effects of missing data on the estimation of the race effect on incarceration between Black and White defendants. In the administrative data we estimated this effect to be 25.5\%, i.e., a Black defendant is 1.255 times more likely to be incarcerated than an otherwise identical White defendant\footnote{This estimate was not made with a causal inference procedure so it should not be treated as a valid estimate of the real race effect. Instead, this is only used as a plausible population value for our simulated data set. Within the simulated analyses, we make similarly flawed estimations, but this is in line with the majority of the criminological literature on race effects in sentencing. In addition, we do not attempt to address the missingness that is present in the Pennsylvania data, which we show can lead to flawed and unreliable estimates with our simulations.}. 

From the observed PCS administrative data we also collected the observed missing data patterns (i.e., which combinations of variables have missing values for a given defendant) and the proportions with which each pattern appears. The six observed and two unobserved missingness patterns in the PCS data are detailed in Table I. Table I shows that most defendants had complete records (96.5\%), 3\% were missing only race, and all other patterns were observed in less than 0.5\% of the observations. In addition to these eight patterns, three cases were also missing previous criminal history (PRS), but we will disregard these three cases for brevity\footnote{Each additional variable doubles the number of missing data patterns, so a fourth variable would result in 16 patterns with only 7 observed patterns.}.  The three variables we will induce missing values for in our simulation are: recommended minimum sentence of incarceration, age (linear and quadratic terms), and race, with the most common pattern with missing values being race missing on its own.

\subsection{Missingness Model}

We then built a multinomial missingness model based on the one from Perkins et al. (2018). Let \(V\) be the set of variables that are completely observed in every missing data pattern: incarceration, OGS, OGS-squared, PRS, crime type, gender, case disposition, and county. Let \(W_r\) be the the set of variables that may have missing values in some patterns but are not necessarily missing in pattern \(r\): age, age-squared, recommended minimum sentence of incarceration, and race. Statistically, there are three core assumptions that can be made about how missing data patterns are generated. Missing values can be missing completely at random (MCAR), missing at random (MAR), or missing not at random (MNAR). MCAR implies that the missing data patterns are determined independently of the data and are dispersed randomly throughout. MCAR is a typical (implicit) assumption of criminologists, who tend to remove cases with missing data without testing their mechanism. MAR data are missing dependent on the values of the observed data (e.g., age may be more likely to be missing for male defendants). MNAR data are missing due to the unobserved value of the missing variable or the value of a different unobserved factor (e.g., age may be more likely missing for younger defendants).

\begin{center}
    \textbf{Table I about here}
\end{center}

The probability that any given case belongs to a particular pattern is given by \(p_{i,r}\) for \(i = 1,\dots,n\) and \(r = 1,\dots,8\). The probabilities of membership to each missingness pattern will roughly mirror the proportions seen in the PCS data. Exceptions are made for patterns seven and eight to give them 1/864,422 chance of being selected instead of zero chance.

The multinomial model that gives the matrix of probabilities \(\mathbf{P} = \{p_{i,r}\}_{1 \leq i \leq n, 1 \leq r \leq 8}\) of having each pattern for each observation is given by Eq. \ref{eq:miss-model}.

\begin{equation} \label{eq:miss-model}
    P(R = r | V, L_r, W_r) = \frac{\exp(\alpha_r + \beta_r V + \gamma_r W_r)}{1 + \sum_{k = 1}^R\exp(\alpha_k + \beta_k V + \gamma_k W_k)}
\end{equation}

The parameters \(\gamma_r\) are set to create MAR, MNAR missingness and to induce over- or under-estimation of the race effect odds ratio. For MAR, we set the parameter(s) to zero if they correspond to the variable(s) that are missing for that pattern. For MNAR, the parameter(s) can be non-zero even when they correspond to variable(s) that are missing. To set the missing data patterns, we sample from the categorical distribution with the above probabilities for \(p_{i,r} = P(R = r | \mathbf{v}_i, \mathbf{l}_{r,i}, \mathbf{w}_{r,i})\) for \(r = 1,\dots, 8\) and \(i = 1,\dots, n\). The intercept parameters \(\alpha_r\) are determined by back calculating from the observed proportion of each missingness pattern as in (Perkins et al. 2018).

To achieve the desired effects of the missing data in the complete case analyses, we included interactions between non-incarceration and the defendant being Black \(Z_1 = I(INCAR = 0)I(RACE = BLACK)\) and the interactions between incarceration and the defendant being Black \(Z_{2} = I(INCAR = 1)I(RACE = BLACK)\). These variables are missing for the same patterns for which race is missing. To get over-estimates, we emphasize \(Z_1\). To get under-estimates, we de-emphasize \(Z_1\) and emphasize \(Z_2\).

\subsection{Analysis Methods}

\subsubsection{Complete Case Analysis}

Complete Case Analysis (CCA) is the default procedure for most statistical packages and thus used by many analysts. This ad hoc method removes all incomplete observations to create a complete rectangular data set with no missing values (White \& Carlin, 2010). For example, if we have a data set containing sentencing decision, age, sex, and prior record for 10,000 defendants and 200 defendants are missing values for prior record, then the complete case analysis would only evaluate the 9,800 cases with complete records, dropping all cases that are missing criminal history.

This can be deeply problematic when values are missing at random (van der Heijden et al., 2006). If female defendants are more likely to be missing age observations, then more female defendant cases will be dropped from the analysis potentially biasing the estimates.

\subsubsection{Multiple Imputation}

There are several methods for more rigorous treatment of incomplete data. Principal among these are data augmentation for Bayesian analysis (Tanner \& Wong, 1987), EM algorithm for certain problems (Dempster et al., 1977), and multiple imputation for a broad class of problems including our particular simulation (Rubin, 1987; Little \& Wang, 1996). Further reviews and extensions of multiple imputation have been added to the literature since (Schafer, 1999; Raghunathan et al., 2001; Schafer, 2003; Harel \& Zhou, 2007).

Multiple imputation (MI) consists of three stages: (i) imputation, (ii) analysis, and
(iii) combining (Rubin 1987). To multiply impute an incomplete data set, we take the data \(D\) and split it into observed and missing parts \(D = (D_{obs}, D_{miss})\), create \(M\) copies of \(D_{obs}\), and create \(M\) sets of imputed data \(D_{imp}^m\) to complete \(D_{miss}\) and pair with each copy of \(D_{obs}\). Our new collection of completed data sets is \(\mathbf{D}^* = \{D_1,\dots, D_M\}\) where \(D_m = (D_{obs}, D_{imp}^m)\). The same complete data analysis is then performed on each of the \(M\) completed data sets to obtain estimates \(Q_m\) and variances \(U_m\). In our case, we perform a logistic regression including all of the independent and control variables to collect the odds ratio of the race effect \(Q_m\) with \(U_m\) as its variance. These values can then be combined to produce point estimates, to perform null hypothesis significance tests, or to create confidence intervals (Rubin 1987). The point estimate is the mean of the estimates from each completed data set, \(\bar{Q} = \frac{1}{M} \sum_{m=1}^M Q_m\), with variance \(T = \bar{U} + (1 + \frac{1}{M})B\) which is a combination of the average variance of each completed data estimate \(\bar{U} = \frac{1}{M}\sum_{m=1}^M U_m\) and the variance of the estimates between themselves \(B = \frac{1}{M-1}\sum_{m=1}^M (Q_m - \bar{Q})^2\). These formulas are known as Rubin's Rules (Rubin 1987; Schafer 1999).

The multiple imputation procedure is very flexible and allows researchers to choose many different ways to produce imputations and to model what causes the values to be missing in the first place. One popular implementation is MICE, or Multiple Imputation by Chained Equations (Raghunathan et al., 2001; Azur et al., 2011), which imputes each incomplete variable in sequence based on the observed data. In our simulations we  use MICE to multiply impute the missing values with predictive mean matching in the simulated data and create our MI estimates for the race effect.

\subsection{Simulations}

In our simulation study, we are particularly interested in the impacts of research decisions like sample size and number of imputations and the impacts of varying missingness models’ like MAR and MNAR (i.e., out of the researcher’s control). Additionally, we choose the missingness model parameters to demonstrate that incomplete data can result in dramatic over- or under-estimation of the race effect. The simulation procedure is as follows.

For each type of missingness (MAR/MNAR):
\begin{enumerate}
    \item Draw a new sample of size \(n\) from the population data \(D_{POP}\) without replacement. This sampled data set is called complete, \(D_{COMP}\). This data set is used for all five analyses in this iteration with the same missing data patterns imposed for the CCA and MI analyses.
  
  \item Missing data patterns are then assigned with the model specified in Eq. \ref{eq:miss-model}. Parameters are set to determine if it is a MAR or MNAR mechanism and whether the CCA estimates will be over- or under-estimates of the race effect. This creates a triplet of data sets, \((D_{COMP}, D_{OVER}, D_{UNDR})\) for each type of missingness. The intended direction of the CCA estimate relative to the complete data estimate (over or under) is denoted by \(Dir\).\footnote{Note that while the missing values are created under MAR or MNAR models, the CCA is known to be unbiased under a MCAR mechanism and the MI procedure is known to be unbiased under a MAR mechanism (Harel \& Zhou, 2007; White \& Carlin, 2010). For our MI analyses, we assume that the missing data is MAR while CCA implicitly assumes MCAR.}
  
  \item Logistic regressions are fit on the complete sampled data set \(D_{COMP}\), the over-estimate data set \(D_{OVER}\), and the under-estimate data set \(D_{UNDR}\) with both CCA and with MI (Rubin 1987) with \(M\) imputations. This creates five analyses for each iteration: one complete data analysis on \(D_{COMP}\), two CCA analyses on \((D_{OVER}, D_{UNDR})\), and two MI analyses also on \((D_{OVER}, D_{UNDR})\) with \(M\) imputations.

  \begin{itemize}
  \item
    For each regression, the odds ratio \(OR\) and proportion of incomplete cases\footnote{The number of complete cases divided by the total number of cases.} \(p_{miss}\) are recorded and used to calculate the bias, \(b_{OR} = \hat{OR}_{Race}(D_{Dir}) - \exp(\beta_{Race})\), and difference between the incomplete data estimate and the complete data estimate, \(d_{Dir}^{Analysis} = \hat{OR}_{Race}^{Analysis}(D_{Dir}) - \hat{OR}_{Race}(D_{Comp})\). The differences \(d_{OR}\) will be our primary quantity of interest.
  \end{itemize}
  \item Repeat 1-3.
\end{enumerate}

The simulations were run with sample sizes of \(n = \) 500; 1,000; 2,500; 5,000; 10,000; 25,000; and 50,000 with \(M = \) 3; 5; 8; 25; 100 imputations for \(n < 10,000\) and \(M = \) 3; 5; 8; 25 imputations for \(n \geq 10,000\). 225 simualtion iterations were used for \(n < 10,000\) and 105 iterations were used for \(n \geq 10,000\).

To determine the number of imputations to perform, the rate of missing information can be used (Harel, 2007; Graham et al., 2007; Bodner 2008; White et al., 2011; von Hippel 2020). The rate of missing information quantifies the amount of information within the missing data compared to the (typically hypothetical) complete data set. Through another set of simulations (omitted for brevity), it was determined that there should be between five and eight imputations to achieve a relative efficiency of >0.99 using the methods described by (Rubin, 1987; Harel, 2007). For a relative efficiency of ~0.95, three imputations is sufficient. The primary comparison of interest from the simulations is between missing and complete data estimates using the same sample from the simulated population data, $d_{Dir}^{Analysis}$, where the missing data estimate is made using CCA or MI (see Appendix for detailed descriptions of the selected parameters). Following the suggestion from Bodner (2007) and White et al. (2011), our incomplete case rate of roughly 3\% would indicate we need at least 3 imputations for reliable point estimation. We include large numbers of imputations as well to illustrate the results under best practices (Harel, 2007). 

\section{RESULTS}

The simulation study results are displayed in Figures 1-4. The boxplots are ordered by increasing sample size from left-to-right for each of Figure 1-4. Points above and below the boxplots represent outliers. Figures 1,2, and 4 are subdivided into 12 sub-figures by the methodology (Complete Data or COMP, CCA, or MI with separate frames for each number of imputations) and by the missingness mechanism (MAR or MNAR) with which the missing data were imposed. Within each sub-figure, there is a boxplot for each sample size for that combination of analysis and missingness setting. For the incomplete data analyses, there are additional boxplots for simulations that are designed to induce over- or under-estimation of the race effect. Each boxplot is shaded to correspond to a sample size. Figure 3 is divided into two frames with twelve boxplots within each frame for the combinations of sample size and under- or over-estimation. The boxplots in Figure 3 display the distribution of the proportions of missingness in that combination of simulation settings.

\begin{center}
    \textbf{Figure 1 about here}
\end{center}

The COMP frame on the far left of Figure 1, and 4 represents the sampling distributions of the race effect estimate \(\hat{OR}_{Race}^{Analysis}(D_{Dir})\), and its statisitcal bias \(b_{OR} = \hat{OR}_{Race}(D_{Dir}) - \exp(\beta_{Race})\) respectively when no data are missing. The six boxplots in the frame show the sampling distribution under successively larger sample sizes (from left to right). The CCA frame shows the sampling distributions under complete case analysis once data are missing. The MI frames show the sampling distributions under multiple imputation with the given number of imputations. Figure 2 displays the sampling distributions of the difference between the incomplete data estimate and complete data estimate  \(d_{Dir}^{Analysis} = \hat{OR}_{Race}^{Analysis}(D_{Dir}) - \hat{OR}_{Race}(D_{Comp})\). Figure 2 is structured similarly to Figures 1 and 4 but lacks the COMP frame as \(d^{Complete}_{Dir} = 0\) for all simulation iterations and settings.

\begin{center}
    \textbf{Figure 2 about here}
\end{center}

In our simulations we assume that the missing data mechanism is MAR and ignorable for the MI analysis, so we omit any modeling of the mechanism in our imputations. As always, we implicitly assume MCAR for the CCA analysis. Estimates evaluated with MI under the MAR generated data are unbiased and closely resemble the complete data estimates demonstrating that the MI procedure is able to recover most of the information lost when removing incomplete cases with CCA. While not addressing the missingness model is appropriate for the MAR generated data, it fails to overcome the bias induced by the MNAR generated data resulting in only slightly mitigated biases in the MI estimates with MNAR data. More complex imputation models for non-ignorable models (MNAR) are available but are not the topic of this manuscript.

According to the simulations, the median difference between the CCA estimates and the complete data estimates gets slightly closer to zero as the sample size increases from 500 cases to 50,000 cases. However the spread of the differences shrinks as well and still indicates that there is enough bias even for large samples to still flip the direction of the race effect. Increasing the number of imputations from three to eight or even 100 does not result in dramatically different results. The increase in relative statistical efficiency is fairly small, going from roughly 95\% to 99+\% as imputations increase from three to eight, so the impact is negligible. The increase in efficiency from eight imputations to 100 imputations is very small as both provide efficiencies well over 100\%. For our simulations, a large number of imputations does not seem to be required, but this must be tested in practice for each data set (Harel, 2007).

\begin{center}
    \textbf{Figure 3 about here}
\end{center}

Specifically looking at Figure 2, we can see that the average differences from the complete data estimate are almost certainly non-zero for the CCA estimates under both MAR and MNAR while they tend to zero for MI with all numbers of imputations under MAR. Under MNAR, the MI estimates’ differences also tend to be non-zero. This is increasingly true for large sample sizes as well. We can also see that the settings of the missingness mechanism can induce biases in the direction of the estimate. From Table II, when inducing under-estimation, the average CCA estimate is less than 1 for all MNAR simulations and all MAR simulations with sample sizes greater than n = 500 (the average OR for MAR CCA with n = 500 was 1.007). An estimate less than 1 would mean that the race effect is flipped and Black defendants are less likely to be incarcerated than White defendants that are in otherwise similar legal situations and have similar demographics. When considering the 95\% confidence interval from the sampling distribution, the CCA estimates under MNAR fail to include 1 for sample sizes of \(n \geq 5000\), so that we will nearly always find that the race effect is flipped from what is true in our simulated population.

Depending on the missingness model parameters, between 5-10\% of cases are incomplete on average for each simulated sample regardless of sample size as shown in Figure 3. While this is slightly higher than the proportion of incomplete cases in the observed data, it still seems to be a plausible amount of missingness (Bales \& Piquero, 2012; Kutateladze et al., 2014; Cassidy \& Rydberg, 2020). Even this relatively small amount of missing values produces biases in the complete case analysis estimates under both MAR and MNAR missingness mechanisms. We have chosen the parameters for the missingness model to flip the direction of the race effect from the true underlying effect when estimating with CCA. As a result, the estimates made with complete case analysis are unreliable and do not provide viable estimates of the race effect.

\begin{center}
    \textbf{Table II about here}
\end{center}

Statistical bias refers to the expected difference between the estimate and the true population parameter value and is useful as an estimate suitability measure. In general, we want estimates to be unbiased so that we can accurately assess the true value of the race effect. From our simulation results, there is considerable bias in the CCA estimates under MAR and MNAR mechanisms and in the MI estimates under the MNAR mechanism as seen in Figure 4. The bias is corrected for when we use multiple imputation with an appropriate number of imputations under the MAR mechanism (Allison, 2000; Harel, 2007; Graham et al., 2007). Multiple imputation can also be modified to produce unbiased estimates under MNAR mechanisms by including some model for the missingness mechanism (Little 1993; Glynn et al., 1993; Little \& Wang, 1996; Bartlett et al., 2015). Alternative tools to MI have been developed to address MNAR or non-ignorable missing data that do not rely on multiple imputation (Ibrahim, 1990; Vach \& Schumacher, 1993; Little \& Wang, 1996), though those tools are beyond the scope of this paper.

\begin{center}
    \textbf{Figure 4 about here}
\end{center}

From Figures 1-4, it is apparent that as the sample size increases, the variances of the estimated odds ratio, the differences, and the bias all decrease as expected from statistical theory. While this is typically useful, as evidenced by the complete data analyses, where the estimate converges to the true odds ratio and the bias converges to zero, the decreasing variance can result in greater (unfounded) confidence in unreliable inference when using CCA on incomplete data. When we have a small sample, there’s enough variance in the estimates, that the true value may still be captured by the range of the sampling distribution, visualized here as boxplots, while for large samples, the low variance results in narrow sampling distribution ranges that do not contain the true value. In a hypothesis testing context this would mean we have low power with small samples and high(er) power with large(r) samples, but we are detecting false positives (where the null hypothesis is the odds ratio being the true population value). 

The simulations demonstrate that regardless of sample size and with less than 10\% of cases being incomplete, complete case analysis (CCA) can still produce biased estimates for the race effect’s odds ratio. When MAR conditions are satisfied, the estimates of the odds ratio calculated under multiple imputation are unbiased as is also true when estimating the odds ratio with complete data providing suitable estimation of the race effect. These results largely hold for all sample sizes and numbers of imputations that we decided to investigate, including intermediary cases not included here for brevity, such as M = 5 imputations.

\section{DISCUSSION}

Through this simulation study we have demonstrated that the default incomplete data method in the criminological literature, complete case analysis, fails to provide unbiased estimates of the race effect in particular, and potentially any effect of interest in general. Removing incomplete cases from the analysis results in misleading inferences of the effects of interest. Depending on the unknown missingness mechanism, these inferences can overstate the race effect or, to the contrary, make it disappear entirely. The problems with complete cases analysis are remedied when analyzing data with multiple imputation if the MAR assumption holds. In practice, this requires subject matter expertise to gauge potential causes of incomplete data to assess the plausibility of the MAR assumption and use of as many relevant data points as possible to make the assumption more statistically plausible. 

Since it is impossible to distinguish between MAR and MNAR with observed data alone, sensitivity analyses should be performed to assess the effects of different assumptions and research decisions on the final inferential outcome (van Buuren et al., 1999; Siddique et al., 2012; Cro et al., 2020). Sensitivity analyses typically consist of developing a missing data mechanism model within the imputation procedure that can be adjusted through its parameters to reflect different mechanisms (MAR/MNAR) (Cro et al., 2020).

In sum, our analyses support that the race effect so often observed in sentencing and criminological research may vary, in part, because the field does not consistently identify nor properly address incomplete data. This should be added to the growing list of key methodological concerns as described by Baumer (2013), Bushway et al. (2007), Lynch (2019), MacDonald and Donnelly (2019), Mitchell (2005), Spohn (2015), Ulmer (2012), and others. Furthermore, the solutions offered here could also be extended to address sample selection bias (i.e., another form of non-ignorable missingness), which remains a contested issue particularly in research on sentence length disparities.  

\section{CONCLUSION}

Race effects persist in criminological research, whereby Black individuals tend to receive less desirable outcomes in policing, court, and correctional contexts (as compared to their White counterparts). The magnitude of effects, however, tends to vary within and between studies for several theoretical and methodological reasons. Our paper contributed to the criminological literature another potential explanation for this variation: the treatment of incomplete data.

Incomplete data are common in criminological research and, depending on the missingness mechanism, even small amounts can change the existence or magnitude of observed race effects. To date, the field has failed to address incomplete data in a consistent and appropriate manner, the potential impact of which was observed here in terms of varied race effects on the likelihood of an incarceration sentence.

The differences in race effects that stem from incomplete data is problematic not only in a statistical sense but in a practical one. It is best practice that criminal justice policies are rooted in findings from criminological research and racial impact statements that use administrative data such as those used here. Once implemented, policies and practices can cost taxpayers millions of dollars, remain in place for decades, and impact the thousands of individuals who cycle through the criminal legal system each year. Any attempt to measure race effects should be based on solid methodological grounds such as those summarized here. This would only increase stakeholders’ confidence in policy evaluations and increase the likelihood that policy impacts are as anticipated. Correctional population forecasting by race (a more recently endorsed best practice according to the Model Penal Code: Sentencing, see Reitz \& Klingele, 2019) is one policy area where the issues and solutions presented here are particularly relevant.

Future criminological research should make the identification, treatment, and reporting of incomplete data and their associated assumptions a priority. While missing data mechanisms can only truly be known in a data simulation such as this, researchers should report on the assumed mechanism and justify their assumption (e.g., a test for MCAR was passed/failed, or there's a compelling subject-matter reason to think data are MNAR). In line with recent calls for more open science in criminology and a reduction of questionable research practices (see Chin et al., 2021), scholars should request this information when reviewing manuscripts. These combined efforts will help criminologists to build a trustworthy evidence base regarding racial disparities in the U.S. criminal legal system.

\section{REFERENCES}

Albonetti, C. A. (1991). An integration of theories to explain judicial discretion. Social Problems. 38: 247--266.

Allison, P.D. (2000). Multiple imputation for missing data: A cautionary tale. Sociological Methods \& Research. 28: 301--309. https://doi.org/10.1177/0049124100028003003

Azur, M.J., Stuart, E.A., Frangakis, C., \& Leaf, P.J. (2011). Multiple imputation by chained equations: What is it and how does it work? International Journal of Methods in Psychiatric Research. 20: 40--49. https://doi.org/10.1002/mpr.329

Bales, W.D., \& Piquero, A.R. (2012). Racial/ethnic differentials in sentencing to incarceration. Justice Quarterly. 29: 742--773. https://doi.org/10.1080/07418825.2012.659674

Bartlett, J.W., Harel, O., \& Carpenter, J.R. (2015). Asymptotically unbiased estimation of exposure odds ratios in complete records logistic regression. American Journal of Epidemiology. 182: 730--736. https://doi.org/10.1093/aje/kwv114

Belin, T. R. (2009). Missing data: What a little can do, and what researchers can do in response. American Journal of Ophthalmology. 148: 820--822.

Blumer, H. (1958). Race prejudice as a sense of group position. The Pacific Sociological Review. 1: 3--7.

Bodner, T.E. (2008). What improves with increased missing data imputations? Structural Equation Modeling: A Multidisciplinary Journal 15: 651–675. https://doi.org/10.1080/10705510802339072 

Bushway, S., Johnson, B. D., \& Slocum, L. A. (2007). Is the magic still there? The use of the Heckman two-step correction for selection bias in criminology. Journal of Quantitative Criminology. 23: 151--178.

Bushway, S. D., \& Piehl, A. M. (2001). Judging judicial discretion: Legal factors and racial discrimination in sentencing. Law and Society Review. 35: 733--764.

Cassidy, M., \& Rydberg, J. (2020). Does sentence type and length matter?: Interactions of age, race, ethnicity, and gender on jail and prison sentences. Criminal Justice and Behavior. 47: 61--79.

Chin, J. M., Pickett, J. T., Vazire, S., \& Holcombe, A. O. (2021). Questionable research practices and open science in quantitative criminology. Journal of Quantitative Criminology. 1--31.

Cro S., Morris T.P., Kenward M.G., Carpenter, J.R. (2020). Sensitivity analysis for clinical trials with missing continuous outcome data using controlled multiple imputation: A practical guide. Statistics in Medicine. 39: 2815–2842. https://doi.org/10.1002/sim.8569 

Dempster, A.P., Laird, N.M., \& Rubin, D.B. (1977). Maximum likelihood from incomplete data via the EM algorithm. Journal of the Royal Statistical Society Series B (Methodological). 39: 1--38.

Doerner, J. K., \& Demuth, S. (2010). The independent and joint effects of race/ethnicity, gender, and age on sentencing outcomes in U.S. federal courts. Justice Quarterly. 27: 1--27.

Feldmeyer, B., Warren, P. Y., Siennick, S. E., \& Neptune, M. (2015). Racial, ethnic, and immigrant threat: Is there a new criminal threat on state sentencing? Journal of Research in Crime and Delinquency. 52: 62--92.

Franklin, T. W., \& Henry, T. K. S. (2020). Racial disparities in federal sentencing outcomes: Clarifying the role of criminal history. Crime \& Delinquency. 66: 3--32.

Gaston, S. (2019). Producing race disparities: A study of drug arrests across place and race. Criminology. 57: 424--451.

Glynn, R.J., Laird, N.M., Rubin, D.B. (1993). Multiple Imputation in mixture models for nonignorable nonresponse with follow-ups. Journal of the American Statistical Association 88: 984–993. https://doi.org/10.2307/2290790 

Graham, J.W., Olchowski, A.E., \& Gilreath, T.D. (2007). How many imputations are really needed? Some practical clarifications of multiple imputation theory. Prevention Science. 8: 206--213. https://doi.org/10.1007/s11121-007-0070-9

Green, E. (1961). Judicial Attitudes in Sentencing: A Study of the Factors Underlying the Sentencing Practice of the Criminal Court of Philadelphia, London, Macmillan.

Green, E. (1964). Inter-and intra-racial crime relative to sentencing. The Journal of Criminal Law, Criminology, and Police Science. 55: 348--358.

Harel, O. (2007). Inferences on missing information under multiple imputation and two-stage multiple imputation. Statistical Methodology. 4: 75--89. https://doi.org/10.1016/j.stamet.2006.03.002

Harel, O., \& Zhou, X. (2007). Multiple imputation: Review of theory, implementation and software. Statistics in Medicine. 26: 3057--3077. https://doi.org/10.1002/sim.2787

Heckman, J. J. (1976). The common structure of statistical models of truncation, sample selection and limited dependent variables and a simple estimator for such models. In Annals of Economic and Social Measurement, NBER, pp. 475--492.

Hepburn, J. R. (1978). Race and the decision to arrest: An analysis of warrants issued. Journal of Research in Crime and Delinquency. 15: 54--73.

Hester, R., \& Sevigny, E. L. (2016). Court communities in local context: A multilevel analysis of felony sentencing in South Carolina. Journal of Crime and Justice. 39: 55--74.

Hickert, A., Bushway, S. D., Harding, D. J., \& Morenoff, J. D. (2022). Prior punishments and cumulative disadvantage: How supervision status impacts prison sentences. Criminology. 60: 27--59.

Ibrahim, J.G. (1990). Incomplete data in generalized linear models. Journal of the American Statistical Association. 85: 765--769. https://doi.org/10.2307/2290013

Johnson, E. H. (1957). Selective factors in capital punishment. Social Forces. 165--169.

Johnson, B. D., \& Lee, J. G. (2013). Racial disparity under sentencing guidelines: A survey of recent research and emerging perspectives. Sociology Compass. 7: 503--514.

Jordan, K. L., \& McNeal, B. A. (2016). Juvenile penalty or leniency: Sentencing of juveniles in the criminal justice system. Law and Human Behavior. 40: 387--400.

King, R. D., \& Light, M. T. (2019). Have racial and ethnic disparities in sentencing declined? Crime and Justice. 48: 365--437.

Kurlychek, M. C., \& Johnson, B. D. (2019). Cumulative disadvantage in the American criminal justice system. Annual Review of Criminology. 2: 291--319.

Kutateladze, B. L., Andiloro, N. R., \& Johnson, B. D. (2016). Opening Pandora’s box: How does defendant race influence plea bargaining? Justice Quarterly. 33: 398--426.

Kutateladze, B.L., Andiloro, N.R., Johnson, B.D., \& Spohn, C.C. (2014). Cumulative disadvantage: Examining racial and ethnic disparity in prosecution and sentencing. Criminology. 52: 514--551. https://doi.org/10.1111/1745-9125.12047

Light, M. T. (2022). The declining significance of race in criminal sentencing: Evidence from U.S. federal courts. Social Forces. 100: 1110--1141.

Little, R.J.A. (1993). Pattern-mixture models for multivariate incomplete data. Journal of the American Statistical Association. 88: 125–134. https://doi.org/10.2307/2290705 

Little, R.J.A., \& Wang, Y. (1996). Pattern-mixture models for multivariate incomplete data with covariates. Biometrics. 52: 98--111. https://doi.org/10.2307/2533148

Lynch, M. (2019). Focally concerned about focal concerns: A conceptual and methodological critique of sentencing disparities research. Justice Quarterly. 36: 1148--1175.

Martinez, B. P., Petersen, N., \& Omori, M. (2020). Time, money, and punishment: Institutional racial-ethnic inequalities in pretrial detention and case outcomes. Crime \& Delinquency. 66: 837--863.

Mears, D. P., Brown, J. M., Cochran, J. C., \& Siennick, S. E. (2021). Extended solitary confinement for managing prison systems: Placement disparities and their implications. Justice Quarterly. 38: 1492--1518.

Metcalfe, C., \& Chiricos, T. (2018). Race, plea, and charge reduction: An assessment of racial disparities in the plea process. Justice Quarterly. 35: 223--253.

Mitchell, O. (2005). A meta-analysis of race and sentencing research: Explaining the inconsistencies. Journal of Quantitative Criminology. 21: 439--466.

Mitchell, O., Yan, S., \& Mora, D. O. (2022). Trends in prison sentences and racial disparities: 20-years of sentencing under Florida’s Criminal Punishment Code. Journal of Research in Crime and Delinquency. 1--39.

Omori, M., \& Johnson, O. (2019). Racial inequality in punishment. In Oxford Research Encyclopedia of Criminology and Criminal Justice.

Omori, M., \& Petersen, N. (2020). Institutionalizing inequality in the courts: Decomposing racial and ethnic disparities in detention, conviction, and sentencing. Criminology. 58: 678--713.

Owusu-Bempah, A., \& Luscombe, A. (2021). Race, cannabis and the Canadian war on drugs: An examination of cannabis arrest data by race in five cities. International Journal of Drug Policy. 91: 102937.

Perkins, N.J., Cole, S.R., \& Harel, O. (2018). Principled approaches to missing data in epidemiologic studies. American Journal of Epidemiology. 187: 568--575. https://doi.org/10.1093/aje/kwx348

Pratt, T. C. (1998). Race and sentencing: A meta-analysis of conflicting empirical research results. Journal of Criminal Justice. 26: 513--523.

Raghunathan, T.E., Lepkowski, J.M., Hoewyk, J.V., \& Solenberger, P. (2001). A multivariate technique for multiply imputing missing values using a sequence of regression models. Statistics Canada. 27: 85--95.

Rehavi, M. M., \& Starr, S. B. (2014). Racial disparity in federal criminal sentences. Journal of Political Economy. 122: 1320--1354.

Reitz, K. R., \& Klingele, C. M. (2019). Model Penal Code: Sentencing—Workable limits on mass punishment. Crime and Justice. 48: 255--311.

Ridgeway, G., Moyer, R. A., \& Bushway, S. D. (2020). Sentencing scorecards: Reducing racial disparities in prison sentences at their source. Criminology \& Public Policy. 19: 1113--1138.

Rubin, D.B. (1987). Multiple Imputation for Nonresponse in Surveys, Wiley, New York.

Schafer, J.L. (1999). Multiple imputation: A primer. Statistical Methods in Medical Research. 8: 3--15. https://doi.org/10.1177/096228029900800102

Schafer, J.L. (2003). Multiple imputation in multivariate problems when the imputation and analysis models differ. Statistica Neerlandica. 57: 19--35. https://doi.org/10.1111/1467-9574.00218

Schlesinger, T. (2008). The cumulative effects of racial disparities in criminal processing. Advocate. 13: 22--34.

Siddique, J., Harel, O., \& Crespi, C.M. (2012). Addressing missing data mechanism uncertainty using multiple-model multiple imputation: Application to a longitudinal clinical trial. Ann Appl Stat. 6: 1814–1837. https://doi.org/10.1214/12-AOAS555 

Sidi, Y., \& Harel, O. (2018). The treatment of incomplete data: Reporting, analysis, reproducibility, and replicability. Social Science \& Medicine. 209: 169--173.

Skeem, J. L., \& Lowenkamp, C. T. (2016). Risk, race, and recidivism: Predictive bias and disparate impact. Criminology. 54: 680--712.

Snowball, L., \& Weatherburn, D. (2007). Does racial bias in sentencing contribute to Indigenous overrepresentation in prison? Australian \& New Zealand Journal of Criminology. 40: 272--290.

Spohn, C. (2000). Thirty years of sentencing reform: The quest for a racially neutral sentencing process. In J. Horney (Ed.), Policies, Processes, and Decisions of the Criminal Justice System, Washington, DC: National Institute of Justice, pp. 427--501.

Spohn, C. (2015). Evolution of sentencing research. Criminology \& Public Policy. 14: 225--232.

Spohn, C., \& Holleran, D. (2000). The imprisonment penalty paid by young, unemployed black and Hispanic male offenders. Criminology. 38: 281--306.

Steffensmeier, D., Ulmer, J., \& Kramer, J. (1998). The interaction of race, gender, and age in criminal sentencing: The punishment cost of being young, black, and male. Criminology. 36: 763--798.

Steinmetz, K. F., \& Henderson, H. (2016). Inequality on probation: An examination of differential probation outcomes. Journal of Ethnicity in Criminal Justice. 14: 1--20.

Stevens, T., \& Morash, M. (2015). Racial/ethnic disparities in boys’ probability of arrest and court actions in 1980 and 2000: The disproportionate impact of “getting tough” on crime. Youth Violence and Juvenile Justice. 13: 77--95.

Tanner, M.A., \& Wong, W.H. (1987). The calculation of posterior distributions by data augmentation. Journal of the American Statistical Association. 82: 528--540. https://doi.org/10.2307/2289457

Ulmer, J. T., \& Johnson, B. (2004). Sentencing in context: A multilevel analysis. Criminology. 42: 137--178.

Ulmer, J. T., Light, M., Kramer, J., \& Eisenstein, J. (2011). Does increased judicial discretion lead to increased disparity? The “liberation” of judicial sentencing discretion in the wake of the Booker/Fanfan decision. Justice Quarterly. 28: 799--837.

Vach, W., \& Schumacher, M. (1993). Logistic regression with incompletely observed categorical covariates: A comparison of three approaches. Biometrika. 80: 353--362. https://doi.org/10.2307/2337205 

van Buuren, S., Boshuizen, H.C., \& Knook, D.L. (1999). Multiple imputation of missing blood pressure covariates in survival analysis. Statistics in Medicine 18:681–694.

van der Heijden, G.J., Donders, A.R., Stijnen, T., \& Moons, K.G. (2006). Imputation of missing values is superior to complete case analysis and the missing-indicator method in multivariable diagnostic research: A clinical example. Journal of Clinical Epidemiology. 59: 1102--1109. https://doi.org/10.1016/j.jclinepi.2006.01.015

von Hippel, P.T. (2020). How many imputations do you need? A two-stage calculation using a quadratic rule. Sociological Methods \& Research. 49: 699–718. https://doi.org/10.1177/0049124117747303 

White, I.R., \& Carlin, J.B. (2010). Bias and efficiency of multiple imputation compared with complete-case analysis for missing covariate values. Statistics in Medicine. 29: 2920--2931. https://doi.org/10.1002/sim.3944

White, I.R., Royston, P., \& Wood, A.M. (2011). Multiple imputation using chained equations: Issues and guidance for practice. Statistics in Medicine. 30: 377–399. https://doi.org/10.1002/sim.4067 

Wooldredge, J. (2012). Distinguishing race effects on pretrial release and sentencing decisions. Justice Quarterly. 29: 41--75.

Wooldredge, J., Frank, J., Goulette, N., \& Travis, L. III (2015). Is the impact of cumulative disadvantage on sentencing greater for Black defendants? Criminology \& Public Policy. 14: 187--223.

Wu, J. (2016). Racial/ethnic discrimination and prosecution: A meta-analysis. Criminal Justice and Behavior. 43: 437--458.

\newpage
\section{APPENDIX}

\subsection{Missingness Model Parameters}

The multinomial model uses two variables to determine the probability of an observation being missing: incarceration and whether the defendant’s race is Black. Two interaction variables were created for non-incarceration and race being Black and being incarcerated and race being Black. The intercept of the model is set for each of the eight missing data patterns to reflect the observed probability of that pattern in the Pennsylvania data. The parameters of the model vary by pattern and to induce over- or under-estimation of the race effect in the complete case analysis. See Table I for a list of the missing data patterns in the PCS data.

\subsubsection{MNAR Missingness}

The missingness model assigns slightly more missingness than we saw in the administrative data while still having a relatively low number of incomplete cases.

\begin{equation}\label{eq:miss-model-4par}
    P(Pattern ~ r ~ for ~ Subject ~ i) = \frac{\exp\left(\alpha_r + \beta_{1r} INCAR_i + \beta_{2r} Black_i + \beta_{3r} Z_{1i} + \beta_{4r} Z_{2i}\right)}{1 + \sum_{r=1}^8 \exp\left(\alpha_r + \beta_{1r} INCAR_i + \beta_{2r} Black_i + \beta_{3r} Z_{1i} + \beta_{4r} Z_{2i}\right)}
\end{equation}

\begin{center}
    \textbf{Table III about here}
\end{center}

To over-estimate the race effect, missingness in race is conditional on race being Black and incarceration so we set the parameters of the model to be \(\exp(\beta) = (10, 0.1, 0.001, 100)^T \in \mathbb{R}^4\). The first element in $\beta$ emphasizes missingness based on incarceration, the second de-emphasizes missingness based on being Black while the third element corresponding to \(Z_1\)greatly de-emphasizes missingness for defendants who aren’t incarcerated and are Black and the fourth corresponding to \(Z_2\) greatly emphasizes missingness for individuals who are incarcerated and are Black.

\begin{equation}\label{eq:miss-model-3par}
    P(Pattern ~ r ~ for ~ Subject ~ i) = \frac{\exp\left(\alpha_r + \beta_{1r} INCAR_i + \beta_{2r} Black_i + \beta_{3r} Z_{1i}\right)}{1 + \sum_{r=1}^8 \exp\left(\alpha_r + \beta_{1r} INCAR_i + \beta_{2r} Black_i + \beta_{3r} Z_{1i}\right)}
\end{equation}

\begin{center}
    \textbf{Table IV about here}
\end{center}

To under-estimate the effect, missingness in race is again conditional on race being Black and incarceration, so we set the parameters of the model to be \(\exp(\beta) = (5,5,10)^T \in \mathbb{R}^3\). The first element in \(\beta\) emphasizes missingness based on incarceration, the second element emphasizes missingness based on being Black while the third element greatly emphasizes missingness for defendants who aren’t incarcerated and are Black corresponding to \(Z_1\).

\subsubsection{MAR Missingness}

We used similar parameter settings for the MAR simulations as we did for the MNAR simulations, but now missingness based on race is only for patterns 1,3,5, and 7 which have no missing variables, missing in age, missing in recommended minimum, and missing in both age and recommended minimum. This maintains the MAR assumption since the missingness is not contingent on race for the patterns where race is missing. However, since we are performing the analyses with complete case analysis, we are still systematically excluding data from our analysis in such a way that will dramatically bias the results based on the values of race and incarceration.

Slightly different parameter values are used for patterns 3 and 5 compared to patterns 1 and 7. See Table I for a list of the missing data patterns. We used Eq. \ref{eq:miss-model-3par} for the over-estimation in the MAR context. 

\begin{center}
    \textbf{Table V about here}
\end{center}

To get an over-estimated race effect estimate with CCA, I de-emphasize missingness based on \(Z_1\) for patterns where race is observed.

\begin{center}
    \textbf{Table VI about here}
\end{center}

\begin{equation}\label{eq:miss-model-2par}
    P(Pattern ~ r ~ for ~ Subject ~ i) = \frac{\exp\left(\alpha_r + \beta_{1r} Z_{1i} + \beta_{2r} Z_{2i}\right)}{1 + \sum_{r=1}^8 \exp\left(\alpha_r + \beta_{1r} Z_{1i} + \beta_{2r} Z_{2i}\right)}
\end{equation}

To under-estimate the effect with CCA, we emphasize \(Z_1\) and de-emphasize \(Z_2\) for patterns where race is observed.

\end{document}