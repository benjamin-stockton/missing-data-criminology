% Options for packages loaded elsewhere
\PassOptionsToPackage{unicode}{hyperref}
\PassOptionsToPackage{hyphens}{url}
\PassOptionsToPackage{dvipsnames,svgnames,x11names}{xcolor}
%
\documentclass[
  letterpaper,
  DIV=11,
  numbers=noendperiod]{scrartcl}

\usepackage{amsmath,amssymb}
\usepackage{lmodern}
\usepackage{iftex}
\ifPDFTeX
  \usepackage[T1]{fontenc}
  \usepackage[utf8]{inputenc}
  \usepackage{textcomp} % provide euro and other symbols
\else % if luatex or xetex
  \usepackage{unicode-math}
  \defaultfontfeatures{Scale=MatchLowercase}
  \defaultfontfeatures[\rmfamily]{Ligatures=TeX,Scale=1}
\fi
% Use upquote if available, for straight quotes in verbatim environments
\IfFileExists{upquote.sty}{\usepackage{upquote}}{}
\IfFileExists{microtype.sty}{% use microtype if available
  \usepackage[]{microtype}
  \UseMicrotypeSet[protrusion]{basicmath} % disable protrusion for tt fonts
}{}
\makeatletter
\@ifundefined{KOMAClassName}{% if non-KOMA class
  \IfFileExists{parskip.sty}{%
    \usepackage{parskip}
  }{% else
    \setlength{\parindent}{0pt}
    \setlength{\parskip}{6pt plus 2pt minus 1pt}}
}{% if KOMA class
  \KOMAoptions{parskip=half}}
\makeatother
\usepackage{xcolor}
\usepackage[top=30mm,left=30mm]{geometry}
\setlength{\emergencystretch}{3em} % prevent overfull lines
\setcounter{secnumdepth}{5}
% Make \paragraph and \subparagraph free-standing
\ifx\paragraph\undefined\else
  \let\oldparagraph\paragraph
  \renewcommand{\paragraph}[1]{\oldparagraph{#1}\mbox{}}
\fi
\ifx\subparagraph\undefined\else
  \let\oldsubparagraph\subparagraph
  \renewcommand{\subparagraph}[1]{\oldsubparagraph{#1}\mbox{}}
\fi


\providecommand{\tightlist}{%
  \setlength{\itemsep}{0pt}\setlength{\parskip}{0pt}}\usepackage{longtable,booktabs,array}
\usepackage{calc} % for calculating minipage widths
% Correct order of tables after \paragraph or \subparagraph
\usepackage{etoolbox}
\makeatletter
\patchcmd\longtable{\par}{\if@noskipsec\mbox{}\fi\par}{}{}
\makeatother
% Allow footnotes in longtable head/foot
\IfFileExists{footnotehyper.sty}{\usepackage{footnotehyper}}{\usepackage{footnote}}
\makesavenoteenv{longtable}
\usepackage{graphicx}
\makeatletter
\def\maxwidth{\ifdim\Gin@nat@width>\linewidth\linewidth\else\Gin@nat@width\fi}
\def\maxheight{\ifdim\Gin@nat@height>\textheight\textheight\else\Gin@nat@height\fi}
\makeatother
% Scale images if necessary, so that they will not overflow the page
% margins by default, and it is still possible to overwrite the defaults
% using explicit options in \includegraphics[width, height, ...]{}
\setkeys{Gin}{width=\maxwidth,height=\maxheight,keepaspectratio}
% Set default figure placement to htbp
\makeatletter
\def\fps@figure{htbp}
\makeatother
\newlength{\cslhangindent}
\setlength{\cslhangindent}{1.5em}
\newlength{\csllabelwidth}
\setlength{\csllabelwidth}{3em}
\newlength{\cslentryspacingunit} % times entry-spacing
\setlength{\cslentryspacingunit}{\parskip}
\newenvironment{CSLReferences}[2] % #1 hanging-ident, #2 entry spacing
 {% don't indent paragraphs
  \setlength{\parindent}{0pt}
  % turn on hanging indent if param 1 is 1
  \ifodd #1
  \let\oldpar\par
  \def\par{\hangindent=\cslhangindent\oldpar}
  \fi
  % set entry spacing
  \setlength{\parskip}{#2\cslentryspacingunit}
 }%
 {}
\usepackage{calc}
\newcommand{\CSLBlock}[1]{#1\hfill\break}
\newcommand{\CSLLeftMargin}[1]{\parbox[t]{\csllabelwidth}{#1}}
\newcommand{\CSLRightInline}[1]{\parbox[t]{\linewidth - \csllabelwidth}{#1}\break}
\newcommand{\CSLIndent}[1]{\hspace{\cslhangindent}#1}

\KOMAoption{captions}{tablesignature}
\makeatletter
\makeatother
\makeatletter
\makeatother
\makeatletter
\@ifpackageloaded{caption}{}{\usepackage{caption}}
\AtBeginDocument{%
\ifdefined\contentsname
  \renewcommand*\contentsname{Table of contents}
\else
  \newcommand\contentsname{Table of contents}
\fi
\ifdefined\listfigurename
  \renewcommand*\listfigurename{List of Figures}
\else
  \newcommand\listfigurename{List of Figures}
\fi
\ifdefined\listtablename
  \renewcommand*\listtablename{List of Tables}
\else
  \newcommand\listtablename{List of Tables}
\fi
\ifdefined\figurename
  \renewcommand*\figurename{Figure}
\else
  \newcommand\figurename{Figure}
\fi
\ifdefined\tablename
  \renewcommand*\tablename{Table}
\else
  \newcommand\tablename{Table}
\fi
}
\@ifpackageloaded{float}{}{\usepackage{float}}
\floatstyle{ruled}
\@ifundefined{c@chapter}{\newfloat{codelisting}{h}{lop}}{\newfloat{codelisting}{h}{lop}[chapter]}
\floatname{codelisting}{Listing}
\newcommand*\listoflistings{\listof{codelisting}{List of Listings}}
\makeatother
\makeatletter
\@ifpackageloaded{caption}{}{\usepackage{caption}}
\@ifpackageloaded{subcaption}{}{\usepackage{subcaption}}
\makeatother
\makeatletter
\@ifpackageloaded{tcolorbox}{}{\usepackage[many]{tcolorbox}}
\makeatother
\makeatletter
\@ifundefined{shadecolor}{\definecolor{shadecolor}{rgb}{.97, .97, .97}}
\makeatother
\makeatletter
\makeatother
\ifLuaTeX
  \usepackage{selnolig}  % disable illegal ligatures
\fi
\IfFileExists{bookmark.sty}{\usepackage{bookmark}}{\usepackage{hyperref}}
\IfFileExists{xurl.sty}{\usepackage{xurl}}{} % add URL line breaks if available
\urlstyle{same} % disable monospaced font for URLs
\hypersetup{
  pdftitle={Missing Data in Criminology Research},
  pdfauthor={Ben Stockton},
  colorlinks=true,
  linkcolor={blue},
  filecolor={Maroon},
  citecolor={Blue},
  urlcolor={Blue},
  pdfcreator={LaTeX via pandoc}}

\title{Missing Data in Criminology Research}
\author{Ben Stockton}
\date{}

\begin{document}
\maketitle
\ifdefined\Shaded\renewenvironment{Shaded}{\begin{tcolorbox}[interior hidden, boxrule=0pt, borderline west={3pt}{0pt}{shadecolor}, sharp corners, frame hidden, breakable, enhanced]}{\end{tcolorbox}}\fi

\renewcommand*\contentsname{Table of contents}
{
\hypersetup{linkcolor=}
\setcounter{tocdepth}{3}
\tableofcontents
}
// I am assuming that Clare will be writing much of the introduction and
background.

\hypertarget{sec-intro}{%
\section{INTRODUCTION}\label{sec-intro}}

\hypertarget{sec-background}{%
\section{BACKGROUND}\label{sec-background}}

\hypertarget{sec-missing-data-crim-research}{%
\subsection{Missing Data in Criminological
Research}\label{sec-missing-data-crim-research}}

// Put here as a placeholder. Researchers often use administrative data
to investigate various factors' effects on incarceration decisions.
Often administrative data are incomplete with some unobserved
demographic or legal features for some cases or subjects. While the
number of incomplete cases is often seemingly small, even a small amount
of missing data can have an important effect on the results of an
analysis. The size and way this effect manifests is contingent on the
way the incomplete cases are induced or created and the analysis
performed. We will demonstrate that under some conditions, we can
systematically over- or under-estimate the race effect, even flipping
the direction of the effect when data are incomplete. For our
illustrative simulation, we will be measuring the race effect as the
odds ratio of the Black to White defendant's incarceration decision as
estimated by a logistic regression. Administrative data from the
Pennsylvania criminal justice system (2010-2019) were used as a basis
for a simulated population of cases. We also demonstrate that analysts
can account for the missing values to obtain the desired, unbiased
estimates of the race effect using multiple imputation (Rubin 1987).

\hypertarget{sec-race-effect-sentencing}{%
\subsection{Variations in the ``Race/Ethnicity'' Effect in
Sentencing/Corrections}\label{sec-race-effect-sentencing}}

\hypertarget{sec-current-study}{%
\subsection{Current Study}\label{sec-current-study}}

// Include description of data source - Clare

\hypertarget{sec-method}{%
\section{METHOD}\label{sec-method}}

\hypertarget{sec-sim-data}{%
\subsection{Simulated Data Set}\label{sec-sim-data}}

Using an administrative data set from the State of Pennsylvania
(2010-2019) as a guide, we created a simulated data set of the same size
(\(N = 864,422\)) to act as a hypothetical population of records.
Associations within the independent variables and the control variables
add complications of confounding to the analysis which may or may not
bias estimates (Bartlett et al. 2015). To avoid this complication, we
draw from the simulated population which was generated so that the only
relationships are with the sentencing decision dependent variable. For
all of the models in this paper, we used the independent and control
variables race, age, age-squared, offense gravity score (OGS),
OGS-squared, previous criminal history, crime type, recommended minimum,
trial (or plea), gender, county, and year to model the sentencing
decision {[}\textbf{needs a citation}{]}. These variables have varying
levels of evidence of being meaningful for modelling the sentencing
decision. We include as many variables as possible in the simulation
model to create a richer data set.

Summaries of each of the independent and control variables were used to
construct the simulated data. For categorical demographic or legal
variables, we used the marginal probabilities of belonging to each
category to randomly and independently select levels for each variable
for each hypothetical record. The means and variances of the continuous
variables were used to draw from normal or gamma distributions depending
on if the variable should be strictly positive. Using the observed data,
a complete case analysis logistic regression for the sentencing decision
on all of the previously referenced independent and control variables
was used to obtain plausible values for each independent and control
variables' regression coefficient, \(\beta\). From the regression
coefficients, we calculated the fitted probabilities for each case using
our simulated data as the independent and control variables to then draw
an incarceration decision for each case by flipping a coin weighted by
the fitted probabilities. Appending this new simulated dependent
variable to the simulated independent and control variables, we have our
complete simulated data set.

Logistic regression modelling the sentencing decision given the
independent and control variables was then used to illustrate the
potential effects of missing data on estimating the race effect between
Black and White defendants. In the administrative data we estimated this
effect to be 25.5\%, ie a Black defendant are 1.255 times more likely to
be incarcerated than an otherwise identical White defendant. Note that
this estimate was not made with a causal inference procedure so it
should not be treated a valid estimate of the real race effect. Instead,
this is only used as a plausible population value for our simulated data
set. Within the simulated analyses, we make similarly flawed
estimations, but this is unfortunately not out of line with the majority
of the criminological literature on the race effect. In addition, we do
not attempt to address the missingness that is present in the
Pennsylvania data, which we show can lead to flawed and unreliable
estimates with our simulations.

From the observed data, we also collected the observed missing data
patterns (which combinations of variables have missing values for a
given defendant) and the proportions with which each pattern appears.
The six observed and two unobserved missingness patterns, \(R\), in the
Pennsylvania administrative data are detailed in
Table~\ref{tbl-miss-patt}. From Table~\ref{tbl-miss-patt}, we can see
that most defendants had complete records (96.5\%), 3\% were missing
only race, and all other patterns were observed in less than 0.5\% of
the observations. In addition to these eight patterns, three cases were
also missing previous criminal history (PRS), but we will disregard
these three cases for brevity\footnote{Each additional variable doubles
  the number of missing data patterns, so a fourth variable would result
  in 16 patterns with only 7 observed patterns.}. The three variables we
will induce missing values in for our simulation are recommended
minimum, age (and the square of age), and race, with the most common
pattern with missing values being race missing on its own.

\hypertarget{tbl-miss-patt}{}
\begin{longtable}[]{@{}
  >{\raggedright\arraybackslash}p{(\columnwidth - 16\tabcolsep) * \real{0.0972}}
  >{\raggedright\arraybackslash}p{(\columnwidth - 16\tabcolsep) * \real{0.0972}}
  >{\raggedright\arraybackslash}p{(\columnwidth - 16\tabcolsep) * \real{0.0694}}
  >{\raggedright\arraybackslash}p{(\columnwidth - 16\tabcolsep) * \real{0.0694}}
  >{\centering\arraybackslash}p{(\columnwidth - 16\tabcolsep) * \real{0.0694}}
  >{\centering\arraybackslash}p{(\columnwidth - 16\tabcolsep) * \real{0.1111}}
  >{\centering\arraybackslash}p{(\columnwidth - 16\tabcolsep) * \real{0.0833}}
  >{\centering\arraybackslash}p{(\columnwidth - 16\tabcolsep) * \real{0.2917}}
  >{\raggedleft\arraybackslash}p{(\columnwidth - 16\tabcolsep) * \real{0.1111}}@{}}
\toprule()
\begin{minipage}[b]{\linewidth}\raggedright
R
\end{minipage} & \begin{minipage}[b]{\linewidth}\raggedright
INCAR
\end{minipage} & \begin{minipage}[b]{\linewidth}\raggedright
OGS
\end{minipage} & \begin{minipage}[b]{\linewidth}\raggedright
\ldots{}
\end{minipage} & \begin{minipage}[b]{\linewidth}\centering
AGE
\end{minipage} & \begin{minipage}[b]{\linewidth}\centering
RECMIN
\end{minipage} & \begin{minipage}[b]{\linewidth}\centering
RACE
\end{minipage} & \begin{minipage}[b]{\linewidth}\centering
Observed in PA Data
\end{minipage} & \begin{minipage}[b]{\linewidth}\raggedleft
Count
\end{minipage} \\
\midrule()
\endfirsthead
\toprule()
\begin{minipage}[b]{\linewidth}\raggedright
R
\end{minipage} & \begin{minipage}[b]{\linewidth}\raggedright
INCAR
\end{minipage} & \begin{minipage}[b]{\linewidth}\raggedright
OGS
\end{minipage} & \begin{minipage}[b]{\linewidth}\raggedright
\ldots{}
\end{minipage} & \begin{minipage}[b]{\linewidth}\centering
AGE
\end{minipage} & \begin{minipage}[b]{\linewidth}\centering
RECMIN
\end{minipage} & \begin{minipage}[b]{\linewidth}\centering
RACE
\end{minipage} & \begin{minipage}[b]{\linewidth}\centering
Observed in PA Data
\end{minipage} & \begin{minipage}[b]{\linewidth}\raggedleft
Count
\end{minipage} \\
\midrule()
\endhead
r = 1 & & & & & & & Y & 834546 \\
r = 2 & & & & & & ? & Y & 26206 \\
r = 3 & & & & & ? & & Y & 2076 \\
r = 4 & & & & & ? & ? & Y & 34 \\
r = 5 & & & & ? & & & Y & 1465 \\
r = 6 & & & & ? & & ? & Y & 92 \\
r = 7 & & & & ? & ? & & N & 0 \\
r = 8 & & & & ? & ? & ? & N & 0 \\
\bottomrule()
\caption{\label{tbl-miss-patt}Patterns of missingness in the
Pennsylvania data where a blank represents being observed and ``?''
represents missing values. The first six patterns were observed in the
Pennsylvania data while patterns seven and eight were
not.}\tabularnewline
\end{longtable}

\hypertarget{sec-miss-model}{%
\subsection{Missingness Model}\label{sec-miss-model}}

We then built a multinomial missingness model based on the one from
(Perkins et al. 2018). Let \(V\) be the set of variables that are
completely observed in every missing data pattern: incarceration, OGS,
OGS-squared, previous criminal history, crime type, gender, trial,
county and race. Let \(W_r\) be the the set of variables that may have
missing values in some patterns but not necessarily pattern \(r\): age,
age-squared, recommended minimum, and race. Statistically, we have three
core assumptions we can make about how missing data patterns are
generated. Missing values can be missing completely at random (MCAR),
missing at random (MAR), or missing not at random (MNAR). Missing
completely at random data imply that the missing data patterns are
determined entirely independently of the data and are dispersed randomly
throughout the data. MCAR is a typical (implicit) assumption that many
analysts make regarding missing data, even when not appropriate. Missing
at random data are missing dependent on the values of the observed data,
for example age may be more likely to be missing for male defendants.
Missing not at random data are missing due to the unobserved value of
the missing variable, so age may be more likely to be missing for
younger defendants.

The probability that any given case belongs to a particular pattern is
given by \(p_{i,r}\) for \(i = 1,…,n\) and \(r = 1,…,8\). The
probabilities of membership to each missingness pattern will roughly
mirror the proportions seen in the administrative data. Exceptions are
made for patterns seven and eight to give them 1/864,422 chance of being
selected instead of zero chance.

The multinomial model that gives the matrix of probabilities
\(\mathbf{P} = \{p_{i,r}\}_{1 \leq i \leq n, 1 \leq r \leq 8}\) of
having each pattern for each observation is given by

\[
P(R = r | V, L_r, W_r) = \frac{\exp(\alpha_r + \beta_r V + \gamma_r W_r)}{1 + \sum_{k = 1}^R\exp(\alpha_k + \beta_k V + \gamma_k W_k)}
\]

where the parameters \(\gamma_r\) are set to create MAR, MNAR
missingness and to induce over- or under-estimation of the race effect
odds ratio. For MAR, we set the parameter(s) to zero if they correspond
to the variable(s) that are missing for that pattern. For MNAR, the
parameter(s) can be non-zero even when they correspond to variable(s)
that are missing. To set the missing data patterns, we sample from the
categorical distribution with the above probabilities for
\(p_{i,r} = P(R = r | \mathbf{v}_i, \mathbf{l}_{r,i}, \mathbf{w}_{r,i})\)
for \(r = 1,…, 8\) and \(i = 1,.., n\). The intercept parameters
\(\alpha_r\) are determined by back calculating from the observed
proportion of each missingness pattern as in (Perkins et al. 2018).

To achieve the desired effects of the missing data in the complete case
analyses, we included interactions between non-incarceration and the
defendant being Black \(Z_1 = I(INCAR = 0)I(RACE = BLACK)\) and the
interactions between incarceration and the defendant being Black
\(Z_{2} = I(INCAR = 1)I(RACE = BLACK)\). These variables are missing for
the same patterns for which race is missing. To get over-estimates, we
emphasize \(Z_1\). To get under-estimates, we de-emphasize \(Z_1\) and
emphasize \(Z_2\).

\hypertarget{sec-analysis}{%
\subsection{Analysis Methods}\label{sec-analysis}}

\hypertarget{sec-cca}{%
\subsubsection{Complete Case Analysis}\label{sec-cca}}

Complete Case Analysis (CCA) is the default analysis procedure for many
different statistical software packages and used mostly by
non-statisticians. This ad hoc method removes all incomplete
observations to create a complete rectangular data set with no missing
values (White and Carlin 2010). For example, if we have a data set
containing sentencing decision, age, sex, and previous criminal history
for 10,000 defendants and 200 defendants are missing values for previous
criminal history, then the complete case analysis would only evaluate
the 9,800 cases with complete records, dropping all cases that are
missing previous histories.

This can be deeply problematic when values are missing completely at
random (van der Heijden et al. 2006). If female defendants are more
likely to be missing age observations, then more female defendant cases
will be dropped from the analysis potentially biasing our estimates.

\hypertarget{sec-mi}{%
\subsubsection{Multiple Imputation}\label{sec-mi}}

There are several methods for more rigorous treatment of incomplete
data. Principal among these are data augmentation for Bayesian analysis,
EM algorithm for certain problems, and multiple imputation for a broad
class of problems including our particular situation (Dempster et al.
1977; Tanner and Wong 1987; Rubin 1987; Little and Wang 1996). Further
reviews and extensions of multiple imputation have been added to the
literature since (Schafer 1999; Raghunathan et al. 2001; Schafer 2003;
Harel and Zhou 2007).

Multiple imputation (MI) was developed in the 1970s and 1980s (Rubin
1987) and consists of three stages (i) imputation, (ii) analysis, and
(iii) combining. To multiply impute an incomplete data set, we take the
data \(D\) and split it into observed and missing parts
\(D = (D_{obs}, D_{miss})\), create \(M\) copies of \(D_{obs}\), and
create \(M\) sets of imputed data \(D_{imp}^m\) to complete \(D_{miss}\)
and pair with each copy of \(D_{obs}\). Our new collection of completed
data sets is \(\mathbf{D}^* = \{D_1,…, D_M\}\) where
\(D_m = (D_{obs}, D_{imp}^m)\). A complete data analysis is then
performed on each of the \(M\) completed data sets to obtain estimates
\(Q_m\) and variances \$U\_m\$. In our case, we perform logistic
regressions including all of the independent and control variables to
collect the odds ratio of the race effect \(Q_m\) and \(U_m\) is its
variance. These values can then be combined to produce point estimates,
to perform null hypothesis significance tests, or to create confidence
intervals (Rubin 1987). The point estimate is the mean of the estimates
from each completed data set,
\(\bar{Q} = \frac{1}{M} \sum_{m=1}^M Q_m\), with variance
\(T = \bar{U} + (1 + \frac{1}{M})B\) which is a combination of the
average variance of each completed data estimate
\(\bar{U} = \frac{1}{M}\sum_{m=1}^M U_m\) and the variance of the
estimates between themselves
\(B = \frac{1}{M-1}\sum_{m=1}^M (Q_m - \bar{Q})^2\). These formulas are
known as Rubin's Rules (Rubin 1987; Schafer 1999).

The multiple imputation procedure is very flexible and allows an analyst
to choose many different ways to produce imputations and to model what
causes the values to be missing in the first place. One popular
implementation is MICE, or Multiple Imputation by Chained Equations
(Raghunathan et al. 2001; Azur et al. 2011) which imputes each variable
with missing values in a sequence of regressions or other models that
can produce guesses based on the observed data. Our simulations will use
MICE to multiply impute the missing values with predictive mean matching
in the simulated data and create our MI estimates for the race effect.

\hypertarget{sec-simulations}{%
\subsection{Simulations}\label{sec-simulations}}

We performed a simulation study to investigate various research
decisions' and missingness models' impacts on the estimation of the race
effect. In particular, we are interested in the impacts of sample size,
the number of imputations, and MAR vs MNAR missingness. Additionally, we
choose missingness model parameters to demonstrate that incomplete data
can result in dramatic over- and under-estimations of the race effect.

Small, medium, large, and very large sample sizes of
\(n = 500, 1000, 2500, 25000\) with 225 simulation iterations were used
to create approximations of the sampling distributions under different
settings. The simulation procedure is as follows.

For each type of missingness (MAR/MNAR):

\begin{enumerate}
\def\labelenumi{\arabic{enumi}.}
\tightlist
\item
  Draw a new sample of size \(n\) from the population data \(D_{POP}\)
  without replacement. This sampled data set is called complete,
  \(D_{COMP}\). This data set is used for all five analyses in this
  iteration with the same missing data patterns imposed for the CCA and
  MI analyses.
\item
  Missing data patterns are then assigned with the model specified in
  Section~\ref{sec-miss-model}. Parameters are set to determine if it is
  a MAR vs MNAR mechanism and whether the CCA estimates will be over- or
  under-estimates. This creates a triplet of data sets,
  \((D_{COMP}, D_{OVER}, D_{UNDR})\) for each type of missingness. The
  intended direction of the CCA estimate relative to the complete data
  estimate (over or under) is denoted \(Dir\).

  \begin{itemize}
  \tightlist
  \item
    Note that while the missing values are created under MAR or MNAR
    models, the CCA is known to be unbiased under a MCAR mechanism and
    the MI procedure is known to be unbiased under a MAR mechanism
    (Harel and Zhou 2007; White and Carlin 2010).
  \item
    For our MI analyses, we assume that the missing data is MAR while
    CCA implicitly assumes MCAR.
  \end{itemize}
\item
  Logistic regressions are fit on the complete sampled data set
  \(D_{COMP}\), the over-estimate data set \(D_{OVER}\), and the
  under-estimate data set \(D_{UNDR}\) with both CCA and with MI (Rubin
  1987) with \(M\) imputations. This creates five regressions for each
  iteration: a complete data analysis on \(D_{COMP}\), two CCA on
  \((D_{OVER}, D_{UNDR})\), and two MI analyses also on
  \((D_{OVER}, D_{UNDR})\).

  \begin{itemize}
  \tightlist
  \item
    For each regression, the odds ratio \(OR\) and proportion of
    incomplete cases\footnote{The number of complete cases divided by
      the total number of cases.} \(p_{miss}\) are recorded and used to
    calculate the Bias,
    \(b_{OR} = \hat{OR}_{Race}(D_{Dir}) - \exp(\beta_{Race})\), and
    difference between the missing data set estimate and the complete
    data set estimate,
    \(d_{Dir}^{Analysis} = \hat{OR}_{Race}^{Analysis}(D_{Dir}) - \hat{OR}_{Race}(D_{Comp})\).
    The differences \(d_{OR}\) will be our primary quantity of interest.
  \end{itemize}
\end{enumerate}

Repeat for 225 iterations.

The simulations were run across ten different settings as shown in
Table~\ref{tbl-sim-settings}.

\hypertarget{tbl-sim-settings}{}
\begin{longtable}[]{@{}lc@{}}
\toprule()
Sample Size (\(n\)) & No.~of Imp. (\(M\)) \\
\midrule()
\endfirsthead
\toprule()
Sample Size (\(n\)) & No.~of Imp. (\(M\)) \\
\midrule()
\endhead
500 & 3 \\
500 & 5 \\
500 & 8 \\
500 & 25 \\
1000 & 3 \\
1000 & 5 \\
1000 & 8 \\
2500 & 3 \\
2500 & 5 \\
2500 & 8 \\
25000 & 3 \\
\bottomrule()
\caption{\label{tbl-sim-settings}The settings for the different
simulations. All used 225 iterations, except for the 25,000 case
simulation.}\tabularnewline
\end{longtable}

To determine the number of imputations to perform, the rate of missing
information can be used (Harel 2007; Graham et al. 2007). The rate of
missing information quantifies the amount of information within the
missing data compared to the (typically hypothetical) complete data set.
Through another set of simulations (omitted for brevity), it was
determined that there should be between five and eight imputations to
achieve a relative efficiency of \textgreater0.99 using the methods
described by (Rubin 1987; Harel 2007). For a relative efficiency of
\textasciitilde0.95, three imputations is sufficient. The primary
comparison of interest from the simulations is between missing and
complete data estimates using the same sample from the simulated
population data, \(d_{Dir}^{Analysis}\), where the missing data estimate
is made using CCA or MI.

See the Appendix for detailed descriptions of the selected parameters.

\hypertarget{sec-results}{%
\section{RESULTS}\label{sec-results}}

The simulations demonstrate that regardless of sample size and with less
than 10\% of cases being incomplete, complete case analysis can still
produce biased estimates for the race effect's odds ratio. When MAR
conditions are satisfied, the estimates of the odds ratio calculated
under multiple imputation are again unbiased. When the odds ratio is
estimated with the complete data, the estimate is made with maximum
likelihood so it is asymptotically unbiased.

The simulation results are displayed in
\textbf{?@fig-or-sim}\textbf{?@fig-diff-sim}\textbf{?@tbl-prop-miss}.
These figures display the distribution of the race effect odds ratios
estimated by both analyses under both missingness mechanisms with
various simulation settings. The key comparison in the figures is from
the boxplots for MI with MAR and MNAR missingness to the boxplots for
the CCA with the same missingness type. Multiple imputation produces
estimates with a median closer to the complete data estimate in every
case.

Depending on the missingness model parameters, between 5-10\% of cases
are incomplete for each simulated sample regardless of sample size as
shown in \textbf{?@tbl-prop-miss}. While this is slightly higher than
the proportion of incomplete cases in the observed data, it still seems
to be a plausible amount of incompleteness (Bales and Piquero 2012;
Kutateladze et al. 2014; Cassidy and Rydberg 2020) {[}\textbf{more
citations?}{]}. Even this relatively small amount of missingness
produces biases in the complete case analysis estimates under both MAR
and MNAR missingness mechanisms. We have chosen the parameters for the
missingness model to flip the direction of the race effect from the true
underlying effect when estimating with CCA. As a result, the estimates
made with complete case analysis are unreliable and do not provide
suitable estimates of the race effect as we wouldn't know the nature of
a missingness mechanism with a real-world data set.

Statistical bias refers to the expected difference between the estimate
and the true population parameter value. In our situation, we want our
estimates to be unbiased so that we can accurately estimate at the true
value of the race effect.

The bias is corrected for when we use multiple imputation with an
appropriate number of imputations (Allison 2000; Harel 2007; Graham et
al. 2007). The multiple imputation analysis produces unbiased results
that will measure the true race effect on average when the missing data
are generated under a MAR mechanism or when we include some model for
the missingness mechanism in the imputation model to address the MNAR
assumption (Bartlett et al. 2015). Alternative tools to MI have been
developed to address MNAR or non-ignorable missing data that do not rely
on multiple imputation (Ibrahim 1990; Vach and Schumacher 1993; Little
and Wang 1996). Those tools are beyond the scope of this paper.

In our simulations, we assume that the missing data mechanism is MAR and
ignorable, so we omit any modeling of the mechanism in our imputations.
The estimates made with MI under the MAR generated data are unbiased and
closely resemble the complete data estimates demonstrating that the MI
procedure is able to recover most of the information lost when removing
incomplete cases with CCA. While not addressing the missingness model is
appropriate for the MAR generated data, it fails to overcome the bias
induced by the MNAR generated data resulting in only slightly mitigated
biases in the MI estimates with MNAR data.

According to the simulations, the median difference between the complete
case estimates and the complete data estimates gets slightly closer to
zero as the sample size increases from 500 cases to 25000 cases. The
spread of the differences shrinks as well resulting in a sampling
distribution that still indicates that there is bias even for large
samples to still flip the direction of the race effect. Increasing the
number of imputations from three to eight doesn't result in dramatically
different results. The increase in relative efficiency is fairly small,
going from roughly 95\% to 99+\% as imputations increase from three to
eight, so the impact is negligible.

I also did a simulation with 105 iterations, a sample size of 25,000,
and \(M = 3\) imputations, since that still gives roughly 95\% relative
efficiency but reduces the computational cost significantly. From these
simulations, we further establish the patterns seen with the smaller
sample sizes. Roughly 5-10\% of cases are incomplete. The distribution
of the differences is narrower than for the smaller samples and the
median differences are still clearly different from zero and roughly
equivalent to the median differences under the \(n = 2,500\) simulations
as seen in \textbf{?@fig-or-sim}\textbf{?@fig-diff-sim}. As this
simulation is 10x larger than the 2500 sample size simulation, we can
likely expect these patterns to continue for even larger sample sizes.
Large sample sizes are therefore not a cure-all for missing data and
still must be given special attention.

\hypertarget{sec-discussion}{%
\section{DISCUSSION}\label{sec-discussion}}

\hypertarget{sec-conclusion}{%
\section{CONCLUSION}\label{sec-conclusion}}

\hypertarget{references}{%
\section*{REFERENCES}\label{references}}
\addcontentsline{toc}{section}{REFERENCES}

\hypertarget{refs}{}
\begin{CSLReferences}{1}{0}
\leavevmode\vadjust pre{\hypertarget{ref-allisonMultipleImputationMissing2000}{}}%
Allison PD (2000) Multiple {Imputation} for {Missing Data}: {A
Cautionary Tale}. Sociological Methods \& Research 28:301--309.
\url{https://doi.org/10.1177/0049124100028003003}

\leavevmode\vadjust pre{\hypertarget{ref-azurMultipleImputationChained2011}{}}%
Azur MJ, Stuart EA, Frangakis C, Leaf PJ (2011) Multiple imputation by
chained equations: What is it and how does it work? International
Journal of Methods in Psychiatric Research 20:40--49.
\url{https://doi.org/10.1002/mpr.329}

\leavevmode\vadjust pre{\hypertarget{ref-balesRacialEthnicDifferentials2012}{}}%
Bales WD, Piquero AR (2012) Racial/{Ethnic Differentials} in
{Sentencing} to {Incarceration}. Justice Quarterly 29:742--773.
\url{https://doi.org/10.1080/07418825.2012.659674}

\leavevmode\vadjust pre{\hypertarget{ref-bartlettAsymptoticallyUnbiasedEstimation2015}{}}%
Bartlett JW, Harel O, Carpenter JR (2015) Asymptotically {Unbiased
Estimation} of {Exposure Odds Ratios} in {Complete Records Logistic
Regression}. American Journal of Epidemiology 182:730--736.
\url{https://doi.org/10.1093/aje/kwv114}

\leavevmode\vadjust pre{\hypertarget{ref-cassidyDoesSentenceType2020}{}}%
Cassidy M, Rydberg J (2020) Does {Sentence Type} and {Length Matter}?:
{Interactions} of {Age}, {Race}, {Ethnicity}, and {Gender} on {Jail} and
{Prison Sentences}. Criminal Justice and Behavior 47:61

\leavevmode\vadjust pre{\hypertarget{ref-dempsterMaximumLikelihoodIncomplete1977}{}}%
Dempster AP, Laird NM, Rubin DB (1977) Maximum {Likelihood} from
{Incomplete Data} via the {EM Algorithm}. Journal of the Royal
Statistical Society Series B (Methodological) 39:1--38

\leavevmode\vadjust pre{\hypertarget{ref-grahamHowManyImputations2007}{}}%
Graham JW, Olchowski AE, Gilreath TD (2007) How {Many Imputations} are
{Really Needed}? {Some Practical Clarifications} of {Multiple Imputation
Theory}. Prevention Science 8:206--213.
\url{https://doi.org/10.1007/s11121-007-0070-9}

\leavevmode\vadjust pre{\hypertarget{ref-harelInferencesMissingInformation2007}{}}%
Harel O (2007) Inferences on missing information under multiple
imputation and two-stage multiple imputation. Statistical Methodology
4:75--89. \url{https://doi.org/10.1016/j.stamet.2006.03.002}

\leavevmode\vadjust pre{\hypertarget{ref-harelMultipleImputationReview2007}{}}%
Harel O, Zhou X-H (2007) Multiple imputation: Review of theory,
implementation and software. Statistics in Medicine 26:3057--3077.
\url{https://doi.org/10.1002/sim.2787}

\leavevmode\vadjust pre{\hypertarget{ref-ibrahimIncompleteDataGeneralized1990}{}}%
Ibrahim JG (1990) Incomplete {Data} in {Generalized Linear Models}.
Journal of the American Statistical Association 85:765--769.
\url{https://doi.org/10.2307/2290013}

\leavevmode\vadjust pre{\hypertarget{ref-kutateladzeCumulativeDisadvantageExamining2014}{}}%
Kutateladze BL, Andiloro NR, Johnson BD, Spohn CC (2014) Cumulative
{Disadvantage}: {Examining Racial} and {Ethnic Disparity} in
{Prosecution} and {Sentencing}. Criminology 52:514--551.
\url{https://doi.org/10.1111/1745-9125.12047}

\leavevmode\vadjust pre{\hypertarget{ref-littlePatternMixtureModelsMultivariate1996}{}}%
Little RJA, Wang Y (1996) Pattern-{Mixture Models} for {Multivariate
Incomplete Data} with {Covariates}. Biometrics 52:98--111.
\url{https://doi.org/10.2307/2533148}

\leavevmode\vadjust pre{\hypertarget{ref-perkinsPrincipledApproachesMissing2018}{}}%
Perkins NJ, Cole SR, Harel O, et al (2018) Principled {Approaches} to
{Missing Data} in {Epidemiologic Studies}. American Journal of
Epidemiology 187:568--575. \url{https://doi.org/10.1093/aje/kwx348}

\leavevmode\vadjust pre{\hypertarget{ref-raghunathanMultivariateTechniqueMultiply2001}{}}%
Raghunathan TE, Lepkowski JM, Hoewyk JV, Solenberger P (2001) A
{Multivariate Technique} for {Multiply Imputing Missing Values Using} a
{Sequence} of {Regression Models}. Statistics Canada 27:85--95

\leavevmode\vadjust pre{\hypertarget{ref-rubinMultipleImputationNonresponse1987}{}}%
Rubin DB (1987) {Multiple Imputation for Nonresponse in Surveys
\textbar{} Wiley Series in Probability and Statistics}. {Wiley}, {New
York}

\leavevmode\vadjust pre{\hypertarget{ref-schaferMultipleImputationPrimer1999}{}}%
Schafer JL (1999) Multiple imputation: A primer. Statistical Methods in
Medical Research 8:3--15.
\url{https://doi.org/10.1177/096228029900800102}

\leavevmode\vadjust pre{\hypertarget{ref-schaferMultipleImputationMultivariate2003}{}}%
Schafer JL (2003) Multiple {Imputation} in {Multivariate Problems When}
the {Imputation} and {Analysis Models Differ}. Statistica Neerlandica
57:19--35. \url{https://doi.org/10.1111/1467-9574.00218}

\leavevmode\vadjust pre{\hypertarget{ref-tannerCalculationPosteriorDistributions1987}{}}%
Tanner MA, Wong WH (1987) The {Calculation} of {Posterior Distributions}
by {Data Augmentation}. Journal of the American Statistical Association
82:528--540. \url{https://doi.org/10.2307/2289457}

\leavevmode\vadjust pre{\hypertarget{ref-vachLogisticRegressionIncompletely1993}{}}%
Vach W, Schumacher M (1993) Logistic {Regression} with {Incompletely
Observed Categorical Covariates}: {A Comparison} of {Three Approaches}.
Biometrika 80:353--362. \url{https://doi.org/10.2307/2337205}

\leavevmode\vadjust pre{\hypertarget{ref-vanderheijdenImputationMissingValues2006}{}}%
van der Heijden GJMG, T. Donders AR, Stijnen T, Moons KGM (2006)
Imputation of missing values is superior to complete case analysis and
the missing-indicator method in multivariable diagnostic research: {A}
clinical example. Journal of Clinical Epidemiology 59:1102--1109.
\url{https://doi.org/10.1016/j.jclinepi.2006.01.015}

\leavevmode\vadjust pre{\hypertarget{ref-whiteBiasEfficiencyMultiple2010}{}}%
White IR, Carlin JB (2010) Bias and efficiency of multiple imputation
compared with complete-case analysis for missing covariate values.
Statistics in Medicine 29:2920--2931.
\url{https://doi.org/10.1002/sim.3944}

\end{CSLReferences}

\hypertarget{appendix}{%
\section{APPENDIX}\label{appendix}}

\hypertarget{sec-mnar}{%
\subsection{MNAR - Simulation Parameters}\label{sec-mnar}}

The missingness model assigns slightly more missingness than we saw in
the administrative data while still having a relatively low number of
incomplete cases.

\hypertarget{tbl-mnar-over}{}
\begin{longtable}[]{@{}
  >{\raggedright\arraybackslash}p{(\columnwidth - 24\tabcolsep) * \real{0.1124}}
  >{\raggedleft\arraybackslash}p{(\columnwidth - 24\tabcolsep) * \real{0.1011}}
  >{\raggedleft\arraybackslash}p{(\columnwidth - 24\tabcolsep) * \real{0.0449}}
  >{\raggedleft\arraybackslash}p{(\columnwidth - 24\tabcolsep) * \real{0.0449}}
  >{\raggedleft\arraybackslash}p{(\columnwidth - 24\tabcolsep) * \real{0.0449}}
  >{\raggedleft\arraybackslash}p{(\columnwidth - 24\tabcolsep) * \real{0.1124}}
  >{\raggedleft\arraybackslash}p{(\columnwidth - 24\tabcolsep) * \real{0.0787}}
  >{\raggedleft\arraybackslash}p{(\columnwidth - 24\tabcolsep) * \real{0.0674}}
  >{\raggedleft\arraybackslash}p{(\columnwidth - 24\tabcolsep) * \real{0.0449}}
  >{\raggedleft\arraybackslash}p{(\columnwidth - 24\tabcolsep) * \real{0.0674}}
  >{\raggedleft\arraybackslash}p{(\columnwidth - 24\tabcolsep) * \real{0.0787}}
  >{\raggedleft\arraybackslash}p{(\columnwidth - 24\tabcolsep) * \real{0.1124}}
  >{\raggedleft\arraybackslash}p{(\columnwidth - 24\tabcolsep) * \real{0.0899}}@{}}
\toprule()
\begin{minipage}[b]{\linewidth}\raggedright
\end{minipage} & \begin{minipage}[b]{\linewidth}\raggedleft
Incar
\end{minipage} & \begin{minipage}[b]{\linewidth}\raggedleft
OGS
\end{minipage} & \begin{minipage}[b]{\linewidth}\raggedleft
PRS
\end{minipage} & \begin{minipage}[b]{\linewidth}\raggedleft
\ldots{}
\end{minipage} & \begin{minipage}[b]{\linewidth}\raggedleft
Black
\end{minipage} & \begin{minipage}[b]{\linewidth}\raggedleft
Latino
\end{minipage} & \begin{minipage}[b]{\linewidth}\raggedleft
Other
\end{minipage} & \begin{minipage}[b]{\linewidth}\raggedleft
Age
\end{minipage} & \begin{minipage}[b]{\linewidth}\raggedleft
AgeSq
\end{minipage} & \begin{minipage}[b]{\linewidth}\raggedleft
RecMin
\end{minipage} & \begin{minipage}[b]{\linewidth}\raggedleft
Z\_1
\end{minipage} & \begin{minipage}[b]{\linewidth}\raggedleft
Z\_2
\end{minipage} \\
\midrule()
\endfirsthead
\toprule()
\begin{minipage}[b]{\linewidth}\raggedright
\end{minipage} & \begin{minipage}[b]{\linewidth}\raggedleft
Incar
\end{minipage} & \begin{minipage}[b]{\linewidth}\raggedleft
OGS
\end{minipage} & \begin{minipage}[b]{\linewidth}\raggedleft
PRS
\end{minipage} & \begin{minipage}[b]{\linewidth}\raggedleft
\ldots{}
\end{minipage} & \begin{minipage}[b]{\linewidth}\raggedleft
Black
\end{minipage} & \begin{minipage}[b]{\linewidth}\raggedleft
Latino
\end{minipage} & \begin{minipage}[b]{\linewidth}\raggedleft
Other
\end{minipage} & \begin{minipage}[b]{\linewidth}\raggedleft
Age
\end{minipage} & \begin{minipage}[b]{\linewidth}\raggedleft
AgeSq
\end{minipage} & \begin{minipage}[b]{\linewidth}\raggedleft
RecMin
\end{minipage} & \begin{minipage}[b]{\linewidth}\raggedleft
Z\_1
\end{minipage} & \begin{minipage}[b]{\linewidth}\raggedleft
Z\_2
\end{minipage} \\
\midrule()
\endhead
Pattern 1 & 2.302585 & 0 & 0 & 0 & -2.302585 & 0 & 0 & 0 & 0 & 0 &
-6.907755 & 4.60517 \\
Pattern 2 & 2.302585 & 0 & 0 & 0 & -2.302585 & 0 & 0 & 0 & 0 & 0 &
-6.907755 & 4.60517 \\
Pattern 3 & 2.302585 & 0 & 0 & 0 & -2.302585 & 0 & 0 & 0 & 0 & 0 &
-6.907755 & 4.60517 \\
Pattern 4 & 2.302585 & 0 & 0 & 0 & -2.302585 & 0 & 0 & 0 & 0 & 0 &
-6.907755 & 4.60517 \\
Pattern 5 & 2.302585 & 0 & 0 & 0 & -2.302585 & 0 & 0 & 0 & 0 & 0 &
-6.907755 & 4.60517 \\
Pattern 6 & 2.302585 & 0 & 0 & 0 & -2.302585 & 0 & 0 & 0 & 0 & 0 &
-6.907755 & 4.60517 \\
Pattern 7 & 2.302585 & 0 & 0 & 0 & -2.302585 & 0 & 0 & 0 & 0 & 0 &
-6.907755 & 4.60517 \\
Pattern 8 & 2.302585 & 0 & 0 & 0 & -2.302585 & 0 & 0 & 0 & 0 & 0 &
-6.907755 & 4.60517 \\
\bottomrule()
\caption{\label{tbl-mnar-over}The parameters for the missingness model
for CCA over-estimates with MNAR missing values.}\tabularnewline
\end{longtable}

To over-estimate the race effect, missingness in race is conditional on
race being Black and incarceration so we set the parameters of the model
to be \(\exp(\beta) = (10, 1,... ,1)' \in \mathbb{R}^{88}\) and
\(\exp(\gamma) = (0.1, 1,1,1,1,1, 0.001, 100)' \in \mathbb{R}^8\). The
first element in \(\beta\) emphasizes missingness based on
incarceration. The first element in \(\gamma\) de-emphasizes missingness
based on being Black while the seventh element corresponding to \(Z_1\)
greatly de-emphasizes missingness for defendants who aren't incarcerated
and are Black and the eighth corresponding to \(Z_2\) greatly emphasizes
missingness for individuals who are incarcerated and are Black.

\hypertarget{tbl-mnar-under}{}
\begin{longtable}[]{@{}
  >{\raggedright\arraybackslash}p{(\columnwidth - 24\tabcolsep) * \real{0.1205}}
  >{\raggedleft\arraybackslash}p{(\columnwidth - 24\tabcolsep) * \real{0.1084}}
  >{\raggedleft\arraybackslash}p{(\columnwidth - 24\tabcolsep) * \real{0.0482}}
  >{\raggedleft\arraybackslash}p{(\columnwidth - 24\tabcolsep) * \real{0.0482}}
  >{\raggedleft\arraybackslash}p{(\columnwidth - 24\tabcolsep) * \real{0.0482}}
  >{\raggedleft\arraybackslash}p{(\columnwidth - 24\tabcolsep) * \real{0.1084}}
  >{\raggedleft\arraybackslash}p{(\columnwidth - 24\tabcolsep) * \real{0.0843}}
  >{\raggedleft\arraybackslash}p{(\columnwidth - 24\tabcolsep) * \real{0.0723}}
  >{\raggedleft\arraybackslash}p{(\columnwidth - 24\tabcolsep) * \real{0.0482}}
  >{\raggedleft\arraybackslash}p{(\columnwidth - 24\tabcolsep) * \real{0.0723}}
  >{\raggedleft\arraybackslash}p{(\columnwidth - 24\tabcolsep) * \real{0.0843}}
  >{\raggedleft\arraybackslash}p{(\columnwidth - 24\tabcolsep) * \real{0.1084}}
  >{\raggedleft\arraybackslash}p{(\columnwidth - 24\tabcolsep) * \real{0.0482}}@{}}
\toprule()
\begin{minipage}[b]{\linewidth}\raggedright
\end{minipage} & \begin{minipage}[b]{\linewidth}\raggedleft
Incar
\end{minipage} & \begin{minipage}[b]{\linewidth}\raggedleft
OGS
\end{minipage} & \begin{minipage}[b]{\linewidth}\raggedleft
PRS
\end{minipage} & \begin{minipage}[b]{\linewidth}\raggedleft
\ldots{}
\end{minipage} & \begin{minipage}[b]{\linewidth}\raggedleft
Black
\end{minipage} & \begin{minipage}[b]{\linewidth}\raggedleft
Latino
\end{minipage} & \begin{minipage}[b]{\linewidth}\raggedleft
Other
\end{minipage} & \begin{minipage}[b]{\linewidth}\raggedleft
Age
\end{minipage} & \begin{minipage}[b]{\linewidth}\raggedleft
AgeSq
\end{minipage} & \begin{minipage}[b]{\linewidth}\raggedleft
RecMin
\end{minipage} & \begin{minipage}[b]{\linewidth}\raggedleft
Z\_1
\end{minipage} & \begin{minipage}[b]{\linewidth}\raggedleft
Z\_2
\end{minipage} \\
\midrule()
\endfirsthead
\toprule()
\begin{minipage}[b]{\linewidth}\raggedright
\end{minipage} & \begin{minipage}[b]{\linewidth}\raggedleft
Incar
\end{minipage} & \begin{minipage}[b]{\linewidth}\raggedleft
OGS
\end{minipage} & \begin{minipage}[b]{\linewidth}\raggedleft
PRS
\end{minipage} & \begin{minipage}[b]{\linewidth}\raggedleft
\ldots{}
\end{minipage} & \begin{minipage}[b]{\linewidth}\raggedleft
Black
\end{minipage} & \begin{minipage}[b]{\linewidth}\raggedleft
Latino
\end{minipage} & \begin{minipage}[b]{\linewidth}\raggedleft
Other
\end{minipage} & \begin{minipage}[b]{\linewidth}\raggedleft
Age
\end{minipage} & \begin{minipage}[b]{\linewidth}\raggedleft
AgeSq
\end{minipage} & \begin{minipage}[b]{\linewidth}\raggedleft
RecMin
\end{minipage} & \begin{minipage}[b]{\linewidth}\raggedleft
Z\_1
\end{minipage} & \begin{minipage}[b]{\linewidth}\raggedleft
Z\_2
\end{minipage} \\
\midrule()
\endhead
Pattern 1 & 1.609438 & 0 & 0 & 0 & 1.609438 & 0 & 0 & 0 & 0 & 0 &
2.302585 & 0 \\
Pattern 2 & 1.609438 & 0 & 0 & 0 & 1.609438 & 0 & 0 & 0 & 0 & 0 &
2.302585 & 0 \\
Pattern 3 & 1.609438 & 0 & 0 & 0 & 1.609438 & 0 & 0 & 0 & 0 & 0 &
2.302585 & 0 \\
Pattern 4 & 1.609438 & 0 & 0 & 0 & 1.609438 & 0 & 0 & 0 & 0 & 0 &
2.302585 & 0 \\
Pattern 5 & 1.609438 & 0 & 0 & 0 & 1.609438 & 0 & 0 & 0 & 0 & 0 &
2.302585 & 0 \\
Pattern 6 & 1.609438 & 0 & 0 & 0 & 1.609438 & 0 & 0 & 0 & 0 & 0 &
2.302585 & 0 \\
Pattern 7 & 1.609438 & 0 & 0 & 0 & 1.609438 & 0 & 0 & 0 & 0 & 0 &
2.302585 & 0 \\
Pattern 8 & 1.609438 & 0 & 0 & 0 & 1.609438 & 0 & 0 & 0 & 0 & 0 &
2.302585 & 0 \\
\bottomrule()
\caption{\label{tbl-mnar-under}The parameters for the missingness model
for CCA under-estimates with MNAR missing values.}\tabularnewline
\end{longtable}

To under-estimate the effect, missingness in race is again conditional
on race being Black and incarceration, so we set the parameters of the
model to be \(\exp(\beta) = (5, 1,... ,1)' \in \mathbb{R}^{88}\) and
\(\exp(\gamma) = (5, 1,1,1,1,1, 10, 1)' \in \mathbb{R}^8\). The first
element in \(\beta\) emphasizes missingness based on incarceration. The
first element in \(\gamma\) emphasizes missingness based on being Black
while the seventh element greatly emphasizes missingness for defendants
who aren't incarcerated and are Black corresponding to \(Z_1\).

All parameters set to 1 are irrelevant to the missingness model since
the log of 1 is 0.

\hypertarget{sec-mar}{%
\subsection{MAR - Simulation Parameters}\label{sec-mar}}

We used similar parameter settings for the MAR simulations as we did for
the MNAR, but now missingness based on race is only for patterns 1,3,5,
and 7 which are no missing variables, missing in age, missing in
recommended minimum, and missing in both age and recommended minimum.
This maintains the MAR assumption since the missingness isn't contingent
on race for the patterns where race is missing. However, since we are
performing the analyses with complete case analysis, we are still
systematically excluding data from our analysis in such a way that will
dramatically bias the results based on the values of race and
incarceration.

\hypertarget{tbl-mar-over}{}
\begin{longtable}[]{@{}
  >{\raggedright\arraybackslash}p{(\columnwidth - 24\tabcolsep) * \real{0.1176}}
  >{\raggedleft\arraybackslash}p{(\columnwidth - 24\tabcolsep) * \real{0.1059}}
  >{\raggedleft\arraybackslash}p{(\columnwidth - 24\tabcolsep) * \real{0.0471}}
  >{\raggedleft\arraybackslash}p{(\columnwidth - 24\tabcolsep) * \real{0.0471}}
  >{\raggedleft\arraybackslash}p{(\columnwidth - 24\tabcolsep) * \real{0.0471}}
  >{\raggedleft\arraybackslash}p{(\columnwidth - 24\tabcolsep) * \real{0.1176}}
  >{\raggedleft\arraybackslash}p{(\columnwidth - 24\tabcolsep) * \real{0.0824}}
  >{\raggedleft\arraybackslash}p{(\columnwidth - 24\tabcolsep) * \real{0.0706}}
  >{\raggedleft\arraybackslash}p{(\columnwidth - 24\tabcolsep) * \real{0.0471}}
  >{\raggedleft\arraybackslash}p{(\columnwidth - 24\tabcolsep) * \real{0.0706}}
  >{\raggedleft\arraybackslash}p{(\columnwidth - 24\tabcolsep) * \real{0.0824}}
  >{\raggedleft\arraybackslash}p{(\columnwidth - 24\tabcolsep) * \real{0.1176}}
  >{\raggedleft\arraybackslash}p{(\columnwidth - 24\tabcolsep) * \real{0.0471}}@{}}
\toprule()
\begin{minipage}[b]{\linewidth}\raggedright
\end{minipage} & \begin{minipage}[b]{\linewidth}\raggedleft
Incar
\end{minipage} & \begin{minipage}[b]{\linewidth}\raggedleft
OGS
\end{minipage} & \begin{minipage}[b]{\linewidth}\raggedleft
PRS
\end{minipage} & \begin{minipage}[b]{\linewidth}\raggedleft
\ldots{}
\end{minipage} & \begin{minipage}[b]{\linewidth}\raggedleft
Black
\end{minipage} & \begin{minipage}[b]{\linewidth}\raggedleft
Latino
\end{minipage} & \begin{minipage}[b]{\linewidth}\raggedleft
Other
\end{minipage} & \begin{minipage}[b]{\linewidth}\raggedleft
Age
\end{minipage} & \begin{minipage}[b]{\linewidth}\raggedleft
AgeSq
\end{minipage} & \begin{minipage}[b]{\linewidth}\raggedleft
RecMin
\end{minipage} & \begin{minipage}[b]{\linewidth}\raggedleft
Z\_1
\end{minipage} & \begin{minipage}[b]{\linewidth}\raggedleft
Z\_2
\end{minipage} \\
\midrule()
\endfirsthead
\toprule()
\begin{minipage}[b]{\linewidth}\raggedright
\end{minipage} & \begin{minipage}[b]{\linewidth}\raggedleft
Incar
\end{minipage} & \begin{minipage}[b]{\linewidth}\raggedleft
OGS
\end{minipage} & \begin{minipage}[b]{\linewidth}\raggedleft
PRS
\end{minipage} & \begin{minipage}[b]{\linewidth}\raggedleft
\ldots{}
\end{minipage} & \begin{minipage}[b]{\linewidth}\raggedleft
Black
\end{minipage} & \begin{minipage}[b]{\linewidth}\raggedleft
Latino
\end{minipage} & \begin{minipage}[b]{\linewidth}\raggedleft
Other
\end{minipage} & \begin{minipage}[b]{\linewidth}\raggedleft
Age
\end{minipage} & \begin{minipage}[b]{\linewidth}\raggedleft
AgeSq
\end{minipage} & \begin{minipage}[b]{\linewidth}\raggedleft
RecMin
\end{minipage} & \begin{minipage}[b]{\linewidth}\raggedleft
Z\_1
\end{minipage} & \begin{minipage}[b]{\linewidth}\raggedleft
Z\_2
\end{minipage} \\
\midrule()
\endhead
Pattern 1 & 2.302585 & 0 & 0 & 0 & -2.302585 & 0 & 0 & 0 & 0 & 0 &
-11.51293 & 0 \\
Pattern 2 & 2.302585 & 0 & 0 & 0 & 0.000000 & 0 & 0 & 0 & 0 & 0 &
0.00000 & 0 \\
Pattern 3 & 2.302585 & 0 & 0 & 0 & -3.453878 & 0 & 0 & 0 & 0 & 0 &
-17.26939 & 0 \\
Pattern 4 & 2.302585 & 0 & 0 & 0 & 0.000000 & 0 & 0 & 0 & 0 & 0 &
0.00000 & 0 \\
Pattern 5 & 2.302585 & 0 & 0 & 0 & -3.453878 & 0 & 0 & 0 & 0 & 0 &
-17.26939 & 0 \\
Pattern 6 & 2.302585 & 0 & 0 & 0 & 0.000000 & 0 & 0 & 0 & 0 & 0 &
0.00000 & 0 \\
Pattern 7 & 2.302585 & 0 & 0 & 0 & -2.302585 & 0 & 0 & 0 & 0 & 0 &
-11.51293 & 0 \\
Pattern 8 & 2.302585 & 0 & 0 & 0 & 0.000000 & 0 & 0 & 0 & 0 & 0 &
0.00000 & 0 \\
\bottomrule()
\caption{\label{tbl-mar-over}The parameters for the missingness model
for CCA over-estimates with MAR missing values.}\tabularnewline
\end{longtable}

To get an over-estimated race effect estimate with CCA, I de-emphasize
missingness based on \(Z_1\) for patterns where race is observed.

\hypertarget{tbl-mar-under}{}
\begin{longtable}[]{@{}
  >{\raggedright\arraybackslash}p{(\columnwidth - 24\tabcolsep) * \real{0.1205}}
  >{\raggedleft\arraybackslash}p{(\columnwidth - 24\tabcolsep) * \real{0.0723}}
  >{\raggedleft\arraybackslash}p{(\columnwidth - 24\tabcolsep) * \real{0.0482}}
  >{\raggedleft\arraybackslash}p{(\columnwidth - 24\tabcolsep) * \real{0.0482}}
  >{\raggedleft\arraybackslash}p{(\columnwidth - 24\tabcolsep) * \real{0.0482}}
  >{\raggedleft\arraybackslash}p{(\columnwidth - 24\tabcolsep) * \real{0.0723}}
  >{\raggedleft\arraybackslash}p{(\columnwidth - 24\tabcolsep) * \real{0.0843}}
  >{\raggedleft\arraybackslash}p{(\columnwidth - 24\tabcolsep) * \real{0.0723}}
  >{\raggedleft\arraybackslash}p{(\columnwidth - 24\tabcolsep) * \real{0.0482}}
  >{\raggedleft\arraybackslash}p{(\columnwidth - 24\tabcolsep) * \real{0.0723}}
  >{\raggedleft\arraybackslash}p{(\columnwidth - 24\tabcolsep) * \real{0.0843}}
  >{\raggedleft\arraybackslash}p{(\columnwidth - 24\tabcolsep) * \real{0.1084}}
  >{\raggedleft\arraybackslash}p{(\columnwidth - 24\tabcolsep) * \real{0.1205}}@{}}
\toprule()
\begin{minipage}[b]{\linewidth}\raggedright
\end{minipage} & \begin{minipage}[b]{\linewidth}\raggedleft
Incar
\end{minipage} & \begin{minipage}[b]{\linewidth}\raggedleft
OGS
\end{minipage} & \begin{minipage}[b]{\linewidth}\raggedleft
PRS
\end{minipage} & \begin{minipage}[b]{\linewidth}\raggedleft
\ldots{}
\end{minipage} & \begin{minipage}[b]{\linewidth}\raggedleft
Black
\end{minipage} & \begin{minipage}[b]{\linewidth}\raggedleft
Latino
\end{minipage} & \begin{minipage}[b]{\linewidth}\raggedleft
Other
\end{minipage} & \begin{minipage}[b]{\linewidth}\raggedleft
Age
\end{minipage} & \begin{minipage}[b]{\linewidth}\raggedleft
AgeSq
\end{minipage} & \begin{minipage}[b]{\linewidth}\raggedleft
RecMin
\end{minipage} & \begin{minipage}[b]{\linewidth}\raggedleft
Z\_1
\end{minipage} & \begin{minipage}[b]{\linewidth}\raggedleft
Z\_2
\end{minipage} \\
\midrule()
\endfirsthead
\toprule()
\begin{minipage}[b]{\linewidth}\raggedright
\end{minipage} & \begin{minipage}[b]{\linewidth}\raggedleft
Incar
\end{minipage} & \begin{minipage}[b]{\linewidth}\raggedleft
OGS
\end{minipage} & \begin{minipage}[b]{\linewidth}\raggedleft
PRS
\end{minipage} & \begin{minipage}[b]{\linewidth}\raggedleft
\ldots{}
\end{minipage} & \begin{minipage}[b]{\linewidth}\raggedleft
Black
\end{minipage} & \begin{minipage}[b]{\linewidth}\raggedleft
Latino
\end{minipage} & \begin{minipage}[b]{\linewidth}\raggedleft
Other
\end{minipage} & \begin{minipage}[b]{\linewidth}\raggedleft
Age
\end{minipage} & \begin{minipage}[b]{\linewidth}\raggedleft
AgeSq
\end{minipage} & \begin{minipage}[b]{\linewidth}\raggedleft
RecMin
\end{minipage} & \begin{minipage}[b]{\linewidth}\raggedleft
Z\_1
\end{minipage} & \begin{minipage}[b]{\linewidth}\raggedleft
Z\_2
\end{minipage} \\
\midrule()
\endhead
Pattern 1 & 0 & 0 & 0 & 0 & 0 & 0 & 0 & 0 & 0 & 0 & 3.401197 &
-2.302585 \\
Pattern 2 & 0 & 0 & 0 & 0 & 0 & 0 & 0 & 0 & 0 & 0 & 0.000000 &
0.000000 \\
Pattern 3 & 0 & 0 & 0 & 0 & 0 & 0 & 0 & 0 & 0 & 0 & 5.101796 &
-3.453878 \\
Pattern 4 & 0 & 0 & 0 & 0 & 0 & 0 & 0 & 0 & 0 & 0 & 0.000000 &
0.000000 \\
Pattern 5 & 0 & 0 & 0 & 0 & 0 & 0 & 0 & 0 & 0 & 0 & 5.101796 &
-3.453878 \\
Pattern 6 & 0 & 0 & 0 & 0 & 0 & 0 & 0 & 0 & 0 & 0 & 0.000000 &
0.000000 \\
Pattern 7 & 0 & 0 & 0 & 0 & 0 & 0 & 0 & 0 & 0 & 0 & 3.401197 &
-2.302585 \\
Pattern 8 & 0 & 0 & 0 & 0 & 0 & 0 & 0 & 0 & 0 & 0 & 0.000000 &
0.000000 \\
\bottomrule()
\caption{\label{tbl-mar-under}The parameters for the missingness model
for CCA under-estimates with MAR missing values.}\tabularnewline
\end{longtable}

To under-estimate the effect with CCA, we emphasize \(Z_1\) and
de-emphasize \(Z_2\) for patterns where race is observed.



\end{document}
