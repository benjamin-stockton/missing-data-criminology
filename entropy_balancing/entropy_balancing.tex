% Options for packages loaded elsewhere
\PassOptionsToPackage{unicode}{hyperref}
\PassOptionsToPackage{hyphens}{url}
\PassOptionsToPackage{dvipsnames,svgnames,x11names}{xcolor}
%
\documentclass[
  letterpaper,
  DIV=11,
  numbers=noendperiod]{scrartcl}

\usepackage{amsmath,amssymb}
\usepackage{lmodern}
\usepackage{iftex}
\ifPDFTeX
  \usepackage[T1]{fontenc}
  \usepackage[utf8]{inputenc}
  \usepackage{textcomp} % provide euro and other symbols
\else % if luatex or xetex
  \usepackage{unicode-math}
  \defaultfontfeatures{Scale=MatchLowercase}
  \defaultfontfeatures[\rmfamily]{Ligatures=TeX,Scale=1}
\fi
% Use upquote if available, for straight quotes in verbatim environments
\IfFileExists{upquote.sty}{\usepackage{upquote}}{}
\IfFileExists{microtype.sty}{% use microtype if available
  \usepackage[]{microtype}
  \UseMicrotypeSet[protrusion]{basicmath} % disable protrusion for tt fonts
}{}
\makeatletter
\@ifundefined{KOMAClassName}{% if non-KOMA class
  \IfFileExists{parskip.sty}{%
    \usepackage{parskip}
  }{% else
    \setlength{\parindent}{0pt}
    \setlength{\parskip}{6pt plus 2pt minus 1pt}}
}{% if KOMA class
  \KOMAoptions{parskip=half}}
\makeatother
\usepackage{xcolor}
\usepackage[top=30mm,left=30mm]{geometry}
\setlength{\emergencystretch}{3em} % prevent overfull lines
\setcounter{secnumdepth}{5}
% Make \paragraph and \subparagraph free-standing
\ifx\paragraph\undefined\else
  \let\oldparagraph\paragraph
  \renewcommand{\paragraph}[1]{\oldparagraph{#1}\mbox{}}
\fi
\ifx\subparagraph\undefined\else
  \let\oldsubparagraph\subparagraph
  \renewcommand{\subparagraph}[1]{\oldsubparagraph{#1}\mbox{}}
\fi

\usepackage{color}
\usepackage{fancyvrb}
\newcommand{\VerbBar}{|}
\newcommand{\VERB}{\Verb[commandchars=\\\{\}]}
\DefineVerbatimEnvironment{Highlighting}{Verbatim}{commandchars=\\\{\}}
% Add ',fontsize=\small' for more characters per line
\usepackage{framed}
\definecolor{shadecolor}{RGB}{241,243,245}
\newenvironment{Shaded}{\begin{snugshade}}{\end{snugshade}}
\newcommand{\AlertTok}[1]{\textcolor[rgb]{0.68,0.00,0.00}{#1}}
\newcommand{\AnnotationTok}[1]{\textcolor[rgb]{0.37,0.37,0.37}{#1}}
\newcommand{\AttributeTok}[1]{\textcolor[rgb]{0.40,0.45,0.13}{#1}}
\newcommand{\BaseNTok}[1]{\textcolor[rgb]{0.68,0.00,0.00}{#1}}
\newcommand{\BuiltInTok}[1]{\textcolor[rgb]{0.00,0.23,0.31}{#1}}
\newcommand{\CharTok}[1]{\textcolor[rgb]{0.13,0.47,0.30}{#1}}
\newcommand{\CommentTok}[1]{\textcolor[rgb]{0.37,0.37,0.37}{#1}}
\newcommand{\CommentVarTok}[1]{\textcolor[rgb]{0.37,0.37,0.37}{\textit{#1}}}
\newcommand{\ConstantTok}[1]{\textcolor[rgb]{0.56,0.35,0.01}{#1}}
\newcommand{\ControlFlowTok}[1]{\textcolor[rgb]{0.00,0.23,0.31}{#1}}
\newcommand{\DataTypeTok}[1]{\textcolor[rgb]{0.68,0.00,0.00}{#1}}
\newcommand{\DecValTok}[1]{\textcolor[rgb]{0.68,0.00,0.00}{#1}}
\newcommand{\DocumentationTok}[1]{\textcolor[rgb]{0.37,0.37,0.37}{\textit{#1}}}
\newcommand{\ErrorTok}[1]{\textcolor[rgb]{0.68,0.00,0.00}{#1}}
\newcommand{\ExtensionTok}[1]{\textcolor[rgb]{0.00,0.23,0.31}{#1}}
\newcommand{\FloatTok}[1]{\textcolor[rgb]{0.68,0.00,0.00}{#1}}
\newcommand{\FunctionTok}[1]{\textcolor[rgb]{0.28,0.35,0.67}{#1}}
\newcommand{\ImportTok}[1]{\textcolor[rgb]{0.00,0.46,0.62}{#1}}
\newcommand{\InformationTok}[1]{\textcolor[rgb]{0.37,0.37,0.37}{#1}}
\newcommand{\KeywordTok}[1]{\textcolor[rgb]{0.00,0.23,0.31}{#1}}
\newcommand{\NormalTok}[1]{\textcolor[rgb]{0.00,0.23,0.31}{#1}}
\newcommand{\OperatorTok}[1]{\textcolor[rgb]{0.37,0.37,0.37}{#1}}
\newcommand{\OtherTok}[1]{\textcolor[rgb]{0.00,0.23,0.31}{#1}}
\newcommand{\PreprocessorTok}[1]{\textcolor[rgb]{0.68,0.00,0.00}{#1}}
\newcommand{\RegionMarkerTok}[1]{\textcolor[rgb]{0.00,0.23,0.31}{#1}}
\newcommand{\SpecialCharTok}[1]{\textcolor[rgb]{0.37,0.37,0.37}{#1}}
\newcommand{\SpecialStringTok}[1]{\textcolor[rgb]{0.13,0.47,0.30}{#1}}
\newcommand{\StringTok}[1]{\textcolor[rgb]{0.13,0.47,0.30}{#1}}
\newcommand{\VariableTok}[1]{\textcolor[rgb]{0.07,0.07,0.07}{#1}}
\newcommand{\VerbatimStringTok}[1]{\textcolor[rgb]{0.13,0.47,0.30}{#1}}
\newcommand{\WarningTok}[1]{\textcolor[rgb]{0.37,0.37,0.37}{\textit{#1}}}

\providecommand{\tightlist}{%
  \setlength{\itemsep}{0pt}\setlength{\parskip}{0pt}}\usepackage{longtable,booktabs,array}
\usepackage{calc} % for calculating minipage widths
% Correct order of tables after \paragraph or \subparagraph
\usepackage{etoolbox}
\makeatletter
\patchcmd\longtable{\par}{\if@noskipsec\mbox{}\fi\par}{}{}
\makeatother
% Allow footnotes in longtable head/foot
\IfFileExists{footnotehyper.sty}{\usepackage{footnotehyper}}{\usepackage{footnote}}
\makesavenoteenv{longtable}
\usepackage{graphicx}
\makeatletter
\def\maxwidth{\ifdim\Gin@nat@width>\linewidth\linewidth\else\Gin@nat@width\fi}
\def\maxheight{\ifdim\Gin@nat@height>\textheight\textheight\else\Gin@nat@height\fi}
\makeatother
% Scale images if necessary, so that they will not overflow the page
% margins by default, and it is still possible to overwrite the defaults
% using explicit options in \includegraphics[width, height, ...]{}
\setkeys{Gin}{width=\maxwidth,height=\maxheight,keepaspectratio}
% Set default figure placement to htbp
\makeatletter
\def\fps@figure{htbp}
\makeatother
\newlength{\cslhangindent}
\setlength{\cslhangindent}{1.5em}
\newlength{\csllabelwidth}
\setlength{\csllabelwidth}{3em}
\newlength{\cslentryspacingunit} % times entry-spacing
\setlength{\cslentryspacingunit}{\parskip}
\newenvironment{CSLReferences}[2] % #1 hanging-ident, #2 entry spacing
 {% don't indent paragraphs
  \setlength{\parindent}{0pt}
  % turn on hanging indent if param 1 is 1
  \ifodd #1
  \let\oldpar\par
  \def\par{\hangindent=\cslhangindent\oldpar}
  \fi
  % set entry spacing
  \setlength{\parskip}{#2\cslentryspacingunit}
 }%
 {}
\usepackage{calc}
\newcommand{\CSLBlock}[1]{#1\hfill\break}
\newcommand{\CSLLeftMargin}[1]{\parbox[t]{\csllabelwidth}{#1}}
\newcommand{\CSLRightInline}[1]{\parbox[t]{\linewidth - \csllabelwidth}{#1}\break}
\newcommand{\CSLIndent}[1]{\hspace{\cslhangindent}#1}

\KOMAoption{captions}{tablesignature}
\makeatletter
\makeatother
\makeatletter
\makeatother
\makeatletter
\@ifpackageloaded{caption}{}{\usepackage{caption}}
\AtBeginDocument{%
\ifdefined\contentsname
  \renewcommand*\contentsname{Table of contents}
\else
  \newcommand\contentsname{Table of contents}
\fi
\ifdefined\listfigurename
  \renewcommand*\listfigurename{List of Figures}
\else
  \newcommand\listfigurename{List of Figures}
\fi
\ifdefined\listtablename
  \renewcommand*\listtablename{List of Tables}
\else
  \newcommand\listtablename{List of Tables}
\fi
\ifdefined\figurename
  \renewcommand*\figurename{Figure}
\else
  \newcommand\figurename{Figure}
\fi
\ifdefined\tablename
  \renewcommand*\tablename{Table}
\else
  \newcommand\tablename{Table}
\fi
}
\@ifpackageloaded{float}{}{\usepackage{float}}
\floatstyle{ruled}
\@ifundefined{c@chapter}{\newfloat{codelisting}{h}{lop}}{\newfloat{codelisting}{h}{lop}[chapter]}
\floatname{codelisting}{Listing}
\newcommand*\listoflistings{\listof{codelisting}{List of Listings}}
\makeatother
\makeatletter
\@ifpackageloaded{caption}{}{\usepackage{caption}}
\@ifpackageloaded{subcaption}{}{\usepackage{subcaption}}
\makeatother
\makeatletter
\@ifpackageloaded{tcolorbox}{}{\usepackage[many]{tcolorbox}}
\makeatother
\makeatletter
\@ifundefined{shadecolor}{\definecolor{shadecolor}{rgb}{.97, .97, .97}}
\makeatother
\makeatletter
\makeatother
\ifLuaTeX
  \usepackage{selnolig}  % disable illegal ligatures
\fi
\IfFileExists{bookmark.sty}{\usepackage{bookmark}}{\usepackage{hyperref}}
\IfFileExists{xurl.sty}{\usepackage{xurl}}{} % add URL line breaks if available
\urlstyle{same} % disable monospaced font for URLs
\hypersetup{
  pdftitle={Entropy Balancing},
  pdfauthor={Benjamin Stockton},
  colorlinks=true,
  linkcolor={blue},
  filecolor={Maroon},
  citecolor={Blue},
  urlcolor={Blue},
  pdfcreator={LaTeX via pandoc}}

\title{Entropy Balancing}
\author{Benjamin Stockton}
\date{}

\begin{document}
\maketitle
\ifdefined\Shaded\renewenvironment{Shaded}{\begin{tcolorbox}[sharp corners, frame hidden, interior hidden, boxrule=0pt, enhanced, borderline west={3pt}{0pt}{shadecolor}, breakable]}{\end{tcolorbox}}\fi

\renewcommand*\contentsname{Table of contents}
{
\hypersetup{linkcolor=}
\setcounter{tocdepth}{3}
\tableofcontents
}
\hypertarget{entropy-balancing}{%
\section{Entropy Balancing}\label{entropy-balancing}}

To make causal inferences in observational data, the observed
Y(1)\textbar D=1 and counterfactual Y(0)\textbar D=1 are compared to get
the Population Average Treatment Effect on Treated (PATT) defined as
\(\tau = E(Y(1)|D=1) - E(Y(0) | D=1)\). In experimental studies where
treatment assignment is independent of the potential outcomes,
approximate the second expectation by \(E(Y(0) | D=0)\) i.e.~the mean of
the control group. In observational studies, if we can ``assume
ignorable treatment assignment and overlap'', then \(Y(0) \perp D|X\)
and \(P(D = 1| X=x) \leq 1\) for all \(x\) in the support of
\(f_{X|D=1}\). That means that if the confounding covariates are similar
on both the treatment and control groups, we can estimate the
counterfactual average outcome by
\(\tau = E(Y|D=1) = \int E(Y|X=x, D=0) f_{X|D=1}(x) dx.\)

To make causal inferences in observational data, the observed
\(Y(1)|D=1\) and counterfactual \(Y(0)|D=1\) are compared to get the
Population Average Treatment Effect on Treated (PATT) defined as
\(\tau = E(Y(1)|D=1) - E(Y(0) | D=1)\). In experimental studies where
treatment assignment is independent of the potential outcomes,
approximate the second expectation by \(E(Y(0) | D=0)\) i.e.~the mean of
the control group. In observational studies, if we can ``assume
ignorable treatment assignment and overlap'', then \(Y(0) \perp D|X\)
and \(P(D = 1| X=x) \leq 1\) for all \(x\) in the support of
\(f_{X|D=1}\). That means that if the confounding covariates are similar
on both the treatment and control groups, we can estimate the
counterfactual average outcome by
\(\tau = E(Y|D=1) = \int E(Y|X=x, D=0) f_{X|D=1}(x) dx.\)

\begin{quote}
Notice that the last term in this expression is equal to the covariate
adjusted mean, that is, the estimated mean of \(Y\) in the source
population if its covariates were distributed as in the target
population.
\end{quote}

\begin{itemize}
\tightlist
\item
  p. 28 (Hainmueller 2012)
\end{itemize}

\begin{quote}
{[}Rosenbaum and Rubin (1983){]} showed that the multivariate matching
pre-processing problem{]} can be reduced to a single dimension if the
counterfactual mean can be identified as
\(E(Y(0)|D=1) = \int E(Y|p(X) = \rho, D=0) f_{p|D=1}(\rho)d\rho\) where
\(f_{p|D=1}\) is the dist. of the propensity score
\(p(x) = P(D=1 | X=x)\) in the target population.
\end{quote}

\begin{itemize}
\tightlist
\item
  p. 28 (Hainmueller 2012)
\end{itemize}

Propensity score weighting is performed by using a binary response
(logit/probit) regression to estimate a probability \(p_i\) to be in the
treatment given the covariates. These are converted to weights \(d_i\)
by the inverse of the link function \(d_i = p_i / (1-p_i)\) so that the
counterfactual mean is estimated as
\(\widehat{E(Y(0)|D=1)} = \frac{\sum_{i|D=0}Y_i d_i}{\sum_{i|D=0}d_i}.\)

Drawbacks of PSW:

\begin{itemize}
\item
  The true propensity score is valuable because it balances the
  covariate distributions of the two groups, but is unknown and
  difficult to accurately estimate.
\item
  Mis-specified scores can lead to biased estimation of TE
\item
  Practitioners often iterate between weighting and matching, modeling
  the PS, and then evaluating the balance until a suitable balance is
  achieved. Imai, King and Stuart (2008) call this the ``propensity
  score tautology''.

  \begin{itemize}
  \tightlist
  \item
    Despite this balance often isn't achieved and can make balance worse
    among confounders.
  \end{itemize}
\end{itemize}

\hypertarget{entropy-balancing-procedure}{%
\subsection{Entropy Balancing
Procedure:}\label{entropy-balancing-procedure}}

\textbf{Goal:} Estimate \(\tau = E(Y(1) |D=1) - E(Y(0) | D=1).\)

The counterfactual mean could be estimated by
\(\widehat{E(Y(0)|D=1)}=\frac{\sum_{i|D=0} Y_i w_i}{\sum_{i|D=0} w_i}.\)
The weights are found by minimizing \(H(w) = \sum_{i|D=0} h(w_i)\)
subject to \(\sum_{i|D=0} w_i c_{ri}(X_i) = m_r\) for \(r =1,…,R\) and
\(\sum_{i|D=0} w_i = 1\) and \(w_i \geq 0\) for all \(i\) given
\(D_i = 0\) where \(h(.)\) is a distance metric and \(c_{ri}(X_i)=m_r\)
describes the set of \(R\) balance constraints imposed by the covariate
moments of the re-weighted control group.

Authors choose to use \(h(w_i)=w_i\log(w_i/q_i)\) the Kullback entropy
divergence.

In conventional PSW, the researcher (1) estimates \(d_i\) then (2)
checks if the weights balance the covariate distribution. Entropy
balancing reverses the approach by obtaining weights by minimizing a
linear equation with constraints that guarantee balance while remaining
close to the uniform weights to guarantee efficiency.

Optimization is performed using Lagrangian Multipliers.

\hypertarget{main-issues-with-entropy-balancing}{%
\subsubsection{3 Main Issues with Entropy
Balancing:}\label{main-issues-with-entropy-balancing}}

\begin{enumerate}
\def\labelenumi{\arabic{enumi}.}
\tightlist
\item
  ``No weighting solution exists if the balance constraints are
  inconsistent.'' Easy to avoid as constraints are researcher imposed.
\item
  ``Balance constraints are consistent but there exists no set of
  positive weights to actually satisfy the constraints.'' For example,
  there's heavy imbalance in a binary categorical variable between the
  treatment and control groups. ``If there aren't enough controls that
  look anything like the treated units, then the existing data do not
  contain sufficient information to reliably infer the counterfactual.''
\item
  ``A solution exists, but due to limited overlap, the solution involves
  an extreme adjustment to the weights of some control units.'' Only a
  few units receive relatively large weights and all others are set near
  0, then the variance will increase and effective sample size is small.
\end{enumerate}

Issues 2 and 3 are also relevant to most pre-processing that involves
balancing including propensity score weighting.

\hypertarget{entropy-balancing-in-sentencing-research}{%
\subsection{Entropy Balancing in Sentencing
Research}\label{entropy-balancing-in-sentencing-research}}

\hypertarget{implementation-by-ebal}{%
\section{Implementation by ebal}\label{implementation-by-ebal}}

\hypertarget{simulated-data}{%
\subsection{Simulated Data}\label{simulated-data}}

First, we'll demonstrate entropy balancing in a logistic regression
setting to estimate the race effect on the simulated data. This data
contains the same number of observations as the PCS data on the same
variables which have been simulated independently, so a priori we should
expect the marginal means of the variables to be the same for ``Black''
and ``Non-Black'' simulated defendants. The simulated data set is
complete.

\begin{Shaded}
\begin{Highlighting}[]
\FunctionTok{library}\NormalTok{(ebal)}
\FunctionTok{library}\NormalTok{(dplyr)}

\NormalTok{dat }\OtherTok{\textless{}{-}} \FunctionTok{read.csv}\NormalTok{(}\StringTok{"../Data/simulated\_data.csv"}\NormalTok{)}

\NormalTok{dat}\SpecialCharTok{$}\NormalTok{YEAR }\OtherTok{\textless{}{-}} \FunctionTok{factor}\NormalTok{(dat}\SpecialCharTok{$}\NormalTok{YEAR)}
\NormalTok{dat}\SpecialCharTok{$}\NormalTok{COUNTY }\OtherTok{\textless{}{-}} \FunctionTok{factor}\NormalTok{(dat}\SpecialCharTok{$}\NormalTok{COUNTY)}

\NormalTok{dat}\SpecialCharTok{$}\NormalTok{OFF\_RACER }\OtherTok{\textless{}{-}} \FunctionTok{factor}\NormalTok{(dat}\SpecialCharTok{$}\NormalTok{OFF\_RACER, }\AttributeTok{levels =} \FunctionTok{c}\NormalTok{(}\StringTok{"WHITE"}\NormalTok{, }\StringTok{"BLACK"}\NormalTok{, }\StringTok{"LATINO"}\NormalTok{, }\StringTok{"OTHER"}\NormalTok{))}

\NormalTok{dat }\SpecialCharTok{\%\textgreater{}\%}
    \FunctionTok{select}\NormalTok{(}\SpecialCharTok{{-}}\FunctionTok{c}\NormalTok{(INCAR, YEAR, COUNTY, OFF\_RACER)) }\OtherTok{{-}\textgreater{}}\NormalTok{ X}
\NormalTok{X }\OtherTok{\textless{}{-}} \FunctionTok{model.matrix}\NormalTok{(}\SpecialCharTok{\textasciitilde{}}\NormalTok{., }\AttributeTok{data =}\NormalTok{ X)[,}\SpecialCharTok{{-}}\DecValTok{1}\NormalTok{]}

\NormalTok{treatment }\OtherTok{\textless{}{-}} \FunctionTok{ifelse}\NormalTok{(dat}\SpecialCharTok{$}\NormalTok{OFF\_RACER }\SpecialCharTok{==} \StringTok{"BLACK"}\NormalTok{, }\ConstantTok{TRUE}\NormalTok{, }\ConstantTok{FALSE}\NormalTok{)}

\NormalTok{eb.out }\OtherTok{\textless{}{-}} \FunctionTok{ebalance}\NormalTok{(}\AttributeTok{Treatment =}\NormalTok{ treatment, }
                   \AttributeTok{X =}\NormalTok{ X, }\AttributeTok{print.level =} \DecValTok{2}\NormalTok{)}
\end{Highlighting}
\end{Shaded}

\begin{verbatim}
Iteration 1 maximum deviation is = 211656 
Iteration 2 maximum deviation is = 9074 
Converged within tolerance 
Converged within tolerance 
\end{verbatim}

Entropy balancing seems to be performed very quickly. Next we check the
marginal means for each of the covariates

\begin{Shaded}
\begin{Highlighting}[]
\NormalTok{c1 }\OtherTok{\textless{}{-}} \FunctionTok{apply}\NormalTok{(X[treatment,], }\DecValTok{2}\NormalTok{, mean)}


\NormalTok{c2 }\OtherTok{\textless{}{-}} \FunctionTok{apply}\NormalTok{(X[}\SpecialCharTok{!}\NormalTok{treatment,], }\DecValTok{2}\NormalTok{, weighted.mean, }\AttributeTok{w =}\NormalTok{ eb.out}\SpecialCharTok{$}\NormalTok{w)}


\NormalTok{c3 }\OtherTok{\textless{}{-}} \FunctionTok{apply}\NormalTok{(X[}\SpecialCharTok{!}\NormalTok{treatment,], }\DecValTok{2}\NormalTok{, mean)}

\NormalTok{X.means }\OtherTok{\textless{}{-}} \FunctionTok{bind\_rows}\NormalTok{(c1, c2, c3) }\SpecialCharTok{\%\textgreater{}\%} \FunctionTok{as.data.frame}\NormalTok{()}
\FunctionTok{rownames}\NormalTok{(X.means) }\OtherTok{\textless{}{-}} \FunctionTok{c}\NormalTok{(}\StringTok{"Black"}\NormalTok{, }\StringTok{"Non{-}Black {-} EB"}\NormalTok{, }\StringTok{"Non{-}Black {-} Unbalanced"}\NormalTok{)}
\FunctionTok{round}\NormalTok{(}\FunctionTok{t}\NormalTok{(X.means), }\DecValTok{4}\NormalTok{)}
\end{Highlighting}
\end{Shaded}

\begin{verbatim}
                      Black Non-Black - EB Non-Black - Unbalanced
CRIMETYPEDUI         0.2367         0.2367                 0.2379
CRIMETYPEOTHER       0.1241         0.1241                 0.1230
CRIMETYPEPERSONS     0.1560         0.1560                 0.1568
CRIMETYPEPROPERTY    0.2556         0.2556                 0.2548
OGS                  3.4283         3.4283                 3.4184
TRIALYes             0.9782         0.9782                 0.9786
MALEMale             0.7731         0.7731                 0.7739
PRS4/5               0.1813         0.1813                 0.1820
PRSNone              0.4856         0.4856                 0.4859
PRSREVOC/RFEL        0.0300         0.0300                 0.0300
DOSAGE              34.2666        34.2666                34.2765
RECMINYes            0.6701         0.6701                 0.6707
OGSQ                17.9444        17.9444                17.8505
DOSAGEQ           1306.2898      1306.2898              1307.1996
\end{verbatim}

The means are essentially the same between the covariates before
balancing, so it won't have any effect on the actual analysis. This is a
key point regarding use of the modal approach in the simulations.

\begin{Shaded}
\begin{Highlighting}[]
\FunctionTok{c}\NormalTok{(}\FunctionTok{summary}\NormalTok{(eb.out}\SpecialCharTok{$}\NormalTok{w), }\StringTok{"Std. Dev"} \OtherTok{=} \FunctionTok{sd}\NormalTok{(eb.out}\SpecialCharTok{$}\NormalTok{w))}
\end{Highlighting}
\end{Shaded}

\begin{verbatim}
       Min.     1st Qu.      Median        Mean     3rd Qu.        Max. 
0.353880254 0.366262566 0.368087992 0.368239110 0.369965186 0.385803692 
   Std. Dev 
0.002779261 
\end{verbatim}

The weights are essentially uniform still indicating little needs to be
done to balance the covariates.

\begin{Shaded}
\begin{Highlighting}[]
\NormalTok{weights }\OtherTok{\textless{}{-}} \FunctionTok{numeric}\NormalTok{(}\FunctionTok{nrow}\NormalTok{(dat))}
\ControlFlowTok{for}\NormalTok{ (i }\ControlFlowTok{in} \DecValTok{1}\SpecialCharTok{:}\FunctionTok{nrow}\NormalTok{(dat)) \{}
    \ControlFlowTok{if}\NormalTok{ (treatment[i]) \{}
\NormalTok{        weights[i] }\OtherTok{\textless{}{-}} \DecValTok{1}
\NormalTok{    \}}
    \ControlFlowTok{else}\NormalTok{ \{}
\NormalTok{        weights[i] }\OtherTok{\textless{}{-}}\NormalTok{ eb.out}\SpecialCharTok{$}\NormalTok{w[i]}
\NormalTok{    \}}
\NormalTok{\}}

\NormalTok{fit\_sim\_unweighted }\OtherTok{\textless{}{-}} \FunctionTok{summary}\NormalTok{(}\FunctionTok{glm}\NormalTok{(INCAR }\SpecialCharTok{\textasciitilde{}}\NormalTok{ ., }
                          \AttributeTok{data =}\NormalTok{ dat, }
                          \AttributeTok{family =} \FunctionTok{binomial}\NormalTok{(}\AttributeTok{link =} \StringTok{"logit"}\NormalTok{)))}

\NormalTok{fit\_sim\_weighted }\OtherTok{\textless{}{-}} \FunctionTok{summary}\NormalTok{(}\FunctionTok{glm}\NormalTok{(INCAR }\SpecialCharTok{\textasciitilde{}}\NormalTok{ ., }
                          \AttributeTok{data =}\NormalTok{ dat, }
                          \AttributeTok{weights =}\NormalTok{ weights,}
                          \AttributeTok{family =} \FunctionTok{binomial}\NormalTok{(}\AttributeTok{link =} \StringTok{"logit"}\NormalTok{)))}
\end{Highlighting}
\end{Shaded}

\begin{verbatim}
Warning in eval(family$initialize): non-integer #successes in a binomial glm!
\end{verbatim}

\begin{Shaded}
\begin{Highlighting}[]
\FunctionTok{print}\NormalTok{(}\StringTok{"Unweighted Race Effect Estimate:"}\NormalTok{)}
\end{Highlighting}
\end{Shaded}

\begin{verbatim}
[1] "Unweighted Race Effect Estimate:"
\end{verbatim}

\begin{Shaded}
\begin{Highlighting}[]
\NormalTok{fit\_sim\_unweighted}\SpecialCharTok{$}\NormalTok{coefficients[}\StringTok{"OFF\_RACERBLACK"}\NormalTok{,]}
\end{Highlighting}
\end{Shaded}

\begin{verbatim}
    Estimate   Std. Error      z value     Pr(>|z|) 
 0.225326688  0.005659322 39.815138611  0.000000000 
\end{verbatim}

\begin{Shaded}
\begin{Highlighting}[]
\FunctionTok{print}\NormalTok{(}\StringTok{"Weighted Race Effect Estimate:"}\NormalTok{)}
\end{Highlighting}
\end{Shaded}

\begin{verbatim}
[1] "Weighted Race Effect Estimate:"
\end{verbatim}

\begin{Shaded}
\begin{Highlighting}[]
\NormalTok{fit\_sim\_weighted}\SpecialCharTok{$}\NormalTok{coefficients[}\StringTok{"OFF\_RACERBLACK"}\NormalTok{,]}
\end{Highlighting}
\end{Shaded}

\begin{verbatim}
     Estimate    Std. Error       z value      Pr(>|z|) 
 2.246423e-01  7.484070e-03  3.001606e+01 6.057134e-198 
\end{verbatim}

\hypertarget{real-pcs-data}{%
\subsection{Real PCS Data}\label{real-pcs-data}}

\begin{Shaded}
\begin{Highlighting}[]
\NormalTok{pcs }\OtherTok{\textless{}{-}} \FunctionTok{read.csv}\NormalTok{(}\StringTok{"../Data/most\_serious\_sentence\_2010{-}2019\_slim.csv"}\NormalTok{)}

\NormalTok{pcs}\SpecialCharTok{$}\NormalTok{YEAR }\OtherTok{\textless{}{-}} \FunctionTok{factor}\NormalTok{(pcs}\SpecialCharTok{$}\NormalTok{YEAR)}
\NormalTok{pcs}\SpecialCharTok{$}\NormalTok{COUNTY }\OtherTok{\textless{}{-}} \FunctionTok{factor}\NormalTok{(pcs}\SpecialCharTok{$}\NormalTok{COUNTY)}
\NormalTok{pcs}\SpecialCharTok{$}\NormalTok{OFF\_RACER }\OtherTok{\textless{}{-}} \FunctionTok{factor}\NormalTok{(pcs}\SpecialCharTok{$}\NormalTok{OFF\_RACER, }\AttributeTok{levels =} \FunctionTok{c}\NormalTok{(}\StringTok{"WHITE"}\NormalTok{, }\StringTok{"BLACK"}\NormalTok{, }\StringTok{"LATINO"}\NormalTok{, }\StringTok{"OTHER"}\NormalTok{))}

\NormalTok{mice}\SpecialCharTok{::}\FunctionTok{md.pattern}\NormalTok{(pcs, }\AttributeTok{rotate.names =}\NormalTok{ T)}
\end{Highlighting}
\end{Shaded}

\begin{figure}[H]

{\centering \includegraphics{entropy_balancing_files/figure-pdf/unnamed-chunk-5-1.pdf}

}

\end{figure}

\begin{verbatim}
       INCAR CRIMETYPE OGS OGSQ TRIAL MALE COUNTY YEAR PRS DOSAGE DOSAGEQ
834546     1         1   1    1     1    1      1    1   1      1       1
26206      1         1   1    1     1    1      1    1   1      1       1
2076       1         1   1    1     1    1      1    1   1      1       1
34         1         1   1    1     1    1      1    1   1      1       1
1465       1         1   1    1     1    1      1    1   1      0       0
92         1         1   1    1     1    1      1    1   1      0       0
3          1         1   1    1     1    1      1    1   0      1       1
           0         0   0    0     0    0      0    0   3   1557    1557
       RECMIN OFF_RACER      
834546      1         1     0
26206       1         0     1
2076        0         1     1
34          0         0     2
1465        1         1     2
92          1         0     3
3           0         1     2
         2113     26332 31562
\end{verbatim}

\begin{Shaded}
\begin{Highlighting}[]
\NormalTok{pcs }\SpecialCharTok{\%\textgreater{}\%} 
    \FunctionTok{filter}\NormalTok{(}\SpecialCharTok{!}\FunctionTok{is.na}\NormalTok{(OFF\_RACER),}
           \SpecialCharTok{!}\FunctionTok{is.na}\NormalTok{(DOSAGE),}
           \SpecialCharTok{!}\FunctionTok{is.na}\NormalTok{(RECMIN),}
           \SpecialCharTok{!}\FunctionTok{is.na}\NormalTok{(PRS)) }\OtherTok{{-}\textgreater{}}\NormalTok{ pcs.cc}
\NormalTok{pcs.cc }\SpecialCharTok{\%\textgreater{}\%}
    \FunctionTok{select}\NormalTok{(}\SpecialCharTok{{-}}\FunctionTok{c}\NormalTok{(INCAR, YEAR, COUNTY, OFF\_RACER)) }\OtherTok{{-}\textgreater{}}\NormalTok{ X}
\NormalTok{X }\OtherTok{\textless{}{-}} \FunctionTok{model.matrix}\NormalTok{(}\SpecialCharTok{\textasciitilde{}}\NormalTok{., }\AttributeTok{data =}\NormalTok{ X)[,}\SpecialCharTok{{-}}\DecValTok{1}\NormalTok{]}

\NormalTok{treatment }\OtherTok{\textless{}{-}} \FunctionTok{ifelse}\NormalTok{(pcs.cc}\SpecialCharTok{$}\NormalTok{OFF\_RACER }\SpecialCharTok{==} \StringTok{"BLACK"}\NormalTok{, }\ConstantTok{TRUE}\NormalTok{, }\ConstantTok{FALSE}\NormalTok{)}

\NormalTok{eb.pcs }\OtherTok{\textless{}{-}} \FunctionTok{ebalance}\NormalTok{(}\AttributeTok{Treatment =}\NormalTok{ treatment, }
                   \AttributeTok{X =}\NormalTok{ X, }\AttributeTok{print.level =} \DecValTok{2}\NormalTok{)}
\end{Highlighting}
\end{Shaded}

\begin{verbatim}
Iteration 1 maximum deviation is = 28887568 
Iteration 2 maximum deviation is = 26506872 
Iteration 3 maximum deviation is = 23179050 
Iteration 4 maximum deviation is = 18192459 
Iteration 5 maximum deviation is = 8447031 
Iteration 6 maximum deviation is = 222757 
Iteration 7 maximum deviation is = 315.2 
Converged within tolerance 
Converged within tolerance 
\end{verbatim}

\begin{Shaded}
\begin{Highlighting}[]
\NormalTok{c1 }\OtherTok{\textless{}{-}} \FunctionTok{apply}\NormalTok{(X[treatment,], }\DecValTok{2}\NormalTok{, mean)}


\NormalTok{c2 }\OtherTok{\textless{}{-}} \FunctionTok{apply}\NormalTok{(X[}\SpecialCharTok{!}\NormalTok{treatment,], }\DecValTok{2}\NormalTok{, weighted.mean, }\AttributeTok{w =}\NormalTok{ eb.pcs}\SpecialCharTok{$}\NormalTok{w)}


\NormalTok{c3 }\OtherTok{\textless{}{-}} \FunctionTok{apply}\NormalTok{(X[}\SpecialCharTok{!}\NormalTok{treatment,], }\DecValTok{2}\NormalTok{, mean)}

\NormalTok{X.means }\OtherTok{\textless{}{-}} \FunctionTok{bind\_rows}\NormalTok{(c1, c2, c3) }\SpecialCharTok{\%\textgreater{}\%} \FunctionTok{as.data.frame}\NormalTok{()}
\FunctionTok{rownames}\NormalTok{(X.means) }\OtherTok{\textless{}{-}} \FunctionTok{c}\NormalTok{(}\StringTok{"Black"}\NormalTok{, }\StringTok{"Non{-}Black {-} EB"}\NormalTok{, }\StringTok{"Non{-}Black {-} Unbalanced"}\NormalTok{)}
\FunctionTok{round}\NormalTok{(}\FunctionTok{t}\NormalTok{(X.means), }\DecValTok{4}\NormalTok{)}
\end{Highlighting}
\end{Shaded}

\begin{verbatim}
                      Black Non-Black - EB Non-Black - Unbalanced
CRIMETYPEDUI         0.1274         0.1274                 0.2790
CRIMETYPEOTHER       0.1632         0.1632                 0.1084
CRIMETYPEPERSONS     0.1838         0.1838                 0.1458
CRIMETYPEPROPERTY    0.2445         0.2445                 0.2588
OGS                  4.0601         4.0601                 3.1826
OGSQ                24.7268        24.7268                15.3210
RECMINYes            0.5303         0.5303                 0.7202
TRIALYes             0.9591         0.9591                 0.9858
PRS4/5               0.2720         0.2720                 0.1517
PRSNone              0.3863         0.3863                 0.5152
PRSREVOC/RFEL        0.0506         0.0506                 0.0228
MALEMale             0.8272         0.8272                 0.7545
DOSAGE              32.9122        32.9122                34.7886
DOSAGEQ           1213.2078      1213.2078              1342.1189
\end{verbatim}

\begin{Shaded}
\begin{Highlighting}[]
\FunctionTok{c}\NormalTok{(}\FunctionTok{summary}\NormalTok{(eb.pcs}\SpecialCharTok{$}\NormalTok{w), }\StringTok{"Std. Dev"} \OtherTok{=} \FunctionTok{sd}\NormalTok{(eb.pcs}\SpecialCharTok{$}\NormalTok{w))}
\end{Highlighting}
\end{Shaded}

\begin{verbatim}
       Min.     1st Qu.      Median        Mean     3rd Qu.        Max. 
 0.07609185  0.18832187  0.29698135  0.36708400  0.45848952 16.65560610 
   Std. Dev 
 0.28137031 
\end{verbatim}

\begin{Shaded}
\begin{Highlighting}[]
\NormalTok{weights }\OtherTok{\textless{}{-}} \FunctionTok{numeric}\NormalTok{(}\FunctionTok{nrow}\NormalTok{(pcs.cc))}
\ControlFlowTok{for}\NormalTok{ (i }\ControlFlowTok{in} \DecValTok{1}\SpecialCharTok{:}\FunctionTok{nrow}\NormalTok{(pcs.cc)) \{}
    \ControlFlowTok{if}\NormalTok{ (treatment[i]) \{}
\NormalTok{        weights[i] }\OtherTok{\textless{}{-}} \DecValTok{1}
\NormalTok{    \}}
    \ControlFlowTok{else}\NormalTok{ \{}
\NormalTok{        weights[i] }\OtherTok{\textless{}{-}}\NormalTok{ eb.out}\SpecialCharTok{$}\NormalTok{w[i]}
\NormalTok{    \}}
\NormalTok{\}}

\NormalTok{fit\_pcs\_unweighted }\OtherTok{\textless{}{-}} \FunctionTok{summary}\NormalTok{(}\FunctionTok{glm}\NormalTok{(INCAR }\SpecialCharTok{\textasciitilde{}}\NormalTok{ ., }
                          \AttributeTok{data =}\NormalTok{ pcs, }
                          \AttributeTok{family =} \FunctionTok{binomial}\NormalTok{(}\AttributeTok{link =} \StringTok{"logit"}\NormalTok{),}
                          \AttributeTok{model =} \ConstantTok{FALSE}\NormalTok{, }
                          \AttributeTok{y =} \ConstantTok{FALSE}\NormalTok{))}

\NormalTok{fit\_pcs\_weighted }\OtherTok{\textless{}{-}} \FunctionTok{summary}\NormalTok{(}\FunctionTok{glm}\NormalTok{(INCAR }\SpecialCharTok{\textasciitilde{}}\NormalTok{ ., }
                          \AttributeTok{data =}\NormalTok{ pcs.cc, }
                          \AttributeTok{weights =}\NormalTok{ weights,}
                          \AttributeTok{family =} \FunctionTok{binomial}\NormalTok{(}\AttributeTok{link =} \StringTok{"logit"}\NormalTok{),}
                          \AttributeTok{model =} \ConstantTok{FALSE}\NormalTok{, }
                          \AttributeTok{y =} \ConstantTok{FALSE}\NormalTok{))}
\end{Highlighting}
\end{Shaded}

\begin{verbatim}
Warning in eval(family$initialize): non-integer #successes in a binomial glm!
\end{verbatim}

\begin{Shaded}
\begin{Highlighting}[]
\FunctionTok{print}\NormalTok{(}\StringTok{"Unweighted Race Effect Estimate:"}\NormalTok{)}
\end{Highlighting}
\end{Shaded}

\begin{verbatim}
[1] "Unweighted Race Effect Estimate:"
\end{verbatim}

\begin{Shaded}
\begin{Highlighting}[]
\NormalTok{fit\_pcs\_unweighted}\SpecialCharTok{$}\NormalTok{coefficients[}\StringTok{"OFF\_RACERBLACK"}\NormalTok{,]}
\end{Highlighting}
\end{Shaded}

\begin{verbatim}
     Estimate    Std. Error       z value      Pr(>|z|) 
 2.268135e-01  6.511253e-03  3.483408e+01 7.418640e-266 
\end{verbatim}

\begin{Shaded}
\begin{Highlighting}[]
\FunctionTok{print}\NormalTok{(}\StringTok{"Weighted Race Effect Estimate:"}\NormalTok{)}
\end{Highlighting}
\end{Shaded}

\begin{verbatim}
[1] "Weighted Race Effect Estimate:"
\end{verbatim}

\begin{Shaded}
\begin{Highlighting}[]
\NormalTok{fit\_pcs\_weighted}\SpecialCharTok{$}\NormalTok{coefficients[}\StringTok{"OFF\_RACERBLACK"}\NormalTok{,]}
\end{Highlighting}
\end{Shaded}

\begin{verbatim}
     Estimate    Std. Error       z value      Pr(>|z|) 
 2.578608e-01  9.128484e-03  2.824794e+01 1.508926e-175 
\end{verbatim}

\hypertarget{incomplete-data}{%
\section{Incomplete Data}\label{incomplete-data}}

The impact of incomplete data is the next topic to tackle. We know that
CCA can be biased under several (untestable) assumptions for the
missingness mechanisms for logistic regression. Need to show that
missingness also has an impact on weights doubly impacting the
inferential results.

\begin{Shaded}
\begin{Highlighting}[]
\NormalTok{p }\OtherTok{\textless{}{-}} \DecValTok{1} \SpecialCharTok{{-}}\NormalTok{ (}\DecValTok{1} \SpecialCharTok{+} \FunctionTok{exp}\NormalTok{(}\SpecialCharTok{{-}}\NormalTok{(}\SpecialCharTok{{-}}\DecValTok{9} \SpecialCharTok{+}\NormalTok{ .}\DecValTok{5} \SpecialCharTok{*}\NormalTok{ pcs.cc}\SpecialCharTok{$}\NormalTok{DOSAGE }\SpecialCharTok{+}\NormalTok{ .}\DecValTok{5} \SpecialCharTok{*}\NormalTok{ pcs.cc}\SpecialCharTok{$}\NormalTok{OGS)))}\SpecialCharTok{\^{}{-}}\DecValTok{1}
\FunctionTok{boxplot}\NormalTok{(p)}
\end{Highlighting}
\end{Shaded}

\begin{figure}[H]

{\centering \includegraphics{entropy_balancing_files/figure-pdf/unnamed-chunk-8-1.pdf}

}

\end{figure}

\begin{Shaded}
\begin{Highlighting}[]
\NormalTok{mis }\OtherTok{\textless{}{-}} \FunctionTok{sample}\NormalTok{(}\FunctionTok{nrow}\NormalTok{(pcs.cc), }\AttributeTok{size =} \FunctionTok{ceiling}\NormalTok{(}\FloatTok{0.25} \SpecialCharTok{*} \FunctionTok{nrow}\NormalTok{(pcs.cc)), }\AttributeTok{prob =}\NormalTok{ p)}
\FunctionTok{str}\NormalTok{(mis)}
\end{Highlighting}
\end{Shaded}

\begin{verbatim}
 int [1:208637] 238658 215529 148383 356638 752797 68920 88571 434253 722658 330656 ...
\end{verbatim}

\begin{Shaded}
\begin{Highlighting}[]
\NormalTok{pcs.inc }\OtherTok{\textless{}{-}}\NormalTok{ pcs.cc}
\NormalTok{pcs.inc[mis, }\StringTok{"OFF\_RACER"}\NormalTok{] }\OtherTok{\textless{}{-}} \ConstantTok{NA}


\NormalTok{pcs.inc }\SpecialCharTok{\%\textgreater{}\%} 
    \FunctionTok{filter}\NormalTok{(}\SpecialCharTok{!}\FunctionTok{is.na}\NormalTok{(OFF\_RACER),}
           \SpecialCharTok{!}\FunctionTok{is.na}\NormalTok{(DOSAGE),}
           \SpecialCharTok{!}\FunctionTok{is.na}\NormalTok{(RECMIN),}
           \SpecialCharTok{!}\FunctionTok{is.na}\NormalTok{(PRS)) }\OtherTok{{-}\textgreater{}}\NormalTok{ pcs.cc2}
\NormalTok{pcs.cc2 }\SpecialCharTok{\%\textgreater{}\%}
    \FunctionTok{select}\NormalTok{(}\SpecialCharTok{{-}}\FunctionTok{c}\NormalTok{(INCAR, YEAR, COUNTY, OFF\_RACER)) }\OtherTok{{-}\textgreater{}}\NormalTok{ X}
\NormalTok{X }\OtherTok{\textless{}{-}} \FunctionTok{model.matrix}\NormalTok{(}\SpecialCharTok{\textasciitilde{}}\NormalTok{., }\AttributeTok{data =}\NormalTok{ X)[,}\SpecialCharTok{{-}}\DecValTok{1}\NormalTok{]}

\NormalTok{treatment }\OtherTok{\textless{}{-}} \FunctionTok{ifelse}\NormalTok{(pcs.cc2}\SpecialCharTok{$}\NormalTok{OFF\_RACER }\SpecialCharTok{==} \StringTok{"BLACK"}\NormalTok{, }\ConstantTok{TRUE}\NormalTok{, }\ConstantTok{FALSE}\NormalTok{)}

\NormalTok{eb.pcs2 }\OtherTok{\textless{}{-}} \FunctionTok{ebalance}\NormalTok{(}\AttributeTok{Treatment =}\NormalTok{ treatment, }
                   \AttributeTok{X =}\NormalTok{ X, }\AttributeTok{print.level =} \DecValTok{2}\NormalTok{)}
\end{Highlighting}
\end{Shaded}

\begin{verbatim}
Iteration 1 maximum deviation is = 19573597 
Iteration 2 maximum deviation is = 18358262 
Iteration 3 maximum deviation is = 16777100 
Iteration 4 maximum deviation is = 14647799 
Iteration 5 maximum deviation is = 11521672 
Iteration 6 maximum deviation is = 5521037 
Iteration 7 maximum deviation is = 108551 
Iteration 8 maximum deviation is = 84.07 
Converged within tolerance 
Converged within tolerance 
\end{verbatim}

\begin{Shaded}
\begin{Highlighting}[]
\NormalTok{c1 }\OtherTok{\textless{}{-}} \FunctionTok{apply}\NormalTok{(X[treatment,], }\DecValTok{2}\NormalTok{, mean)}


\NormalTok{c2 }\OtherTok{\textless{}{-}} \FunctionTok{apply}\NormalTok{(X[}\SpecialCharTok{!}\NormalTok{treatment,], }\DecValTok{2}\NormalTok{, weighted.mean, }\AttributeTok{w =}\NormalTok{ eb.pcs2}\SpecialCharTok{$}\NormalTok{w)}


\NormalTok{c3 }\OtherTok{\textless{}{-}} \FunctionTok{apply}\NormalTok{(X[}\SpecialCharTok{!}\NormalTok{treatment,], }\DecValTok{2}\NormalTok{, mean)}

\NormalTok{X.means }\OtherTok{\textless{}{-}} \FunctionTok{bind\_rows}\NormalTok{(c1, c2, c3) }\SpecialCharTok{\%\textgreater{}\%} \FunctionTok{as.data.frame}\NormalTok{()}
\FunctionTok{rownames}\NormalTok{(X.means) }\OtherTok{\textless{}{-}} \FunctionTok{c}\NormalTok{(}\StringTok{"Black"}\NormalTok{, }\StringTok{"Non{-}Black {-} EB"}\NormalTok{, }\StringTok{"Non{-}Black {-} Unbalanced"}\NormalTok{)}
\FunctionTok{round}\NormalTok{(}\FunctionTok{t}\NormalTok{(X.means), }\DecValTok{4}\NormalTok{)}
\end{Highlighting}
\end{Shaded}

\begin{verbatim}
                      Black Non-Black - EB Non-Black - Unbalanced
CRIMETYPEDUI         0.1341         0.1341                 0.2904
CRIMETYPEOTHER       0.1618         0.1618                 0.1096
CRIMETYPEPERSONS     0.1933         0.1933                 0.1531
CRIMETYPEPROPERTY    0.2357         0.2357                 0.2541
OGS                  4.4561         4.4561                 3.4369
OGSQ                28.9857        28.9857                17.5288
RECMINYes            0.4369         0.4369                 0.6647
TRIALYes             0.9513         0.9513                 0.9834
PRS4/5               0.3346         0.3346                 0.1822
PRSNone              0.3054         0.3054                 0.4560
PRSREVOC/RFEL        0.0654         0.0654                 0.0279
MALEMale             0.8267         0.8267                 0.7489
DOSAGE              36.8090        36.8090                38.4811
DOSAGEQ           1475.2901      1475.2901              1595.4856
\end{verbatim}

\begin{Shaded}
\begin{Highlighting}[]
\FunctionTok{c}\NormalTok{(}\FunctionTok{summary}\NormalTok{(eb.pcs2}\SpecialCharTok{$}\NormalTok{w), }\StringTok{"Std. Dev"} \OtherTok{=} \FunctionTok{sd}\NormalTok{(eb.pcs2}\SpecialCharTok{$}\NormalTok{w))}
\end{Highlighting}
\end{Shaded}

\begin{verbatim}
       Min.     1st Qu.      Median        Mean     3rd Qu.        Max. 
 0.08111835  0.17454630  0.26378869  0.35167721  0.43921475 11.91469178 
   Std. Dev 
 0.28622420 
\end{verbatim}

\begin{Shaded}
\begin{Highlighting}[]
\NormalTok{weights2 }\OtherTok{\textless{}{-}} \FunctionTok{numeric}\NormalTok{(}\FunctionTok{nrow}\NormalTok{(pcs.cc2))}
\ControlFlowTok{for}\NormalTok{ (i }\ControlFlowTok{in} \DecValTok{1}\SpecialCharTok{:}\FunctionTok{nrow}\NormalTok{(pcs.cc2)) \{}
    \ControlFlowTok{if}\NormalTok{ (treatment[i]) \{}
\NormalTok{        weights2[i] }\OtherTok{\textless{}{-}} \DecValTok{1}
\NormalTok{    \}}
    \ControlFlowTok{else}\NormalTok{ \{}
\NormalTok{        weights2[i] }\OtherTok{\textless{}{-}}\NormalTok{ eb.pcs2}\SpecialCharTok{$}\NormalTok{w[i]}
\NormalTok{    \}}
\NormalTok{\}}

\NormalTok{fit\_pcs\_unweighted2 }\OtherTok{\textless{}{-}} \FunctionTok{summary}\NormalTok{(}\FunctionTok{glm}\NormalTok{(INCAR }\SpecialCharTok{\textasciitilde{}}\NormalTok{ ., }
                          \AttributeTok{data =}\NormalTok{ pcs.cc2, }
                          \AttributeTok{family =} \FunctionTok{binomial}\NormalTok{(}\AttributeTok{link =} \StringTok{"logit"}\NormalTok{),}
                          \AttributeTok{model =} \ConstantTok{FALSE}\NormalTok{, }
                          \AttributeTok{y =} \ConstantTok{FALSE}\NormalTok{))}

\NormalTok{fit\_pcs\_weighted2 }\OtherTok{\textless{}{-}} \FunctionTok{summary}\NormalTok{(}\FunctionTok{glm}\NormalTok{(INCAR }\SpecialCharTok{\textasciitilde{}}\NormalTok{ ., }
                          \AttributeTok{data =}\NormalTok{ pcs.cc2, }
                          \AttributeTok{weights =}\NormalTok{ weights2,}
                          \AttributeTok{family =} \FunctionTok{binomial}\NormalTok{(}\AttributeTok{link =} \StringTok{"logit"}\NormalTok{),}
                          \AttributeTok{model =} \ConstantTok{FALSE}\NormalTok{, }
                          \AttributeTok{y =} \ConstantTok{FALSE}\NormalTok{))}
\end{Highlighting}
\end{Shaded}

\begin{verbatim}
Warning in eval(family$initialize): non-integer #successes in a binomial glm!
\end{verbatim}

\begin{Shaded}
\begin{Highlighting}[]
\FunctionTok{print}\NormalTok{(}\StringTok{"Unweighted Race Effect Estimate with Incomplete Data:"}\NormalTok{)}
\end{Highlighting}
\end{Shaded}

\begin{verbatim}
[1] "Unweighted Race Effect Estimate with Incomplete Data:"
\end{verbatim}

\begin{Shaded}
\begin{Highlighting}[]
\NormalTok{fit\_pcs\_unweighted2}\SpecialCharTok{$}\NormalTok{coefficients[}\StringTok{"OFF\_RACERBLACK"}\NormalTok{,]}
\end{Highlighting}
\end{Shaded}

\begin{verbatim}
     Estimate    Std. Error       z value      Pr(>|z|) 
 2.088674e-01  7.604425e-03  2.746657e+01 4.405814e-166 
\end{verbatim}

\begin{Shaded}
\begin{Highlighting}[]
\FunctionTok{print}\NormalTok{(}\StringTok{"Weighted Race Effect Estimate with Incomplete Data:"}\NormalTok{)}
\end{Highlighting}
\end{Shaded}

\begin{verbatim}
[1] "Weighted Race Effect Estimate with Incomplete Data:"
\end{verbatim}

\begin{Shaded}
\begin{Highlighting}[]
\NormalTok{fit\_pcs\_weighted2}\SpecialCharTok{$}\NormalTok{coefficients[}\StringTok{"OFF\_RACERBLACK"}\NormalTok{,]}
\end{Highlighting}
\end{Shaded}

\begin{verbatim}
     Estimate    Std. Error       z value      Pr(>|z|) 
 2.432750e-01  1.084716e-02  2.242754e+01 2.120633e-111 
\end{verbatim}

\hypertarget{references}{%
\section*{References}\label{references}}
\addcontentsline{toc}{section}{References}

\hypertarget{refs}{}
\begin{CSLReferences}{1}{0}
\leavevmode\vadjust pre{\hypertarget{ref-hainmueller2012}{}}%
Hainmueller J (2012)
\href{https://www.jstor.org/stable/41403737}{Entropy balancing for
causal effects: A multivariate reweighting method to produce balanced
samples in observational studies}. Political Analysis 20:25--46

\end{CSLReferences}



\end{document}
